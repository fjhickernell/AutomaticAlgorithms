% !TEX TS-program = PDFLatexBibtex
%&LaTeX
\documentclass[]{elsarticle}
\setlength{\marginparwidth}{0.5in}
\usepackage{amsmath,amssymb,amsthm,natbib,mathtools,bbm,extraipa,accents,graphicx}
\input FJHDef.tex

\newcommand{\fudge}{\fC}
\newcommand{\dtf}{\textit{\doubletilde{f}}}
\newtheorem{lem}{Lemma}
\theoremstyle{definition}
\newtheorem{defin}{Definition}
\newtheorem{algo}{Algorithm}
\newcommand{\cube}{[0,1)^d}
\renewcommand{\bbK}{\natzero^d}
\DeclareMathOperator{\trail}{trail}
\newcommand{\rf}{\mathring{f}}
\newcommand{\rnu}{\mathring{\nu}}
\allowdisplaybreaks

\begin{document}

\begin{frontmatter}

\title{Positive Definite Kernels Based on Chebychev Functions}
\author{Greg E. Fasshauer and Fred J. Hickernell}
\address{Room E1-208, Department of Applied Mathematics, Illinois Institute of Technology,\\ 10 W.\ 32$^{\text{nd}}$ St., Chicago, IL 60616}
\begin{abstract}
\end{abstract}

\begin{keyword}
%% keywords here, in the form: keyword \sep keyword

%% MSC codes here, in the form: \MSC code \sep code
%% or \MSC[2008] code \sep code (2000 is the default)

\end{keyword}
\end{frontmatter}

\section{Chebychev Polynomials and Bernoulli Polynomials}
Here we summarize some useful properties of some polynomials.  The first are the Chebychev polynomials:
\begin{align}
\text{Definition} \quad & T_k(x) = \cos(k \theta), \qquad \theta=\cos^{-1}(x),\nonumber \\
& \qquad \qquad -1 \le x \le 1, \quad 0 \le \theta \le \pi, \quad \text{\citep[18.5.1]{OlvEtal10a}} \\
\text{Degree} \quad &T_k(x) \text{ is a polynomial of degree } k \in \natzero  \nonumber \\
\text{Orthogonality} \quad & \int_{-1}^{1} \frac{T_k(x)T_{l}(x)}{\sqrt{1-x^2}} \, \dif x = \frac{\delta_{k,l}\pi}{2-\delta_{k,0}} ,  \qquad k,l \in \natzero \nonumber \\
& \qquad \qquad \qquad \qquad \qquad \qquad  \text{\citep[Table 18.3.1]{OlvEtal10a}}
\end{align}
Here are some properties of Bernoulli polynomials.
\begin{align}
\text{Fourier Series} \quad & B_{2m}(\abs x) = (-1)^{m+1} \frac{2 (2m)!} {(2 \pi)^{2m}} \sum_{k=1}^{\infty} \frac{\cos(2 \pi k x)}{k^{2m}}, \nonumber \\
& \qquad \qquad  \quad -1 \le x \le 1, \quad \text{\citep[24.8.1]{OlvEtal10a}} \\
\text{Degree} \quad &B_k(\abs x) \text{ is a polynomial of degree } k \in \natzero  \nonumber \end{align}

Define the orthogonal functions and the probability density function
\begin{gather}
\phi(x) = \sqrt{2-\delta_{k,0}} T_k(x) , \quad k \in \natzero, \qquad \rho(x) = \frac{1}{\pi\sqrt{1-x^2}}, \\
\int_{-1}^{1} \phi_k(x)\phi_{l}(x) \, \rho(x) \, \dif x = \delta_{k,l}, \qquad k,l \in \natzero.
\end{gather}
For a fixed positive-valued sequence of eigenvalues $\{\lambda_k\}_{k\in \natzero}$ satisfying $\sum_{k=0}^{\infty} \lambda_k =1$ define the kernel
\begin{equation} \label{kerneldef}
K(x,t) = \sum_{k=0}^{\infty} \lambda_k \phi_k(x)\phi_k(t)
\end{equation}
We will look at different choices of $\{\lambda_k\}_{k\in \natzero}$.

\section{Geometrically Decaying Eigenvalues}
For a fixed $b$, $0<b<1$, define the eigenvalues as
\begin{equation}
\lambda_0=1-a, \qquad \lambda_k = \frac{a (1-b) b^{k}}{b}, \qquad k \in \naturals
\end{equation}
Note that
\begin{align}
\nonumber
\phi_k(x)\phi_k(t) & = (2-\delta_{k,0}) T_k(x)T_k(t) \\
\nonumber
&=(2-\delta_{k,0}) \cos(k\theta)\cos(k \omega), \\
\nonumber
&\qquad \qquad \qquad x=\cos(\theta), \ t =\cos(\omega), \quad 0 \le \theta,\omega \le \pi \\
&=\frac{(2-\delta_{k,0})[\cos(k(\theta+\omega)) + \cos(k(\theta-\omega))]}{2} \label{sumcos}\\
&=\frac{2-\delta_{k,0}}{4}[\me^{\sqrt{-1}k(\theta+\omega)} + \me^{\sqrt{-1}k(\theta-\omega) } \nonumber \\
& \qquad \qquad + \me^{-\sqrt{-1}k(\theta-\omega)} + \me^{-\sqrt{-1}k(\theta+\omega}]
\label{sumcompexp}
\end{align}
Since
\begin{align*}
\sum_{k=0}^{\infty} \lambda_k (2-\delta_{k,0}) \me^{\sqrt{-1}k\alpha}
&= 1-a + \frac{2a(1-b)}{b}\sum_{k=1}^{\infty} b^k \me^{\sqrt{-1}k\alpha} \\
&= 1-a + \frac{2a(1-b)\me^{\sqrt{-1}\alpha}} {1-b\me^{\sqrt{-1}\alpha}},
\end{align*}
it follows that
\begin{align*}
K(x,t) &= \sum_{k=0}^{\infty} \lambda_k \phi_k(x)\phi_k(t) \\
&= 1-a + \frac{a(1-b)}{2}\left[ \frac{\me^{\sqrt{-1}(\theta+\omega)}}{1-b\me^{\sqrt{-1}(\theta+\omega)}} + \frac{\me^{-\sqrt{-1}(\theta+\omega)}}{1-b\me^{-\sqrt{-1}(\theta+\omega)}} \right . \\
& \qquad \qquad \left . + \frac{\me^{\sqrt{-1}(\theta-\omega)}}{1-b\me^{\sqrt{-1}(\theta-\omega)}} + \frac{\me^{-\sqrt{-1}(\theta-\omega)}}{1-b\me^{-\sqrt{-1}(\theta-\omega)}}  \right] \\
&=  1-a + \frac{a(1-b)}{2} \left[ \frac{\me^{\sqrt{-1}(\theta+\omega)} + \me^{-\sqrt{-1}(\theta+\omega)} - 2b } {1+b^2 -b[\me^{\sqrt{-1}(\theta+\omega)} + \me^{-\sqrt{-1}(\theta+\omega)}] } \right . \\
& \qquad \qquad \left . + \frac{\me^{\sqrt{-1}(\theta-\omega)}+\me^{-\sqrt{-1}(\theta-\omega)} - 2b } {1+b^2 -b[\me^{\sqrt{-1}(\theta-\omega)} + \me^{-\sqrt{-1}(\theta-\omega)}] }  \right] \\
&= 1-a + a(1-b) \left[\frac{\cos(\theta+\omega)-b} {1+b^2 -2b\cos(\theta+\omega) } + \frac{\cos(\theta-\omega)-b} {1+b^2 -2b\cos(\theta-\omega) }  \right] \\
&= 1-a + a(1-b) \left[\frac{xt-\sqrt{(1-x^2)(1-t^2)}-b} {1+b^2 -2b[xt-\sqrt{(1-x^2)(1-t^2)}] } \right. \\
& \qquad \qquad \left .  + \frac{xt+\sqrt{(1-x^2)(1-t^2)}-b} {1+b^2 -2b[xt+\sqrt{(1-x^2)(1-t^2)}] }  \right].
\end{align*}
Furthermore
\begin{align*}
\int_{-1}^1K(x,x_i)K(x,x_j)\rho(x)dx &= \int_{-1}^1\left(\sum_{k=0}^\infty\lambda_k\phi_k(x)\phi_k(x_i)\right)\left(\sum_{k=0}^\infty\lambda_k\phi_k(x)\phi_k(x_j)\right)\rho(x)dx\\
&= \sum_{k=0}^\infty\lambda_k^2\phi_k(x_i)\phi_k(x_j)\\
&= (1-a)^2 + a^2(1-b)^2\left[\frac{xt-\sqrt{(1-x^2)(1-t^2)}-b^2} {1+b^4 -2b^2[xt-\sqrt{(1-x^2)(1-t^2)}] } \right. \\
& \qquad \qquad \left .  + \frac{xt+\sqrt{(1-x^2)(1-t^2)}-b^2} {1+b^4 -2b^2[xt+\sqrt{(1-x^2)(1-t^2)}] }  \right].
\end{align*}

\section{Algebraically Decaying Eigenvalues}
For a fixed $a$, with $0<a\le 1$ and a fixed $b$, $b \in \naturals$, define the eigenvalues as
\begin{equation}
\lambda_0=1-a, \qquad \lambda_k = \frac{a}{\zeta(2b)k^{2b}}, \quad k \in \naturals
\end{equation}
For this choice of algebraically decaying eigenvalues and for $-1 \le \alpha \le 1$ it follows that.
\begin{align*}
\sum_{k=0}^{\infty} \lambda_k (2-\delta_{k,0}) \cos(2 \pi k\alpha)
&= 1-a + \frac{2a}{\zeta(2b)}\sum_{k=1}^{\infty} \frac{\cos(2 \pi k\alpha)}{k^{2b}}\\
&= 1-a + \frac{2a}{\zeta(2b)} (-1)^{b+1} \frac{(2 \pi)^{2b}} {2 (2b)!} B_{2b}(\abs\alpha) \\
&= 1-a + \frac{a (-1)^{b+1}(2 \pi)^{2b}}{(2b)!\zeta(2b)} B_{2b}(\abs\alpha).
\end{align*}
So it follows from \eqref{sumcos} that
\begin{align*}
K(x,t) &= \sum_{k=0}^{\infty} \lambda_k \phi_k(x)\phi_k(t) \\
&= 1-a + \frac{a (-1)^{b+1}(2 \pi)^{2b}}{2(2b)!\zeta(2b)} [B_{2b}(\abs{\theta + \omega}/(2\pi)) + B_{2b}(\abs{\theta - \omega}/(2\pi))]\\
&= 1-a + \frac{a (-1)^{b+1}(2 \pi)^{2b}}{2(2b)!\zeta(2b)} \left[B_{2b}\left(\frac{\abs{\cos^{-1}(x) + \cos^{-1}(t)}}{2\pi}\right) \right .  \\
& \qquad \qquad \left. + B_{2b}\left(\frac{\abs{\cos^{-1}(x) - \cos^{-1}(t)}}{2\pi}\right) \right].
\end{align*}
Furthermore
\begin{align*}
\int_{-1}^1K(x,x_i)K(x,x_j)\rho(x)dx &= \sum_{k=0}^\infty\lambda_k^2\phi_k(x_i)\phi_k(x_j)\\
&= (1-a)^2-\frac{a^2(2 \pi)^{4b}}{2(4b)!\zeta^2(2b)} \left[B_{4b}\left(\frac{\abs{\cos^{-1}(x_i) + \cos^{-1}(x_j)}}{2\pi}\right) \right .  \\
& \qquad \qquad \left. + B_{4b}\left(\frac{\abs{\cos^{-1}(x_i) - \cos^{-1}(x_j)}}{2\pi}\right) \right].
\end{align*}

For example, take $b=1$ and we have
$$K(x,t)=1-a+6a\left(B_2\left(\frac{\abs{\cos^{-1}(x)+\cos^{-1}(t)}}{2\pi}\right)+B_2\left(\frac{\abs{\cos^{-1}(x)-\cos^{-1}(t)}}{2\pi}\right)\right),$$
where
$$B_2(s)=s^2-s+1/6.$$
We can proceed to compute that
\begin{align*}
\int_{-1}^1K(x,x)\rho(x)dx&=\int_{-1}^1\left(1+6aB_2\left(\frac{\cos^{-1}(x)}\pi\right)\right)\rho(x)dx\\
&=1-6a\int_0^\pi B_2\left(\frac\theta\pi\right)\frac1{\pi\sqrt{1-\cos^2\theta}}\dif\cos\theta\\
&=1+6a\int_0^1 B_2(s)ds=1
\end{align*}
\bibliographystyle{amsplain}
\bibliography{FJH22,FJHown22}
\end{document}
