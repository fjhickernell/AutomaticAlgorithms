\documentclass[xcolor=dvipsnames]{beamer}
%\usecolortheme[RGB={195,0,0}]{structure}
\usepackage{graphicx}
\usepackage{amsmath}
\usepackage{amsfonts}   % if you want the fonts
\usepackage{amssymb,verbatim}    % if you want extra symbols
\usepackage{epstopdf}

%\newcommand{\R}{{\mathbb R}}
%\newcommand{\N}{{\mathbb N}}
%\newcommand{\Z}{{\mathbb Z}}
%\newcommand{\TT}{{\mathbb T}}
%\newcommand{\Oh}{{\mathcal O}}
\newcommand{\cl}{{\mathcal L}}
%\newcommand{\I}{{\mathcal I}}
%\newcommand{\bi}{{\boldsymbol{i}}}
\newcommand{\bc}{{\boldsymbol{c}}}
\newcommand{\bx}{{\boldsymbol{x}}}
\newcommand{\bX}{{\boldsymbol{X}}}
\newcommand{\by}{{\boldsymbol{y}}}
\newcommand{\bH}{{\boldsymbol{H}}}
\newcommand{\bI}{{\boldsymbol{I}}}
\newcommand{\bk}{{\boldsymbol{k}}}
\newcommand{\bxi}{{\boldsymbol{\xi}}}
%\newcommand{\bz}{{\boldsymbol{z}}}
\newcommand{\bw}{{\boldsymbol{\omega}}}
\newcommand{\balpha}{{\boldsymbol{\alpha}}}
\newcommand{\bgamma}{{\boldsymbol{\gamma}}}
%\newcommand\expect{\mathbb{E}}
%\newcommand\uset{{\mathfrak{u}}}
%\newcommand\vset{{\mathfrak{v}}}
%\newcommand\wset{{\mathfrak{w}}}
%%\theoremstyle{plain}
%%\newtheorem{theorem}{Theorem}
%%\newtheorem{corollary}{Corollary}
%%\newtheorem{lemma}{Lemma}
%
%%\theoremstyle{definition}
%%\newtheorem{remark}{Remark}
%%\newtheorem{definition}{Definition}
%
\def\il{\left<}
\def\ir{\right>}
%\def\d{\delta }
\def\a{\alpha }
\def\b{\beta }
\def\e{\varepsilon }
%\def\phi{\varphi }
%\def\epsilon{\varepsilon }
\def\g{\gamma }
%\def\gmin{\gamma_1^*}
%\def\rho{\varrho }
%\def\cost{{\rm cost}}
%\def\Z{{\bf Z}}
%\def\comp{{\rm comp\, }}
%\def\no{\noindent}
%\def\lin{{\rm lin}}
%\def\sgn{{\rm sgn\,  }}
%\def\l{{\ell }}
%\def\disc{{\rm disc\, }}
%\def\wtavp{{ \widetilde {\rm av}^{*}_p (n) }}
%\def\uu{\overline{u}}
%\def\lstd{{\Lambda^{\rm std}}}
%\def\lall{{\Lambda^{\rm all}}}
%\newcount\refnum
%\refnum = 0
%\def\refx{\global\advance\refnum by 1 {\the\refnum . \ }}
%
%\def\({\biggl( }
%\def\){\biggr) }
%\def\ma{{\rm max}}
%\def\abs{{\rm abs}}
%\def\rel{{\rm rel}}
%\def\Z{{\bf Z}}
%\def\Q{{\bf Q}}
%\def\C{{\bf C}}
%\def\non{{\rm non}}
%\def\wt{\widetilde }
%\def\lo{{\rm log}}
%\def\c{{\bf c}}
%\def\cost{{\rm cost\, }}
%\def\ari{{\rm ari}}
%\def\info{{\rm info}}
%\def\half{{\textstyle \frac12}}
%
%\newcommand{\tlambda}{\tilde{\lambda}}
%\newcommand{\cg}{\mathcal{H}}
\newcommand{\cC}{\mathcal{C}}
\newcommand{\cH}{\mathcal{H}}
\newcommand{\cP}{\mathcal{P}}
\newcommand{\cX}{\mathcal{X}}
\newcommand{\cY}{\mathcal{Y}}
\newcommand{\indi}{\boldsymbol 1}
\newcommand{\mH}{\mathsf{H}}
%\newcommand{\ba}{\boldsymbol{a}}
%\newcommand{\bc}{\boldsymbol{c}}
%\newcommand{\bk}{\boldsymbol{k}}
\newcommand{\mK}{\mathsf{K}}
\newcommand{\bj}{\boldsymbol{j}}
%\newcommand{\bn}{\boldsymbol{n}}
\newcommand{\bt}{\boldsymbol{t}}
%\newcommand{\by}{\boldsymbol{y}}
%\newcommand{\bzero}{\boldsymbol{0}}
%\newcommand{\tK}{\widetilde{K}}
%\newcommand{\hK}{\widehat{K}}
%\newcommand{\te}{\tilde{e}}
%\newcommand{\tf}{\tilde{f}}
%\newcommand{\tbeta}{\tilde{\beta}}
%\newcommand{\tgamma}{\tilde{\gamma}}
%\newcommand{\natzero}{\mathbb{N}_{0}}
\newcommand{\naturals}{\mathbb{N}}
\newcommand{\reals}{\mathbb{R}}
\newcommand{\dif}{\rm d}
\newcommand{\tr}{\rm{tr}}
%\newcommand{\texta}{\text{a}}
%\newcommand{\textw}{\text{w}}
%\newcommand{\textx}{\text{x}}
%\newcommand{\textstr}{\text{str}}
%\newcommand{\textstrd}{\text{str-}}
%\newcommand{\ip}[3][{}]{\ensuremath{\left \langle #2, #3 \right \rangle_{#1}}}
%\DeclareMathOperator{\wcerr}{error^w}
%\DeclareMathOperator{\acerr}{error^a}
%\DeclareMathOperator{\xcerr}{error^x}
%\DeclareMathOperator{\optdes}{optdes}
%\DeclareMathOperator{\cov}{cov}
%\def\e{\varepsilon}
%\newcommand{\dif}{\textup{d}}
%\newcommand{\me}{\textup{e}}
%\newcommand{\mi}{\textup{i}}
%\newcommand{\stdall}{\vartheta}
%\newcommand{\absnorm}{\psi}
%\newcommand{\tC}{\tilde{C}}

\usetheme{CambridgeUS}
\title[Guaranteed Kernel Interpolation]{Interpolation Using Kernel Methods With Guaranteed Error Bounds} 
\author{Xuan Zhou}
\institute[IIT]{{Department of Applied Mathematics, Illinois Institute of Technology}\\
[3ex]
{Joint Work with Fred J. Hickernell}}

\begin{document}
\begin{center}
\titlepage
\end{center}



\begin{frame}{Scattered Data Interpolation}
\begin{itemize}
\item We want to approximate a real-valued function $f\in\cH_d\subset L^2(\reals^d,\rho_d)$, where $\cH_d$ is a reproducing kernel Hilbert space with some symmetric positive definite kernel $K_d(\bx,\bt)$, $\bx,\bt\in\reals^d$.
and $\rho_d$ is some probability density function.
\item Given the data $(\bx_i,y_i)$, $i=1, \ldots, n$, where $y_i=f(\bx_i)$, The spline algorithm constructs the approximation as
$$(A_nf)(\bx) = \sum_{i=1}^n c_i K(\bx,\bx_i) = \bk^T(\bx) \bc,$$
where $\bc = \left( c_i \right)_{i=1}^n$, $\bk(\bx) = \left( K(\bx,\bx_i) \right)_{i=1}^n$. A good choice of $\bc$ is
\[
\bc = \mK^{-1} \by, \qquad \text{where } \mK = \left( K(\bx_i,\bx_j) \right)_{i,j=1}^n, \quad \by = \left( y_i \right)_{i=1}^n.
\]
\end{itemize}
\end{frame}

\begin{frame}{Error Estimation via Power Functions}
Following the power function approach, we have
\begin{align*}
\sup_{0\ne f\in\cH}\frac{\|f-A_nf\|_{\cl_2}}{\|f\|_\cH^2}&\le\int_\cX\sup_{0\ne f\in\cH}\frac{|f(\bx)-(A_nf)(\bx)|^2}{\|f\|_\cH^2}\rho(\bx)\dif\bx\\
&\le\int_\cX \left(K(\bx,\bx)-\bk^T(\bx)\mK^{-1}\bk(\bx)\right)\rho(\bx)\dif\bx\\
&=\int_\cX K(\bx,\bx)\rho(\bx)\dif\bx-\tr\left(\mK^{-1}\widetilde\mK\right)\\
&=:h^2(n),
\end{align*}
where
$$\widetilde\mK=\int_\cX\bk(\bx)\bk^T(\bx)\rho(\bx)\dif\bx.$$
\end{frame}

\begin{frame}{Weaker Norm}
Let $\|\cdot\|_{\widetilde\cH}$ be a weaker norm defined by $\|f\|_{\widetilde\cH}=\|Tf\|_{\cl_2}$, where $T:\cH\to\cl_2(\cX,\rho)$ is a bounded linear operator and the linear functional $T_\bx:f\mapsto (Tf)(\bx)$ is also bounded. Then we can estimate the weaker norm by
\begin{align*}
\widetilde H_n^2(f)&:=\|T(A_nf)\|_{\cl_2}^2=\|\bc^T\bxi\|_{\cl_2}^2=\bc^T\widetilde\mH\bc,
\end{align*}
where
$$\bxi=(\xi(\cdot,\bx_i))_{i=1}^n=(TK(\cdot,\bx_i))_{i=1}^n,$$
and
$$\widetilde\mH=\int_\cX\bxi(\bx)\bxi^T(\bx)\rho(\bx)\dif\bx.$$
\end{frame}

\begin{frame}{Functions in a Cone}
Let $\cC_\tau=\left\{f\in\cH\left|\|f\|_\cH\le\tau\|f\|_{\widetilde\cH}\right.\right\}$. Then for all $f\in\cC_\tau$,
$$\left|\|f\|_{\widetilde\cH}-\widetilde H_n(f)\right|\le\tilde h(n)\|f\|_\cH\le\tau\tilde h(n)\|f\|_{\widetilde\cH},$$
where
$$\tilde h^2(n)=\int_\cX T_\bx^\cdot T_\bx^{\cdot\cdot}K(\cdot,\cdot\cdot)\rho(\bx)\dif\bx-\tr\left(\mK^{-1}\widetilde\mH\right).$$
If $1-\tau\tilde h(n)>0$, this in turn gives an upper bound
$$\|f\|_\cH\le\frac{\tau\widetilde H_n(f)}{1-\tau\tilde h(n)}.$$
\end{frame}

\begin{frame}
  \frametitle{Problems With Power Function Estimation}
  \begin{itemize}
  \item The matrices involved in the estimation of the error bound are often ill-conditioned.
  \item The estimated error bound is conservative in the sense that it does not converge as fast as the actual error.  
  \end{itemize}
  It would be nice if we could find alternative error bounds that circumvent such limitations.\end{frame}


\end{document}
