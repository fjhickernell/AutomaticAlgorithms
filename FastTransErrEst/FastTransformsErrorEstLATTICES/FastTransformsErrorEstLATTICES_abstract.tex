\documentclass[]{elsarticle}
\setlength{\marginparwidth}{0.5in}
\usepackage{amsmath,amssymb,amsthm,natbib,mathtools,bbm,extraipa,accents,graphicx}
\input FJHDef.tex

\newcommand{\fudge}{\fC}
\newcommand{\dtf}{\textit{\doubletilde{f}}}
\newtheorem{lem}{Lemma}
\theoremstyle{definition}
\newtheorem{defin}{Definition}
\newtheorem{algo}{Algorithm}
\newcommand{\cube}{[0,1)^d}
\renewcommand{\bbK}{\natzero^d}
\DeclareMathOperator{\trail}{trail}
\newcommand{\rf}{\mathring{f}}
\newcommand{\rnu}{\mathring{\nu}}


\begin{document}

\begin{frontmatter}

\title{Error estimation for multidimensional integration based on rank-1 Lattices}
\author{Llu\'{\i}s Antoni Jim\'{e}nez Rugama}
%\address{Room E1-120, Department of Applied Mathematics, Illinois Institute of Technology,\\ 10 W.\ 32$^{\text{nd}}$ St., Chicago, IL 60616}
\begin{abstract}

Many applications in physics, finance or mechanics require high dimensional integration.  This is often performed using Monte Carlo or quasi-Monte Carlo methods since the error of the estimate does not depend on the dimension.  However, there is the problem of determining the sample size needed to satisfy an error tolerance.  Methods of error estimation for quasi-Monte Carlo methods tend to be based on heuristics and lack theoretical guarantees.

This project aims to build an efficient algorithm based on rank-1 lattice rules capable of providing the solution to high dimensional problems with guarantees. The most important assumption is that the integrands lie inside a given cone. The properties of the cone allow us to bound the error. The novel use of cones to guarantee error estimates provides a new perspective that can be extended to other algorithms.
\end{abstract}

\end{frontmatter}


\end{document}
