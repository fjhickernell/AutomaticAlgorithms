%MCQMC, April 6-11, 2014
%Requires graphics files
%  sob256pts.eps, scrsob256pts.eps, PlotFWTCoefUse256.eps KeistercubSobolErrTime.eps

%%%%%%%%%%%%%%%% Springer %%%%%%%%%%%%%%%%%%%%%%%%%%%%%%%%%%
\documentclass[graybox,footinfo]{svmult}

\smartqed
\usepackage{mathptmx}       % selects Times Roman as basic font
\usepackage{helvet}         % selects Helvetica as sans-serif font
\usepackage{courier}        % selects Courier as typewriter font
\usepackage{type1cm}        % activate if the above 3 fonts are
                            % not available on your system
\usepackage{graphicx}       % standard LaTeX graphics tool
                            % when including figure files

\usepackage{array,colortbl}
\usepackage{amsmath,amsfonts,amssymb,bm} % no amsthm, Springer defines Theorem, Lemma, etc themselves
%\usepackage[mathx]{mathabx}
\DeclareFontFamily{U}{mathx}{\hyphenchar\font45}
\DeclareFontShape{U}{mathx}{m}{n}{
      <5> <6> <7> <8> <9> <10>
      <10.95> <12> <14.4> <17.28> <20.74> <24.88>
      mathx10
      }{}
\DeclareSymbolFont{mathx}{U}{mathx}{m}{n}
\DeclareFontSubstitution{U}{mathx}{m}{n}
\DeclareMathAccent{\widecheck}      {0}{mathx}{"71}

% Note that Springer defines the following already:
%
% \D upright d for differential d
% \I upright i for imaginary unit
% \E upright e for exponential function
% \tens depicts tensors as sans serif upright
% \vec depicts vectors as boldface characters instead of the arrow accent
%
% Additionally we throw in the following common used macro's:
\newcommand{\Z}{\mathbb{Z}} % integers
\newcommand{\C}{\mathbb{C}} % complex numbers
\newcommand{\R}{\mathbb{R}} % reals
\newcommand{\N}{\mathbb{N}} % natural numbers {1, 2, ...}
\newcommand{\Q}{\mathbb{Q}} % rationals
\newcommand{\F}{\mathbb{F}} % field, finite field
\newcommand{\floor}[1]{\left\lfloor #1 \right\rfloor} % floor
\newcommand{\ceil}[1]{\left\lceil #1 \right\rceil}    % ceil
\newcommand{\rd}{\,\mathrm{d}} % differential symbol for use in integrals
% vectors as boldsymbols:
\newcommand{\bszero}{\boldsymbol{0}} % vector of zeros
\newcommand{\bsone}{\boldsymbol{1}}  % vector of ones
\newcommand{\bst}{\boldsymbol{t}}    % vector t
\newcommand{\bsu}{\boldsymbol{u}}    % vector u
\newcommand{\bsv}{\boldsymbol{v}}    % vector v
\newcommand{\bsw}{\boldsymbol{w}}    % vector w
\newcommand{\bsx}{\boldsymbol{x}}    % vector x
\newcommand{\bsy}{\boldsymbol{y}}    % vector y
\newcommand{\bsz}{\boldsymbol{z}}    % vector z
\newcommand{\bsDelta}{\boldsymbol{\Delta}}    % vector \Delta
% sets as Euler fraks:
\newcommand{\setu}{\mathfrak{u}}
\newcommand{\setv}{\mathfrak{v}}
% indicator boldface 1:
\DeclareSymbolFont{bbold}{U}{bbold}{m}{n}
\DeclareSymbolFontAlphabet{\mathbbold}{bbold}
\newcommand{\ind}{\mathbbold{1}}


\usepackage{microtype} % good font tricks

\usepackage[colorlinks=true,linkcolor=black,citecolor=black,urlcolor=black]{hyperref}
\urlstyle{same}
\usepackage{bookmark}
\pdfstringdefDisableCommands{\def\and{, }}
\makeatletter % to avoid hyperref warnings:
  \providecommand*{\toclevel@author}{999}
  \providecommand*{\toclevel@title}{0}
\makeatother

%Fred's additions
%\usepackage{amsmath,datetime,xmpmulti,mathtools,bbm,array,booktabs,alltt,xspace,mathabx,tikz,pifont,graphicx}
\usepackage{mathtools,array,booktabs}
\DeclareMathOperator{\ok}{ok}

%%%%%%%%%%%%%%%%%%%%%%%%%%%%%%%%%%%%%%%%%%%%%%%%%%%%%%%%%%%%%%%%%%%%%%%%%%%%%%%%%%%%%%%%%
\begin{document}
\spnewtheorem{algo}{Algorithm}{\bf}{\rm}
\newcommand{\bsi}{\boldsymbol{i}}    % vector i
\newcommand{\bsk}{\boldsymbol{k}}    % vector k
\newcommand{\bsl}{\boldsymbol{l}}    % vector l
\newcommand{\bsr}{\boldsymbol{r}}    % vector r
\newcommand{\bsnu}{\boldsymbol{\nu}}    % vector l
\newcommand{\cc}{\mathcal{C}}
\newcommand{\cl}{\mathcal{L}}
\newcommand{\cn}{\mathcal{N}}
\newcommand{\Order}{\mathcal{O}}
\newcommand{\cp}{\mathcal{P}}
\newcommand{\cx}{\mathcal{X}}
\newcommand{\natm}{\N_{0,m}}
\newcommand{\cube}{[0,1)^d}
\newcommand{\hf}{\hat{f}}
\newcommand{\rf}{\mathring{f}}
\newcommand{\tf}{\tilde{f}}
\newcommand{\hg}{\hat{g}}
\newcommand{\hI}{\hat{I}}
\newcommand{\tvk}{\tilde{\bsk}}
\newcommand{\hS}{\widehat{S}}
\newcommand{\tS}{\widetilde{S}}
\newcommand{\wcS}{\widecheck{S}}
\newcommand{\rnu}{\mathring{\nu}}
\newcommand{\tnu}{\widetilde{\nu}}
\newcommand{\hnu}{\widehat{\nu}}
\newcommand{\homega}{\widehat{\omega}}
\newcommand{\wcomega}{\mathring{\omega}}
\newcommand{\fC}{\mathfrak{C}}
\newcommand{\nodes}{\{\bsz_i\}_{i=0}^{\infty}}
\newcommand{\nodesn}{\{\bsz_i\}_{i=0}^{n-1}}
\newcommand{\norm}[1]{\ensuremath{\left \lVert #1 \right \rVert}} \newcommand{\bigabs}[1]{\ensuremath{\bigl \lvert #1 \bigr \rvert}}
\newcommand{\Bigabs}[1]{\ensuremath{\Bigl \lvert #1 \Bigr \rvert}}
\newcommand{\biggabs}[1]{\ensuremath{\biggl \lvert #1 \biggr \rvert}}
\newcommand{\Biggabs}[1]{\ensuremath{\Biggl \lvert #1 \Biggr \rvert}}
\newcommand{\ip}[3][{}]{\ensuremath{\left \langle #2, #3 \right \rangle_{#1}}}



\title*{Adaptive Multidimensional Integration Based on Rank-1 Lattices}
\author{Fred J. Hickernell \and Llu\'is Antoni Jim\'enez Rugama}
\institute{Fred J. Hickernell \and Llu\'is Antoni
\at Department of Applied Mathematics,  Illinois Institute of Technology, 10 W. 32$^{\text{nd}}$ Street, E1-208, Chicago, IL 60616, USA
\email{hickernell@iit.edu}, \email{ljimene1@hawk.iit.edu}}
\maketitle

\abstract{}

\section{Introduction}
Quasi-Monte Carlo methods are used to approximate multidimensional integration in $\cube$. This 
\section{Digital Sequences}

\section{Walsh Series} \label{WaveWalshsec}

\section{Error Estimation and an Adaptive Cubature Algorithm} \label{ErrEstsec}

\subsection{Wavenumber Map} \label{wavenummapsec}

\subsection{An Adaptive Cubature Algorithm and Its Cost}

\section{Numerical Experiments} \label{numexpsec}


\section{Discussion}


\begin{acknowledgement}
\end{acknowledgement}

\bibliographystyle{spmpsci}
\bibliography{FJH22,FJHown22}

\section*{Appendix:  Fast Computation of the Discrete Walsh Transform}



\end{document}