%MCQMC, April 6-11, 2014
%Requires graphics files
%  sob256pts.eps, scrsob256pts.eps, PlotFWTCoefUse256.eps KeistercubSobolErrTime.eps

%%%%%%%%%%%%%%%% Springer %%%%%%%%%%%%%%%%%%%%%%%%%%%%%%%%%%
\documentclass[graybox,footinfo]{svmult}

\smartqed
\usepackage{mathptmx}       % selects Times Roman as basic font
\usepackage{helvet}         % selects Helvetica as sans-serif font
\usepackage{courier}        % selects Courier as typewriter font
\usepackage{type1cm}        % activate if the above 3 fonts are
                            % not available on your system
\usepackage{graphicx}       % standard LaTeX graphics tool
                            % when including figure files

\usepackage{array,colortbl}
\usepackage{amsmath,amsfonts,amssymb,bm} % no amsthm, Springer defines Theorem, Lemma, etc themselves
%\usepackage[mathx]{mathabx}
\DeclareFontFamily{U}{mathx}{\hyphenchar\font45}
\DeclareFontShape{U}{mathx}{m}{n}{
      <5> <6> <7> <8> <9> <10>
      <10.95> <12> <14.4> <17.28> <20.74> <24.88>
      mathx10
      }{}
\DeclareSymbolFont{mathx}{U}{mathx}{m}{n}
\DeclareFontSubstitution{U}{mathx}{m}{n}
\DeclareMathAccent{\widecheck}      {0}{mathx}{"71}

% Note that Springer defines the following already:
%
% \D upright d for differential d
% \I upright i for imaginary unit
% \E upright e for exponential function
% \tens depicts tensors as sans serif upright
% \vec depicts vectors as boldface characters instead of the arrow accent
%
% Additionally we throw in the following common used macro's:
\newcommand{\Z}{\mathbb{Z}} % integers
\newcommand{\C}{\mathbb{C}} % complex numbers
\newcommand{\R}{\mathbb{R}} % reals
\newcommand{\N}{\mathbb{N}} % natural numbers {1, 2, ...}
\newcommand{\Q}{\mathbb{Q}} % rationals
\newcommand{\F}{\mathbb{F}} % field, finite field
\newcommand{\floor}[1]{\left\lfloor #1 \right\rfloor} % floor
\newcommand{\ceil}[1]{\left\lceil #1 \right\rceil}    % ceil
\newcommand{\rd}{\,\mathrm{d}} % differential symbol for use in integrals
% vectors as boldsymbols:
\newcommand{\bszero}{\boldsymbol{0}} % vector of zeros
\newcommand{\bsone}{\boldsymbol{1}}  % vector of ones
\newcommand{\bst}{\boldsymbol{t}}    % vector t
\newcommand{\bsu}{\boldsymbol{u}}    % vector u
\newcommand{\bsv}{\boldsymbol{v}}    % vector v
\newcommand{\bsw}{\boldsymbol{w}}    % vector w
\newcommand{\bsx}{\boldsymbol{x}}    % vector x
\newcommand{\bsy}{\boldsymbol{y}}    % vector y
\newcommand{\bsz}{\boldsymbol{z}}    % vector z
\newcommand{\bsDelta}{\boldsymbol{\Delta}}    % vector \Delta
% sets as Euler fraks:
\newcommand{\setu}{\mathfrak{u}}
\newcommand{\setv}{\mathfrak{v}}
% indicator boldface 1:
\DeclareSymbolFont{bbold}{U}{bbold}{m}{n}
\DeclareSymbolFontAlphabet{\mathbbold}{bbold}
\newcommand{\ind}{\mathbbold{1}}


\usepackage{microtype} % good font tricks

\usepackage[colorlinks=true,linkcolor=black,citecolor=black,urlcolor=black]{hyperref}
\urlstyle{same}
\usepackage{bookmark}
\pdfstringdefDisableCommands{\def\and{, }}
\makeatletter % to avoid hyperref warnings:
  \providecommand*{\toclevel@author}{999}
  \providecommand*{\toclevel@title}{0}
\makeatother

%Fred's additions
%\usepackage{amsmath,datetime,xmpmulti,mathtools,bbm,array,booktabs,alltt,xspace,mathabx,tikz,pifont,graphicx}
\usepackage{mathtools,array,booktabs}
\DeclareMathOperator{\ok}{ok}

%%%%%%%%%%%%%%%%%%%%%%%%%%%%%%%%%%%%%%%%%%%%%%%%%%%%%%%%%%%%%%%%%%%%%%%%%%%%%%%%%%%%%%%%%
\begin{document}
\spnewtheorem{algo}{Algorithm}{\bf}{\rm}
\newcommand{\bsa}{\boldsymbol{a}}    %%% vector a
\newcommand{\bsh}{\boldsymbol{h}}    %%% vector h
\newcommand{\bsi}{\boldsymbol{i}}    % vector i
\newcommand{\bsj}{\boldsymbol{j}}    %%% vector j
\newcommand{\bsk}{\boldsymbol{k}}    % vector k
\newcommand{\bsl}{\boldsymbol{l}}    % vector l
\newcommand{\bsr}{\boldsymbol{r}}    % vector r
\newcommand{\bsnu}{\boldsymbol{\nu}}    % vector nu
\newcommand{\dif}{{\rm d}}			%%% Differential dx
\newcommand{\me}{\text{e}}			%%% Do not
\newcommand{\cc}{\mathcal{C}}
\newcommand{\cm}{\mathcal{M}}		%%%
\newcommand{\cl}{\mathcal{L}}
\newcommand{\cn}{\mathcal{N}}
\newcommand{\Order}{\mathcal{O}}
\newcommand{\cp}{\mathcal{P}}
\newcommand{\cx}{\mathcal{X}}
\newcommand{\natm}{\N_{0,m}}
\newcommand{\cube}{[0,1)^d}
\newcommand{\hf}{\hat{f}}
\newcommand{\rf}{\mathring{f}}
\newcommand{\tf}{\tilde{f}}
\newcommand{\hg}{\hat{g}}
\newcommand{\hI}{\hat{I}}
\newcommand{\tvk}{\tilde{\bsk}}
\newcommand{\hS}{\widehat{S}}
\newcommand{\tS}{\widetilde{S}}
\newcommand{\wcS}{\widecheck{S}}
\newcommand{\rnu}{\mathring{\nu}}
\newcommand{\tnu}{\widetilde{\nu}}
\newcommand{\hnu}{\widehat{\nu}}
\newcommand{\hbsnu}{\widehat{\bsnu}}   %%%
\newcommand{\homega}{\widehat{\omega}}
\newcommand{\wcomega}{\mathring{\omega}}
\newcommand{\fC}{\mathfrak{C}}
\newcommand{\nodes}{\{\bsz_i\}_{i=0}^{\infty}}
\newcommand{\nodesn}{\{\bsz_i\}_{i=0}^{n-1}}
\newcommand{\norm}[1]{\ensuremath{\left \lVert #1 \right \rVert}}
\newcommand{\abs}[1]{\ensuremath{\left |  #1 \right |}} %%%
\newcommand{\bigabs}[1]{\ensuremath{\bigl \lvert #1 \bigr \rvert}}
\newcommand{\Bigabs}[1]{\ensuremath{\Bigl \lvert #1 \Bigr \rvert}}
\newcommand{\biggabs}[1]{\ensuremath{\biggl \lvert #1 \biggr \rvert}}
\newcommand{\Biggabs}[1]{\ensuremath{\Biggl \lvert #1 \Biggr \rvert}}
\newcommand{\ip}[3][{}]{\ensuremath{\left \langle #2, #3 \right \rangle_{#1}}}



\title*{Adaptive Multidimensional Integration Based on Rank-1 Lattices}
\author{Fred J. Hickernell \and Llu\'is Antoni Jim\'enez Rugama}
\institute{Fred J. Hickernell \and Llu\'is Antoni
\at Department of Applied Mathematics,  Illinois Institute of Technology, 10 W. 32$^{\text{nd}}$ Street, E1-208, Chicago, IL 60616, USA
\email{hickernell@iit.edu}, \email{ljimene1@hawk.iit.edu}}
\maketitle

\abstract{}

\section{Introduction}
Quasi-Monte Carlo methods are used to approximate multidimensional integration in $\cube$ although this can be generalized in any other domain. People have traditionally used these methods, together with the Monte Carlo methods, because they do not suffer from the \textit{Curse of dimensionality}, i.e. the convergence rate does not depend on the dimension. For example, the dimension-dependent Simpson's rule has $\Order(n^{-4/d})$ and is improved when the dimension $d$ is greater than 4, by the Quasi-Monte Carlo method which has $\Order(n^{-(1-\varepsilon)})$ for small $\varepsilon$. Here $n$ represents the number of data points.

However, the convergence rate does not give us enough information about the absolute error which can be used for building an adaptive algorithm. Although a first approach could be using the Koksma-Hlawka inequality, the main practical disadvantage for it is computing the variation $V(f)$. Thus, in this paper we define a new bound on the absolute error based on the decay of the Fourier coefficients. This will 
Intuitively, a function whose Fourier coefficients decay quickly would 
We use Rank-1 Lattices...

\section{Rank-1 Integration Lattices in $\cube$}
The integrands that will be considered are periodic functions over $\cube$. One may change the domain to some more general by using transformations such as explained in.... ....

As defined in \cite{SloJoe94}, integration lattices $\cm$ are discrete groups in $\R^d$ containing $\Z^d$. Since our domain of interest is $\cube$, the points in the lattice we consider are $\cp:=\cm/\Z^d$. More concisely, the lattice $\cp_m$ with $b^m$ points is built using a generating vector $\bsh$ in $\Z^d$ where $\gcd(h_1,\dots,h_d,b^m)=1$:

\begin{equation}
\cp_m :=\left\{\bsh\frac{j}{b^m}\pmod 1\, ;\, j\in\F_{b^m}\right\}
\end{equation}

Plot example...

In the following, a structure to these points is provided.


\subsection{Group-Like Structures}
Consider $\cube$ with the additive operation $\oplus:\cube \times \cube \to \cube$, $\bsx\oplus\bsy=\bsx+\bsy\pmod 1$. Indeed, $(\cube,\oplus)$ is an Abelian group. Note that $\bszero$ is the additive identity and that the unique additive inverse of $\bsx$ is $\ominus \bsx:=\bold{1}-\bsx$, where $\bsx \ominus \bst$ means $\bsx \oplus (\ominus \bst)$. Moreover, such a set $\cube$ is also a field over $\Z$ when the multiplicative operation is seen by means of $\oplus$:

\[
a \bsx:=\underbrace{\bsx \oplus \cdots \oplus \bsx}_{a \text{ times}}\ \forall a \in \N, \qquad a \bsx:=\underbrace{\ominus\bsx \ominus \cdots \ominus \bsx}_{-a \text{ times}}\ \forall a \in \Z\setminus\N_0.
\]
 
The set $\Z^d$ is used to index Fourier series expressions for the integrands. Hence, for that application we define the bilinear operation $\ip{\cdot}{\cdot}: \Z^d \times \cube \to \cube$,
\begin{subequations} \label{bilinear}
\begin{equation}
\ip{\bsk}{\bsx}=\bsk^T\bsx\pmod 1.
\end{equation}

For all $\bst, \bsx \in \cube$, $\bsk, \bsl \in \Z^d$, and $a \in \Z$, it follows that

\begin{gather}
\ip{\bsk}{\bszero} = \ip{\bszero}{\bsx} = 0,\\
\ip{\bsk}{a \bsx \oplus \bst} = a\ip{\bsk}{\bsx} + \ip{\bsk}{\bst} \pmod 1 \label{bilinearlinxprop} \\
\ip{a \bsk + \bsl}{\bsx} = a\ip{\bsk}{\bsx} + \ip{\bsl}{\bsx} \pmod 1, \label{bilinearlinkprop}\\
\ip{\bsk}{\bsx} = 0 \ \forall \bsk \in \Z^d \ \implies \ \bsx=\bszero.\label{bilinearlinzeroprop}
\end{gather}
\end{subequations}

\subsection{Embedded Rank-1 Lattices and Dual Nets}
By construction, $\cp_m := \{\bsz_i\}_{i=0}^{b^m-1}$ doted with $\oplus$ is an Abelian group. Suppose that there exists a sequence of points in $\cube$, denoted $\cp_\infty =\{\bsz_i\}_{i=0}^{\infty}$ and closed under $\oplus$. Thus, $\{0\}=\cp_0\subseteq\dots\subseteq\cp_m\subseteq\dots\subseteq\cp_\infty$. is an embedded tower of Rank-1 Lattices.

Furthermore, $\cp_\infty$ is assumed to satisfy the following properties:

\begin{subequations} \label{cpinfvector}
\begin{gather}
\{\bsz_{1}, \bsz_{b}, \bsz_{b^2}, \ldots\} \text{ is a set of linearly independent points}, \\
b\bsz_{b^m}=\bsz_{b^{m-1}},\label{latpropb}\\
\bsz_i = \sum_{l=0}^{\infty} i_l \bsz_{b^l}, \qquad \text{where }\bsi=(i_0, i_1, i_2, \ldots ) \in \F_b^\infty, \label{latpropc}\\
\ip{\bsk}{\bsz_i} =  0 \ \forall i \in \N_0   \ \implies \ \bsk=\bszero. \label{latpropd}
\end{gather}
\end{subequations}
Note that from \eqref{bilinear} together with \eqref{latpropb} it follows,
\begin{equation}\label{assumgenip}
\ip{\bsk}{\bsz_{b^{m-1}}}=\ip{b\bsk}{\bsz_{b^m}}
\end{equation}
This property is relevant for defining a Fast Fourier Transform adapted to our Rank-1 Lattices in section \ref{...}....

One example is the extensible $Rank-1$ Lattices from \cite{HicNie03a}. In this case, the generating vector $\bsh$ can be seen in each coordinate as an infinite digit integer. In addition, for every $\cp_m$ we can choose a generator. If we want $\bst_{b^{m-1}}=\bsh\frac{j_m}{b^m}$ to the the generator of $\cp_m$, it only suffices to verify that $\gcd(j_m,b^m)=1$ with $j_m\in\F_{b^m}$. This sequence $j_0,j_1,\dots$ defines $\cp_\infty$. In addition, to satisfy equation \eqref{latpropb},  $j_m$'s: $ b^{m-1} \mid j_m-j_{m-1}\Rightarrow j_m=j_{m-1}+b^{m-1},\;\forall m\in\N$. A part from that, \eqref{latpropc} implies that the order of the elements in $\cp_\infty$ given the generators must follow the Sobol order.

The bilinear operation previously mentioned let us also define the \emph{dual Lattice} corresponding to $\cp_m$ as
\begin{align*}
\cp^{\perp}_m &= \{\bsk \in \Z^d : \ip{\bsk}{\bsz_i} = 0, \ i=0, \ldots, b^m-1\} \\
&= \{\bsk \in \Z^d : \ip{\bsk}{\bsz_{b^{l}}} = 0, \ l=0, \ldots, m-1\}.
\end{align*}
By this definition $\cp^{\perp}_{0}=\Z^d$ and the properties \eqref{bilinear} together with \eqref{cpinfvector}, imply also that $\cp^{\perp}_m$ are subgroups where $\Z^d=\cp^{\perp}_{0}\supseteq\dots\supseteq\cp^{\perp}_{m}\supseteq\dots\supseteq\cp^{\perp}_{\infty}=\{0\}$.

One example of Rank-1 Lattice and its dual Lattice can be found below:
\begin{figure}[h!]
\centering
\begin{tabular}{>{\centering}p{5cm}>{\centering}p{5cm}}
\includegraphics[width=5cm]{Images/sob256pts.eps} &
\includegraphics[width=5cm]{Images/scrsob256pts.eps}\tabularnewline
a) & b)
\end{tabular}
\caption{a) 256 Sobol' points, b) 256 scrambled and digitally shifted Sobol' points \label{Sobolfig}}
\end{figure}

\subsection{Fast Fourier Transform}
The next goal is to define the map $\hbsnu : \Z^d \to \F_b^{\infty}$, and $\tnu_m : \Z^d \to \F_{b^m}$ that facilitates the calculation of the discrete Fourier transform introduced below.

\begin{definition} \label{numapdef} For every $\bsk \in \Z^d$, let
\begin{subequations} \label{numapdefeq}
\begin{gather}
\hbsnu(\bsk)=(\hnu_0(\bsk), \hnu_1(\bsk), \hnu_2(\bsk), \ldots ), \\
\hnu_0(\bsk) = b\ip{\bsk}{\bst_{1}}, \qquad \hnu_m(\bsk)=b\ip{\bsk}{\bst_{b^m}}-\ip{\bsk}{\bst_{b^{m-1}}}, \quad m \in \N, \\
\tnu_m(\bsk) = \sum_{l=0}^{m-1} \hnu_{l}(\bsk) b^{l}, \quad m \in \N.
\end{gather}
\end{subequations}
\end{definition}

These maps have certain desirable properties.

\begin{lemma} \label{numaplem} The following is true for the maps defined in Definition \ref{numapdef}:
\begin{enumerate}
\renewcommand{\labelenumi}{\alph{enumi})}

\item $\hbsnu(\bszero)=\bszero$ and $\tnu_m(\bszero) = 0$ for all $m \in \N$.

\item $\hnu_m(\bsk)\in \{0,\dots,b-1\}$ and $\tnu_m(\bsk)\in \{0,\dots,b^m-1\}$ for all $m\in\N_0$.

\item for all $m\in \N_0$ and all $\bsnu \in \F_b^{m}$ there exist a unique $\bsk \in \Z^d$ with $\hbsnu(\bsk)=(\nu_0, \ldots, \nu_{m-1}, \ldots)$.

\item for any $m \in \N_0$, $i \in \{0, \ldots, b^m-1\}$,  $\tnu_m(\bsk)=\nu=(\nu_0, \nu_1, \ldots)$, and $\bsi=(i_0, i_1, \ldots)$, it follows that
\begin{align} \label{nuwisum}
\begin{split}
\ip{\bsk}{\bst_i} &= \sum_{l=0}^{m-1} i_l [\nu \pmod  {b^{(l+1)}}]  b^{-(l+1)} \pmod 1
\end{split}
\end{align}

\end{enumerate}
\end{lemma}

\begin{proof}
\begin{enumerate}[a)]
\item Directly from definition.
\item Using \eqref{latpropc} and by construction, $\hnu_0(\bsk)\in\{0,\dots,b-1\}$ and $\hnu_m(\bsk)\in (-1,b)$. Using the assumption \eqref{latpropb}, $\hnu_m(\bsk)\pmod 1=\bsk^Tb\bst_{b^m}\pmod 1-\bsk^T\bst_{b^{m-1}}\pmod 1=0$. Then, $\hnu_m(\bsk)\in (-1,b)\cap\Z=\{0,\dots,b-1\},\;\forall m\in\N_0$.

\item For injection, we prove that $\hbsnu(\bsk)=\hbsnu(\bsl) \Rightarrow \bsk = \bsl$. If $\hbsnu(\bsk)=\hbsnu(\bsl)$, $\hnu_m(\bsk)=\hnu_m(\bsl),\;\forall m\in\N_0$. In particular for $m=0$, this implies $\ip{\bsk}{\bst_1}-\ip{\bsl}{\bst_1}=0$. Assume now that $\ip{\bsk}{\bst_{b^m}}-\ip{\bsl}{\bst_{b^m}}=0$. Since $\hnu_{m+1}(\bsk)-\hnu_{m+1}(\bsl)=0$, then $\ip{\bsk}{\bst_{b^{m+1}}}-\ip{\bsl}{\bst_{b^{m+1}}}=0$. By induction $\ip{\bsk}{\bst_{b^{m}}}-\ip{\bsl}{\bst_{b^{m}}}=\ip{\bsk-\bsl}{\bst_{b^{m}}}=0$ for all $m\in\N_0$. Thus, by \eqref{latpropd}, $\bsk=\bsl$.

\vspace{2mm}
    For the surjection, by \eqref{bilinearlinzeroprop} there exists $\bsk$ such that $\hnu_0(\bsk)=\nu\neq 0$. Furthermore due to the property \eqref{bilinearlinkprop}, $\hnu_0(a\bsk)=a\nu \pmod b$ and recalling the Lagrange's Theorem, any element $\nu$ different of the identity generates the group $\F_b$. Therefore, $a=1,...,b-1$ gives us any element of $\F_b$.

    Now, for any $l\leq m\in\N$ and using \eqref{assumgenip},
\begin{align*}
\hnu_l(\bsk+b^m\bsa)&=
\begin{cases}
b\ip{\bsk}{\bst_{b^l}}-\ip{\bsk}{\bst_{b^{l-1}}}+b\ip{\bsa}{\bst_{1}} \pmod b  &\mbox{if } l=m, \\
b\ip{\bsk}{\bst_{b^l}}-\ip{\bsk}{\bst_{b^{l-1}}} &\mbox{if } l<m   \end{cases}\\
&=
\begin{cases}
\hnu_l(\bsk)+\hnu_0(\bsa) \pmod b  &\mbox{if } l=m, \\
\hnu_l(\bsk) &\mbox{if } l<m   \end{cases}\\
\end{align*}
%Remark that for the first equality we use the fact that $\ip{\bsk}{\bst_{b^{m-1}}}=\ip{\bsk}{\bst_{b^{m-1}}}\pmod b$.
Therefore, for all $m\in \N_0$ and all $(\nu_0,\dots,\nu_m) \in \F_b^{m+1}$ there exist a $\bsk \in \Z^d$ with $\hnu_l(\bsk)=\nu_l$, $l=0,\dots,m$. This means $\hbsnu(\bsk)$ is bijective.

\item Follows by applying  \eqref{bilinearlinxprop} and Definition \ref{numapdef}:
\begin{align*}
\ip{\bsk}{\bst_i} &= \ip{\bsk}{\sum_{l=0}^{m-1} i_l \bst_{b^{l}}} = \sum_{l=0}^{m-1} i_l \ip{\bsk}{\bst_{b^{l}}} \pmod 1 \\
& = \sum_{l=0}^{m-1} i_l \sum_{j=0}^{l} \nu_jb^{j-(l+1)} \pmod 1\\
&=\sum_{l=0}^{m-1} i_l [\nu \pmod  {b^{(l+1)}}]  b^{-(l+1)} \pmod 1.
\end{align*}

\end{enumerate}
\end{proof}

\section{Fourier Series}

The integrands are assumed to be periodic and belong to some subset of $\cl_2(\cube)$, the space of square integrable functions. For non periodic integrands we suggest some transforms in \eqref{periodizing}. The $\cl_2$ inner product is defined as
\[
\ip[2]{f}{g} = \int_{\cube} f(\bsx) \overline{g(\bsx)} \, \dif \bsx.
\]
Let $\{\varphi(\cdot,\bsk) \in \cl_2(\cube) : \bsk \in \Z^d\}$ be the complete orthonormal Fourier function \emph{basis} for $\cl_2(\cube)$, i.e.,
\[
\varphi(\bsx,\bsk)  = \me^{2 \pi \sqrt{-1} \ip{\bsk}{\bsx}}, \qquad \bsk \in \Z^d, \ \bsx \in \cube.
\]
Then any function in $\cl_2$ may be written in series form as
\begin{equation} \label{Fourierdef}
f(\bsx) = \sum_{\bsk \in \Z^d} \hf(\bsk) \varphi(\bsx,\bsk), \quad \text{where } \hf(\bsk) = \ip[2]{f}{\varphi(\cdot,\bsk)},
\end{equation}
and the inner product of two functions in $\cl_2$ is the $l_2$ inner product of their series coefficients:
\[
\ip[2]{f}{g} = \sum_{\bsk \in \Z^d} \hf(\bsk)\overline{\hg(\bsk)} =: \ip[2]{\bigl(\hf(\bsk)\bigr)_{\bsk \in \Z^d}}{\bigl ( \hg(\bsk)\bigr )_{\bsk \in \Z^d}}.
\]

Because the set $\cp_m$ is closed under $\oplus$, we can derive a useful formula for the average of any Fourier basis function over $\cp_m$. For all $\bsk \in \Z^d$ and $\bsx \in \cp$, it follows that
\begin{align*}
\nonumber
0 & = \frac{1}{b^m} \sum_{i=0}^{b^m-1} [\varphi(\bsz_i,\bsk) - \varphi(\bsz_i \oplus \bsx,\bsk)]
= \frac{1}{b^m} \sum_{i=0}^{b^m-1} [\me^{2 \pi \sqrt{-1} \ip{\bsk}{\bsz_i}} - \me^{2 \pi \sqrt{-1} \ip{\bsk}{\bsz_i \oplus \bsx}}]\\
\nonumber
& = \frac{1}{b^m} \sum_{i=0}^{b^m-1} [\me^{2 \pi \sqrt{-1} \ip{\bsk}{\bsz_i}} - \me^{2 \pi \sqrt{-1} \{\ip{\bsk}{\bsz_i}+\ip{\bsk}{\bsx}\}}] \quad \text{by } \eqref{bilinearlinxprop}\\
\label{sumeq}
& = [1 - \me^{2 \pi \sqrt{-1} \ip{\bsk}{\bsx})}] \frac{1}{b^m} \sum_{i=0}^{b^m-1}  \me^{2 \pi \sqrt{-1} \ip{\bsk}{\bsz_i}}.
\end{align*}
By this equality it follows that the average of a Fourier basis function sampled over the points in a Lattice is either one or zero, depending on whether the wavenumber $\bsk$ is in the dual set or not:
\begin{equation}\label{avrFourier}
\frac{1}{b^m} \sum_{i=0}^{b^m-1}  \me^{2 \pi \sqrt{-1} \ip{\bsk}{\bsz_i}} = \ind_{\cp_m^{\perp}}(\bsk) = \begin{cases} 1 , & \bsk \in \cp_m^{\perp}\\
 0,  & \bsk \in \Z^d \setminus \cp_m^{\perp}.
 \end{cases}
\end{equation}

We can approximate multivariate integrals by taking the average of the integrand sampled over a shifted Lattice, namely,

\begin{equation} \label{cubaturedef}
\hI_m(f) := \frac{1}{b^m} \sum_{i=0}^{b^m-1} f(\bsz_i \oplus\bsDelta_i).
\end{equation}

Using the series decomposition defined in \eqref{Fourierdef} and equation \eqref{avrFourier}, it follows that the error of the Lattice rule is the sum of the Fourier coefficients of the integrand for those wavenumbers in the dual Lattice:

\begin{align}
\nonumber
\biggabs{ \int_{\cube} f(\bsx) \, \D \bsx - \hI_m(f)} 
& = \Biggabs {\hf(\bszero) - \sum_{\bsk \in \Z^d} \hf(\bsk) \hI_m\left(\E^{2 \pi \sqrt{-1} \ip{\bsk}{\cdot}}\right)} \\
\nonumber
& = \Biggabs {\hf(\bszero) - \sum_{\bsk \in \Z^d} \hf(\bsk) \ind_{\cp_m^{\perp}}(\bsk) \E^{2 \pi \sqrt{-1} \ip{\bsk}{\bsDelta}}} \\ 
& = \Biggabs {\sum_{\bsk \in \cp_m^{\perp}\setminus \{\bszero\} } \hf(\bsk) \E^{2 \pi \sqrt{-1} \ip{\bsk}{\bsDelta}}}. \label{err1}
\end{align}

Adaptive Algorithm \ref{adapalgo} that we construct in section \ref{ErrEstsec} is built with this expression in terms of Fourier coefficients.

%\subsection{The Discrete Fourier Transform}

One does not have to assume the knowledge of the Fourier coefficients since they can be estimated by the the discrete Fourier transform, defined as follows:

\begin{align}
\nonumber
\tf_m(\bsk)
&:= \hI_m\left(\E^{-2 \pi \sqrt{-1} \ip{\bsk}{\cdot}} f(\cdot) \right) \\
\nonumber
&= \frac{1}{b^m} \sum_{i=0}^{b^m-1} \me^{-2 \pi \sqrt{-1} \ip{\bsk}{\bsz_i\oplus\bsDelta}} f(\bsz_i\oplus\bsDelta) \\
\nonumber
&= \frac{1}{b^m}  \sum_{i=0}^{b^m-1} \left[\me^{-2 \pi \sqrt{-1} \ip{\bsk}{\bsz_i\oplus\bsDelta}}\sum_{\bsl \in \Z^d} \hf(\bsl) \me^{2 \pi \sqrt{-1} \ip{\bsl}{\bsz_i\oplus\bsDelta}} \right] \\
\nonumber
& = \sum_{\bsl \in \Z^d} \hf(\bsl)  \frac{1}{b^m}  \sum_{i=0}^{b^m-1}  \me^{2 \pi \sqrt{-1} \ip{\bsl - \bsk}{\bsz_i\oplus\bsDelta}} \\
\nonumber
& = \sum_{\bsl \in \Z^d} \hf(\bsl) \me^{2 \pi \sqrt{-1} \ip{\bsl - \bsk}{\bsDelta}}  \frac{1}{b^m}  \sum_{i=0}^{b^m-1}  \me^{2 \pi \sqrt{-1} \ip{\bsl - \bsk}{\bsz_i}} \\
\displaybreak[0] \nonumber
& = \sum_{\bsl \in \Z^d} \hf(\bsl) \me^{2 \pi \sqrt{-1} \ip{\bsl - \bsk}{\bsDelta}} \ind_{\cp_m^{\perp}}(\bsl - \bsk) \\
\nonumber
& = \sum_{\bsl \in \cp^{\perp}_m} \hf(\bsk+\bsl) \me^{2 \pi \sqrt{-1} \ip{\bsl}{\bsDelta}} \\
&= \hf(\bsk) + \sum_{\bsl \in \cp^{\perp}_m\setminus \bszero} \hf(\bsk+\bsl) \me^{2 \pi \sqrt{-1} \ip{\bsl}{\bsDelta}}, \qquad \forall \bsk \in \Z^d. \label{tfassum}
\end{align}
It is seen here that the discrete transform $\tf_m(\bsk)$ is equal to the integral transform $\hf(\bsk)$, defined in \eqref{Fourierdef}, plus the \emph{aliasing} terms corresponding to $\hf(\bsl)$ where $\bsl-\bsk \in \cp_{m}^{\perp}\setminus \bszero$.

\subsection{Periodizing Transformations}\label{periodizing}
For non periodic integrands $f$ we can use transformations $\phi$ such that:
\begin{equation}\label{transeq}
\int_{\cube} f(\bsx)  \, \dif \bsx=\int_{\cube} g(\bst)  \, \dif \bst
\end{equation}
where
\begin{equation}\label{gdef}
g(t_1,\dots,t_d)=f(\phi(t_1),\dots,\phi(t_d))\phi'(t_1)\cdots\phi'(t_d)
\end{equation}

One example is the Baker's transform $\phi(t)=1-2\abs{t-0.5}$. A simple calculation in one dimension proves the equality \eqref{transeq} and by the Fubini theorem it can be extended to any dimension. Remark that this transform may generate a periodically extended $g$ not differentiable. Therefore, for smoother integrands $f$ we may loose differentiability after applying the transformation. However, really smooth transformations may introduce waviness. In any of both cases, it implies slower Fourier coefficients decay. See section \eqref{sumscoeff} for further implications on this paper.

Another option is described in \cite{SloJoe94}. Here, $\phi$ must be an increasing function which maps $[0,1]$ onto $[0,1]$. If $\phi'(0)=\phi'(1)=0$, by \eqref{gdef} function $g$ becomes periodic. The simplest polynomial fulfilling this conditions is $\phi(t)=3t^2-2t^3$. Again, as for the Baker's transform, this may produce a non differentiable $g$. One solution, if $f\in\cc^n$, is finding the polynomial requiring $\phi^{(i)}(0)=\phi^{(i)}(1)=0$  where $i=1,\dots,n+1$ for $f\in\cc^n$, implying $g\in\cc^n$. A part from polynomials, one could also consider the transform proposed by Sidi in \cite{sidi1993new}: $\phi(t)=t-\sin(2\pi t)/(2\pi)$.

\subsection{Computation of the Discrete Transform}
The discrete transform defined in \eqref{tfdef} may also be expressed as
\begin{align}
\nonumber
\tf_m(\bsk)
&= \frac{1}{b^m} \sum_{i=0}^{b^m-1} \me^{-2 \pi \sqrt{-1} \ip{\bsk}{\bst_i\oplus \bsDelta}} f(\bst_i\oplus \bsDelta) \\
\nonumber
&= \frac{\me^{-2 \pi \sqrt{-1} \ip{\bsk}{\bsDelta}}}{b^m} \sum_{i=0}^{b^m-1} \me^{-2 \pi \sqrt{-1} \ip{\bsk}{\bst_i}} f(\bst_i\oplus \bsDelta).
\end{align}
Letting $y_i=f(\bst_i\oplus \bsDelta)$,
\[
Y_{m,0}(i_0,\ldots, i_{m-1}) = y_i, \qquad i=i_0 + i_1 b + \cdots + i_{m-1} b^{m-1},
\]
and invoking Lemma \ref{numaplem}, for any $\bsk \in \Z^d$ with $\tnu_m(\bsk)=\nu = \nu_0 + \nu_1 b  + \cdots + \nu_{m-1} b^{m-1}$ one may write
\begin{align}
\nonumber
\tf_m(\bsk) &= \me^{-2 \pi \sqrt{-1} \ip{\bsk}{\bsDelta}}  Y_{m,m}(\nu_0, \ldots, \nu_{m-1}) , \\
\nonumber
\MoveEqLeft{Y_{m,m}(\nu_0, \ldots, \nu_{m-1})}\\
\nonumber
& : = \frac{1}{b^m} \sum_{i=0}^{b^m-1} \me^{-2 \pi \sqrt{-1} \ip{\bsk}{\bst_i}} y_i \\
\nonumber
& = \frac{1}{b^m} \sum_{i_{m-1}=0}^{b-1} \cdots \sum_{i_0=0}^{b-1} \me^{-2 \pi \sqrt{-1}\sum_{l=0}^{m-1} i_l [\nu \pmod  {b^{(l+1)}}]  b^{-(l+1)}} Y_{m,0}(i_0,\ldots, i_{m-1}) \\
\nonumber
& = \frac{1}{b} \sum_{i_{m-1}=0}^{b-1}\me^{-2 \pi \sqrt{-1}  i_{m-1}[\nu \pmod  {b^{m}}]  b^{-m}}  \cdots \\
&\qquad \qquad \frac{1}{b} \sum_{i_0=0}^{b-1} \me^{-2 \pi \sqrt{-1} i_0 [\nu \pmod  {b}]  b^{-1}} Y_{m,0}(i_0,\ldots, i_{m-1})
\nonumber
\end{align}
This sum can be computed recursively:
\begin{multline*}
Y_{m,l+1}(\nu_0, \ldots, \nu_{l},i_{l+1}, \ldots, i_m) \\
= \frac{1}{b} \sum_{i_l=0}^{b-1} \me^{-2 \pi \sqrt{-1}  i_{l}[\nu \pmod  {b^{(l+1)}}]  b^{-(l+1)}} Y_{m,l}(\nu_0, \ldots, \nu_{l-1},i_{l}, \ldots, i_m)
\end{multline*}

In light of this development we define $\mathring{f}_m(\nu)=Y_{m,m}(\nu_0, \ldots, \nu_{m-1})$ for $\nu=0, \ldots, b^{m}-1$. Then
\[
\tf(\bsk) = \me^{-2 \pi \sqrt{-1} \ip{\bsk}{\bsDelta}} \mathring{f}_m(\tnu(\bsk)).
\]

\section{Error Estimation and an Automatic Algorithm}

\subsection{Wavenumber Map}

Now we are going to map the non-negative numbers into the space of all wavenumbers using the dual sets.  For every $\kappa \in \N_0$, we assign a wavenumber $\tvk(\kappa) \in \Z^d$ iteratively according to the following constraints:
\begin{enumerate}
\renewcommand{\labelenumi}{\roman{enumi})}

\item $\tvk(0)= \bszero$;

\item For any $\lambda, m \in \N_0$ and $\kappa=0, \ldots, b^m-1$, it follows that  $\tnu_m(\tvk(\kappa))=\tnu_m(\tvk(\kappa+\lambda b^m))$.

\end{enumerate}
This last condition implies that $\tvk(\kappa) - \tvk(\kappa+\lambda b^m) \in \cp_m^{\perp}$.

This wavenumber map allows us to introduce a shorthand notation that facilitates the later analysis for $\kappa \in \N_0$ and $m \in \N$:
\begin{align*}
\hf_{\kappa} & =\hf(\tvk(\kappa)), \\
\tf_{m,\kappa}& = \tf_m(\tvk(\kappa))= \me^{-2 \pi \sqrt{-1} \ip{\tvk(\kappa)}{\bsDelta}} \rf_m(\tnu_m(\tvk(\kappa)))\\
&= \me^{-2 \pi \sqrt{-1} \ip{\tvk(\kappa)}{\bsDelta}} \rf_{m,\kappa},
\end{align*}
where $\rf_{m,\kappa}:=\rf_m(\tnu_m(\tvk(\kappa)))$. According to \eqref{tfassum}, it follows that
\begin{equation}
\tf_{m,\kappa} = \hf_{\kappa} + \sum_{\lambda=1}^{\infty} \hf_{\kappa+\lambda b^{m}} \me^{2 \pi \sqrt{-1} \ip{\tvk(\kappa+\lambda b^{m}) - \tvk(\kappa)}{\bsDelta}}.
\label{tfassumc}
\end{equation}
We want to use $\tf_{m,\kappa}$ to estimate $\hf_{\kappa}$ if $m$ is signficantly larger than $\lfloor \log_b(\kappa) \rfloor$.

\subsection{Sums of Series Coefficients and Their Bounds}\label{sumscoeff}
Consider the following sums of the series coefficients defined for $l,m \in \N_0$, $l \le m$:
\begin{gather*}
S(m) =  \sum_{\kappa=\left \lfloor b^{m-1} \right \rfloor}^{b^{m}-1} \bigabs{\hf_{\kappa}}, \qquad
\hS(l,m)  = \sum_{\kappa=\left \lfloor b^{l-1} \right \rfloor}^{b^{l}-1} \sum_{\lambda=1}^{\infty} \bigabs{ \hf_{\kappa+\lambda b^{m}}}, \\
\wcS(m)=\hS(0,m) + \cdots + \hS(m,m)=
\sum_{\kappa=b^{m}}^{\infty} \bigabs{\hf_{\kappa}}, \\
\tS(l,m) = \sum_{\kappa=\left \lfloor b^{l-1}\right \rfloor}^{b^{l}-1} \bigabs{\tf_{m,\kappa}} = \sum_{\kappa=\left \lfloor b^{l-1}\right \rfloor}^{b^{l}-1} \bigabs{\rf_{m,\kappa}}.
\end{gather*}
The first three kinds of sums, $S(\cdot)$, $\hS(\cdot,\cdot)$, and $\wcS(\cdot)$, which involve the true series coefficients, cannot be observed, but the last one, $\tS(\cdot, \cdot)$, which involves the discrete transform coefficients, can easily be observed.

We now make critical assumptions that $\hS(l,m)$ and $\wcS(m)$ can be bounded above in terms of $S(l)$, provided that $l$ is large enough.  Let $l,m \in \N_0$ with $l \le m$, and fix $l_* \in \N$.  It is assumed that there exist known,  non-negative valued functions $\homega$ and $\wcomega$ with $\lim_{m \to \infty} \wcomega(m) = 0$ such that
\begin{equation} \label{conecond}
\hS(l,m) \le \homega(m-l) \wcS(m) \quad \forall l, \qquad
\wcS(m) \le \wcomega(m-l) S(l) \quad \forall l_* \le l.
\end{equation}
By the definition of $\wcS(m)$, the choice $\homega(m):=1$ for all $m$ is always guaranteed to work.  However, one might also consider choosing $\homega(m)=C b^{-m}$ for some $C$.  The reason for enforcing the second assumption only  for $l \ge l_*$ is that for small $l$, one might have a coincidentally small $S(l)$, since it only involves $b^l$ coefficients, while $\wcS(m)$ is large.

Under this assumption, for $l, m \in \N$, $l_* \le l \le m$, it is possible to bound the sum of the true coefficients, $S(l)$, in terms of the observed sum of the discrete coefficients, $\tS(l,m)$, as follows:
\begin{align*}
S(l) &= \sum_{\kappa=b^{l-1}}^{b^{l}-1} \bigl \lvert \hf_{\kappa}\bigr\rvert= \sum_{\kappa=b^{l-1}}^{b^{l}-1} \abs{\tf_{m,\kappa} - \sum_{\lambda=1}^{\infty} \hf_{\kappa+\lambda b^{m}} \me^{2 \pi \sqrt{-1} \ip{\tvk(\kappa+\lambda b^{m}) \ominus \tvk(\kappa)}{\bsDelta}/b}}\\
&\le \sum_{\kappa=b^{l-1}}^{b^{l}-1} \bigl \lvert \tf_{m,\kappa} \bigr\rvert + \sum_{\kappa=b^{l-1}}^{b^{l}-1} \sum_{\lambda=1}^{\infty} \bigl \lvert \hf_{\kappa+\lambda b^{m}}\bigr\rvert = \tS(l,m) + \hS(l,m) \\
&\le \tS(l,m) + \homega(m-l) \wcomega(m-l) S(l) \\
S(l) & \le \frac{\tS(l,m)}{1 - \homega(m-l) \wcomega(m-l)} \qquad \text{provided that } \homega(m-l) < 1.
\end{align*}

Using this upper bound, one can then conservatively bound the error of integration using the shifted node set.  For for $l, m \in \N$, $l_* \le l \le m$, it follows that
\begin{align*}
\MoveEqLeft{\abs{\int_{\cube} f(\bsx) \, \dif \bsx - \frac{1}{b^m} \sum_{i=0}^{b^m-1} f(\bsx_i) }}\\
&= \abs{\hf(\bszero) - \tf_m(\bszero)} = \abs{\hf_0 - \tf_{m,0}} = \abs{\sum_{\lambda=1}^{\infty} \hf_{\lambda b^{m}} \me^{2 \pi \sqrt{-1}\ip{\tvk(\lambda b^{m})}{\bsDelta}}}\\
&\le \sum_{\lambda=1}^{\infty} \bigabs{\hf_{\lambda b^{m}}}
= \hS(0,m) \le \homega(m) \wcS(m) \le \homega(m) \wcomega(m-l) S(l) \\
& \le \frac{\tS(l,m)\homega(m) \wcomega(m-l)}{1 - \homega(m-l) \wcomega(m-l)}.
\end{align*}

This error bound suggests the following algorithm.  Choose $r \in \N$ such that $\homega(r)\wcomega(r)<1$.  For $j \in \N$ define
\[
l_j=j+l_*-1, \qquad  m_j=j+l_*+r-1, \qquad  \fC = \frac{\wcomega(r)}{1 - \homega(r)\wcomega(r)}.
\]
Define $l_j=l_*+j-1$ and $m_j=l_j+r$.  Given a tolerance $\varepsilon$, and an integrand $f$, do the following:  for $j=1, 2, \ldots$ check whether
\[
\fC \homega(m_j)  \tS(l_j,m_j) \le \varepsilon.
\]
If so, we're done.  If not, increment $j$ by one and repeat.

Given $\homega$, $\wcomega$, and $r$, one can compute $\fC$.  Alternatively, given $\fC$, $\homega$, and $r$, one can compute $\wcomega(r)$:
\[
\fC = \frac{\wcomega(r)}{1 - \homega(r)\wcomega(r)} \ \iff \ \wcomega(r)= \frac{\fC}{1+\fC\homega(r)}.
\]


\section{Walsh Series} \label{WaveWalshsec}

\section{Error Estimation and an Adaptive Cubature Algorithm} \label{ErrEstsec}

\subsection{Wavenumber Map} \label{wavenummapsec}

\subsection{An Adaptive Cubature Algorithm and Its Cost}

\section{Numerical Experiments} \label{numexpsec}


\section{Discussion}


\begin{acknowledgement}
\end{acknowledgement}

\bibliographystyle{spmpsci.bst}
\bibliography{FJH22,FJHown22}

\section*{Appendix:  Fast Computation of the Discrete Walsh Transform}



\end{document}