%%%%%%%%%%%%%%%%%%%% author.tex %%%%%%%%%%%%%%%%%%%%%%%%%%%%%%%%%%%
%
% sample root file for your "contribution" to a contributed volume
%
% Use this file as a template for your own input.
%
%%%%%%%%%%%%%%%% Springer %%%%%%%%%%%%%%%%%%%%%%%%%%%%%%%%%%
\documentclass[graybox,footinfo]{svmult}

\smartqed
\usepackage{mathptmx}       % selects Times Roman as basic font
\usepackage{helvet}         % selects Helvetica as sans-serif font
\usepackage{courier}        % selects Courier as typewriter font
\usepackage{type1cm}        % activate if the above 3 fonts are
                            % not available on your system
\usepackage{graphicx}       % standard LaTeX graphics tool
                            % when including figure files

\usepackage{array,colortbl}
\usepackage{amsmath,amsfonts,amssymb,bm} % no amsthm, Springer defines Theorem, Lemma, etc themselves

% Note that Springer defines the following already:
%
% \D upright d for differential d
% \I upright i for imaginary unit
% \E upright e for exponential function
% \tens depicts tensors as sans serif upright
% \vec depicts vectors as boldface characters instead of the arrow accent
%
% Additionally we throw in the following common used macro's:
\newcommand{\Z}{\mathbb{Z}} % integers
\newcommand{\C}{\mathbb{C}} % complex numbers
\newcommand{\R}{\mathbb{R}} % reals
\newcommand{\N}{\mathbb{N}} % natural numbers {1, 2, ...}
\newcommand{\Q}{\mathbb{Q}} % rationals
\newcommand{\F}{\mathbb{F}} % field, finite field
\newcommand{\floor}[1]{\left\lfloor #1 \right\rfloor} % floor
\newcommand{\ceil}[1]{\left\lceil #1 \right\rceil}    % ceil
\newcommand{\rd}{\,\mathrm{d}} % differential symbol for use in integrals
% vectors as boldsymbols:
\newcommand{\bszero}{\boldsymbol{0}} % vector of zeros
\newcommand{\bsone}{\boldsymbol{1}}  % vector of ones
\newcommand{\bst}{\boldsymbol{t}}    % vector t
\newcommand{\bsu}{\boldsymbol{u}}    % vector u
\newcommand{\bsv}{\boldsymbol{v}}    % vector v
\newcommand{\bsw}{\boldsymbol{w}}    % vector w
\newcommand{\bsx}{\boldsymbol{x}}    % vector x
\newcommand{\bsy}{\boldsymbol{y}}    % vector y
\newcommand{\bsz}{\boldsymbol{z}}    % vector z
\newcommand{\bsDelta}{\boldsymbol{\Delta}}    % vector \Delta
% sets as Euler fraks:
\newcommand{\setu}{\mathfrak{u}}
\newcommand{\setv}{\mathfrak{v}}
% indicator boldface 1:
\newcommand{\ind}{\mathbbold{1}}


\usepackage{microtype} % good font tricks

\usepackage[colorlinks=true,linkcolor=black,citecolor=black,urlcolor=black]{hyperref}
\urlstyle{same}
\usepackage{bookmark}
\pdfstringdefDisableCommands{\def\and{, }}
\makeatletter % to avoid hyperref warnings:
  \providecommand*{\toclevel@author}{999}
  \providecommand*{\toclevel@title}{0}
\makeatother

%%%%%%%%%%%%%%%%%%%%%%%%%%%%%%%%%%%%%%%%%%%%%%%%%%%%%%%%%%%%%%%%%%%%%%%%%%%%%%%%%%%%%%%%%
\begin{document}

\title*{Contribution Titles are With Capitals}
% Use \titlerunning{Short Title} for an abbreviated version of
% your contribution title if the original one is too long
\author{Tim Arnolds \and Annie G.\ Beatson \and Elmo Watson}
% Use \authorrunning{Short Title} for an abbreviated version of
% your contribution title if the original one is too long
\institute{
Tim Arnolds \and Elmo Watson 
\at Disney University, 22A Sesame Street, New York, USA 
\email{tim.arnolds@disney.com}, \email{ewatson@disney.com}
%
\and 
%
Annie G.\ Beatson 
\at Beatson University, PO Box 45, Dunedin, New Zealand
\email{annie@beatson.nz}
}
\maketitle

\abstract{Each contribution should be preceded by an abstract (10--15 lines long) that summarizes the content. The abstract will appear \textit{online} at \url{www.SpringerLink.com} and be available with unrestricted access. This allows unregistered users to read the abstract as a teaser for the complete contribution. 
Please use the `starred' version of the new Springer \texttt{abstract} command for typesetting the text of the online abstracts (cf. source file of this chapter template \texttt{abstract}) and include them with the source files of your manuscript. Use the plain \texttt{abstract} command if the abstract is also to appear in the printed version of the book.}

\section{Section Headings are With Capitals}\label{sec:1}

We give some sample text for illustrative purposes.
Note that you have to write \verb+\qed+ at the end of a proof explicitly.
Furthermore there is no \verb+\qedhere+ command so it is best to end your proofs with a sentence.
The sample text comes from \cite{ACN2013}.

Blah blah blah\ldots\ The optimization problem becomes
\begin{align}
 \underset{Q_{\bullet k}\in\mathbb{R}^{2m}}{\text{maximize}} \qquad& \left( \left.\frac{\partial f}{\partial z_k}\right|_{\bsz=\hat{\bsz}_k}\right)^2 \label{eq:optLT}\\
  \text{subject to} \qquad& \lVert Q_{\bullet k} \rVert=1, \nonumber\\
                    \qquad& \langle Q^*_{\bullet j},Q_{\bullet k} \rangle=0, \quad j=1,\ldots,k-1.\nonumber
\end{align}

Similar to \cite{IT2006},
to find the partial derivatives $\partial \hat{S}_m/\partial z_i$ needed for the optimization algorithm, we obtain the recursive relations (with initial conditions $\partial \log \hat{S}_0 /\partial z_i=0$ and $\partial \hat{V}_0/\partial z_i=0$)
\begin{align}
  \frac{\partial \log\hat{S}_{k+1}}{\partial z_i}
  &=
  \frac{\partial\log\hat{S}_{k}}{\partial z_i} + \frac{\partial\hat{V}_k}{\partial z_i}f^1_{k} + \left(\rho q_{2k+1,i}+\sqrt{1-\rho^2}q_{2k+2,i}\right
) f^2_{k}
  , \label{eq:recLT1} \\
  \frac{\partial \hat{V}_{k+1}}{\partial z_i}
 &=
 \frac{\partial \hat{V}_k}{\partial z_i} f^3_k + q_{2k+1,i} f^4_k
  \label{eq:recLT2}
  ,
\end{align}
where $k$ goes from $0$ to $m-1$.
The chain rule is used to obtain
\begin{align*}
 \frac{\partial \hat{S}_m}{\partial z_i}
 &=
 \hat{S}_m \frac{\partial \log \hat{S}_m}{\partial z_i}
 .
\end{align*}
We will use the following lemma to calculate the transformation matrix.
\begin{lemma}\label{lem:RecAid}
  The recursion
  \begin{align*}
    F_{k+1} &= a_k F_k + b_k q_k ,
    \\
    G_{k+1} &= c_k G_k + d_k q_k + e_k F_k,
  \end{align*}
  with initial values $F_0 = G_0 = 0$ can be written at index $k+1$ as a linear combination of the $q_{\ell}$, $\ell=0,\ldots,k$, as follows
  \begin{align*}
    F_{k+1} &= \sum_{\ell=0}^k q_{\ell} \, b_{\ell} \prod_{j=\ell+1}^k a_j ,
    \\
    G_{k+1} &= \sum_{\ell=0}^k q_{\ell}
      \left(
        d_{\ell} \prod_{j=\ell+1}^k c_j
        +
        b_{\ell} \sum_{t=\ell+1}^k e_t \prod_{v=t+1}^k c_v \prod_{v=\ell+1}^{t-1} a_v\right)
    .
  \end{align*}
\end{lemma}
\begin{proof}
  The formula for $F_{k+1}$ follows immediately by induction.
  For the expansion of $G_{k+1}$ we first rewrite this formula in a more explicit recursive form
  \begin{align*}
    G_{k+1}
    &=
    \sum_{\ell=0}^{k} q_{\ell} d_{\ell} \prod_{j=\ell+1}^k c_j
    +
    \sum_{\ell=0}^{k-1} q_{\ell} b_{\ell} \sum_{t=\ell+1}^k e_t \prod_{v=t+1}^k c_v  \prod_{v=\ell+1}^{t-1} a_v
    \\
    &=
    \sum_{\ell=0}^{k} q_{\ell} d_{\ell} \prod_{j=\ell+1}^k c_j
    +
    \sum_{t=1}^k e_t \prod_{v=t+1}^k c_v \left( \sum_{\ell=0}^{t-1} q_{\ell} b_{\ell} \prod_{v=\ell+1}^{t-1} a_v \right)
    .
  \end{align*}
  The part in-between the braces equals $F_t$ and the proof now follows by induction on~$k$.
  \qed
\end{proof}

\begin{proposition}\label{prop:optvLTc}
 The column vector $Q_{\bullet k}$ that solves the optimization problem~\eqref{eq:optLT} for a call option under the Heston model is given by $Q_{\bullet k} = \pm \bsv / \|\bsv\|$ where
\begin{align*}
  v_{2\ell+1} &= \hat{S}_m f^2_{\ell} \rho + \hat{S}_m f^4_{\ell} \sum_{t=\ell+1}^{m-1}f^1_t\prod_{v=\ell+1}^{t-1}f^3_v
  ,
  \\
   v_{2\ell+2} &= \hat{S}_m f^2_{\ell} \sqrt{1-\rho^2},
\end{align*}
for $\ell=0,\ldots,m-1$.
\end{proposition}
\begin{proof}
  By \cite[Theorem 1]{IT2006} the solution to the optimization problem~\eqref{eq:optLT} is given by
  \begin{align*}
   Q_{\bullet k}
   &=
   \pm \frac{\bsv}{\| \bsv \|}
   ,
  \end{align*}
  where $\bsv$ is determined from
  \begin{align*}
    Q_{\bullet k}' \bsv
    &=
    \frac{\partial \hat{S}_m}{\partial z_k}
    =
    \hat{S}_m \frac{\partial \log \hat{S}_m}{\partial z_k}
    .
  \end{align*}
  With the help of Lemma~\ref{lem:RecAid} we find from~\eqref{eq:recLT1} and~\eqref{eq:recLT2}
  \begin{align*}
    \frac{\partial \log \hat{S}_m}{\partial z_k}
    &=
    \sum_{\ell=0}^{m-1} q_{2\ell+1,k} \left( \rho f^2_{\ell} + f^4_{\ell} \sum_{t=\ell+1}^{m-1} f^1_t \prod_{v=\ell+1}^{t-1} f^3_v \right)
    +
    \sum_{\ell=0}^{m-1} q_{2\ell+2,k} \sqrt{1-\rho^2} f^2_{\ell}
    ,
  \end{align*}
  from which the result now follows. 
  \qed
\end{proof}

\begin{remark}
    This is a remark.
\end{remark}

\section{Section Heading}
\label{sec:2}

This section could be similar to Section~\ref{sec:1} but we decided to include a table here.
Please see Table~\ref{tab:1}.


%
% For tables use
%
\begin{table}
\caption{Please write your table caption here}\label{tab:1}
%
% Follow this input for your own table layout
%
\begin{tabular}{p{2cm}p{2.4cm}p{2cm}p{4.9cm}}
\hline\noalign{\smallskip}
Classes & Subclass & Length & Action Mechanism  \\
\noalign{\smallskip}\svhline\noalign{\smallskip}
Translation & mRNA$^a$  & 22 (19--25) & Translation repression, mRNA cleavage\\
Translation & mRNA cleavage & 21 & mRNA cleavage\\
Translation & mRNA  & 21--22 & mRNA cleavage\\
Translation & mRNA  & 24--26 & Histone and DNA Modification\\
\noalign{\smallskip}\hline\noalign{\smallskip}
\end{tabular}
$^a$ Table foot note (with superscript)
\end{table}

%%%%%%%%%%%%%%%%%%%%%%%%%%%%%%%%%%%%%%%%%%%%%%%%%%%%%%%%%%%%%%%%%%%%%%%%%%%%%%%%%%%%%%%%%%%
%%% The acknowledgements
\begin{acknowledgement}
To include acknowledgements use the \verb|acknowledgement| environment.
\end{acknowledgement}


%%%%%%%%%%%%%%%%%%%%%%%%%%%%%%%%%%%%%%%%%%%%%%%%%%%%%%%%%%%%%%%%%%%%%%%%%%%%%%%%%%%%%%%%%%%
%%% The bibliography
%
% BibTeX users please use
\bibliographystyle{spmpsci}
\bibliography{mybibfile}
% and then copy paste the contents of the .bbl file here for the final version.
%
% E.g.:
\begin{thebibliography}{99.}%

\bibitem{ACN2013}
N.~Achtsis, R.~Cools and D.~Nuyens.
\newblock Conditional sampling for barrier option pricing under the Heston model.
\newblock In J.~Dick, F.~Y.\ Kuo, G.~W.\ Peters and I.~H.\ Sloan, editors, {\em {M}onte {C}arlo
  and Quasi-{M}onte {C}arlo Methods 2012}, pages 253--269. Springer-Verlag, 2013.

\bibitem{CKN2006}
R.~Cools, F.~Y. Kuo, and D.~Nuyens.
\newblock Constructing embedded lattice rules for multivariate integration.
\newblock {\em SIAM Journal on Scientific Computing}, 28(6):2162--2188, 2006.

\bibitem{DP2010}
J.~Dick and F.~Pillichshammer.
\newblock {\em Digital Nets and Sequences: Discrepancy Theory and Quasi-Monte
  Carlo Integration}.
\newblock Cambridge University Press, 2010.

\bibitem{IT2006}
J.~{Imai} and K.~S.\ {Tan}.
\newblock A general dimension reduction technique for derivative pricing.
\newblock {\em Journal of Computational Finance}, 10(2):129--155, 2006.
        
\bibitem{LEC2009}
 P.~L'{\'E}cuyer.
\newblock Quasi-Monte Carlo methods with applications in finance.
\newblock {\em Finance and Stochastics}, 13(3):307--349, 2009.

\end{thebibliography}
\end{document}
