\documentclass[]{elsarticle}
\setlength{\marginparwidth}{0.5in}
\usepackage{amsmath,amssymb,amsthm,natbib,mathtools,bbm,graphicx}
\input FJHDef.tex

\newcommand{\fudge}{\fC}

\begin{document}

\begin{frontmatter}

\title{Error Estimation for Quasi-Monte Carlo Methods}
\author{Fred J. Hickernell}
\address{Room E1-208, Department of Applied Mathematics, Illinois Institute of Technology,\\ 10 W.\ 32$^{\text{nd}}$ St., Chicago, IL 60616}
\begin{abstract} 
\end{abstract}

\begin{keyword}
%% keywords here, in the form: keyword \sep keyword

%% MSC codes here, in the form: \MSC code \sep code
%% or \MSC[2008] code \sep code (2000 is the default)

\end{keyword}
\end{frontmatter}

\section{Bases and Node Sets}

\subsection{Group-Like Structures}
Consider the half open $d$-dimensional unit cube, $\cx:=[0,1)^d$, on which the functions of interest are to be defined. Suppose that there exists a commutative unidal structure on $\cx$, i.e., there exists a commutative addition operation $\oplus:\cx \times \cx \to \cx$ with identity element $\vzero$ (the zero vector): 
\[
\vx \oplus \vt = \vt \oplus \vx, \quad \vx \oplus \vzero =\vx \qquad \forall \vx,\vt \in \cx.
\]
Every $\vx \in \cx$ is assumed to have a unique additive inverse,  denoted $\ominus \vx$, and $\vx \ominus \vt$ means $\vx \oplus (\ominus \vt)$.  Thus, $\vx \ominus \vx = \vzero$.  Associativity is not assumed, and so there may exist $\vt \in \cx$, $\vt \ne \ominus \vx$, such that $\vx \oplus \vt = \vzero$.  This means that $\cx$ might not be a group.  

However, it is assumed that for some subsets of $\cx$, denoted $\tcx$, which are closed under $\oplus$ and for which associativity also holds:
\begin{equation} \label{assocprop}
\vx \oplus (\vt \oplus \vu) = (\vx \oplus \vt) \oplus \vu \qquad \forall \vx,\vt,\vu \in \tcx.
\end{equation}
As a consequence, such subsets, $\tcx$, are commutative groups.

Let $\bbK$ denote some subset of the $d$-dimensional vector of integers that contains $\vzero$.  Important examples are the set of integer vectors, $\integers^d$, and the set of non-negative integer vectors, $\natzero^d$.  The set $\bbK$ is used to index the series expressions for the functions to be integrated.  Suppose also that there exists an Abelian group structure on $\bbK$, with the additive operation $\oplus$.  Moreover, assume that there exists an operation $\otimes: \bbK \times \cx \to [0,1)$ that returns zero if either argument is zero and also has a distributive property:
\begin{subequations} \label{distrib}
\begin{gather}
\vk \otimes \vzero = \vzero \otimes \vx = 0 \qquad \forall \vk \in \bbK, \vx \in \cx,\\
\vk \otimes (\vx \oplus \vt) = (\vk\otimes\vx) + (\vk\otimes\vt) \pmod 1 \qquad \forall \vk \in \bbK, \vx \in \cx, \vt \in \tcx, \\
(\vk \oplus \vl) \otimes \vx = (\vk\otimes\vx) + (\vl\otimes\vx) \pmod 1 \qquad \forall \vk,\vl \in \bbK, \vx \in \cx.
\end{gather}
\end{subequations}

\subsection{Examples of Group-Like Structures}
The general notation introduced in the previous subsection and continued in the subsections below is intended to include the algebra behind both \emph{integration lattices} and \emph{digital nets}.  This subsection defines these two special kinds of operators $\oplus$, $\ominus$ and $\otimes$.

Integration lattices are sets that are closed under addition and subtraction modulo one.  In this setting $\bbK=\integers^d$, and
\begin{gather*}
\vx \oplus \vt = \vx + \vt \pmod 1, \qquad \ominus \vx = -\vx \pmod 1 \qquad \forall \vx, \vt \in \cx, \\
\vk \oplus \vl = \vk + \vl, \qquad \ominus \vk = -\vk \qquad \forall \vk, \vl \in \bbK, \\
\vk \otimes \vx = \vk^T \vx \pmod 1 \qquad \forall \vx \in \cx, \vk \in \bbK.
\end{gather*}
All the properties of the previous section can be shown to hold.  Specifically, associativity, \eqref{assocprop}, and the distributive property, \eqref{distrib}, hold for $\tcx = \cx=[0,1)^d$, so $\cx$ is a group.  

The digital net setting deals with $b$-ary expansions of $\cx$, where $b$ is prime, and $\bbK=\naturals_0^d$.  Let $\vx=(x_1, \ldots, x_d)$, and let $x_j={}_b 0.x_{j1} x_{j2} \cdots $ be the proper $b$-ary expansion (no infinite trail of $b-1$s) of $x_j \in [0,1)$.  Furthermore, let $\vk=(k_1, \ldots, k_d)$, and let $k_j=(\cdots k_{j2} k_{j1})_b$ be the $b$-ary expansion of $k_j \in \naturals_0$.  Specifically
\begin{gather*}
\vx = \left(\sum_{\ell=1}^{\infty} x_{j\ell} b^{-\ell}\right)_{j=1}^d, \qquad \ominus \vx = \left(\sum_{\ell=1}^{\infty} [-x_{j\ell} \bmod b] b^{-\ell}\right)_{j=1}^d \qquad \forall \vx \in \cx  \\ 
\vx \oplus \vt = \left(\sum_{\ell=1}^{\infty} [x_{j\ell} + t_{j\ell} \bmod b] b^{-\ell}\right)_{j=1}^d \qquad \forall \vx, \vt \in \cx, \\
\vk =  \left(\sum_{\ell=0}^{\infty} k_{j\ell} b^{\ell}\right)_{j=1}^d, \qquad \ominus \vk = \left(\sum_{\ell=0}^{\infty} [-k_{j\ell} \bmod b] b^{\ell}\right)_{j=1}^d \qquad \forall \vk \in \bbK, \\
\vk \oplus \vl = \left(\sum_{\ell=0}^{\infty} [k_{j\ell} + l_{j\ell} \bmod b] b^{\ell}\right)_{j=1}^d \qquad \forall \vk, \vl \in \bbK, \\
\vk \otimes \vx = \left( \left[\frac 1b \sum_{\ell=0}^{\infty} k_{j\ell}x_{j,\ell+1} \right] \bmod 1 \right)_{j=1}^d \qquad \forall \vx \in \cx, \vk \in \bbK.
\end{gather*}

What is $\tcx$?

\subsection{Fourier Series}

The integrands are assumed to belong to some subset of $\cl_2(\cx)$, the space of square integrable functions.  The $\cl_2$ inner product is defined as 
\[
\ip[2]{f}{g} = \int_{\cx} f(\vx) \overline{g(\vx)} \, \dif \vx.
\]
Let $\{\varphi(\cdot,\vk) \in \cl_2(\cx) : \vk \in \bbK\}$ be some complete orthonormal \emph{basis} for $\cl_2(\cx)$. In particular, let 
\[
\varphi(\vx,\vk)  = \me^{2 \pi \sqrt{-1} \vk \otimes \vx}, \qquad \vk \in \bbK, \vx \in \cx.
\]
Then any function in $\cl_2$ may be written in series form as
\begin{equation} \label{Fourierdef}
f(\vx) = \sum_{\vk \in \bbK} \hf(\vk) \varphi(\vx,\vk), \quad \text{where } \hf(\vk) = \ip[2]{f}{\varphi(\cdot,\vk)},
\end{equation}
and the inner product of two functions in $\cl_2$ is the $\ell_2$ inner product of their series coefficients:
\[
\ip[2]{f}{g} = \sum_{\vk \in \bbK} \hf(\vk)\overline{\hg(\vk)} =: \ip[2]{\bigl(\hf(\vk)\bigr)_{\vk \in \bbK}}{\bigl ( \hg(\vk)\bigr )_{\vk \in \bbK}}.
\]

\subsection{Node Sets and Their Dual Sets} Now suppose that $\cp$ is any finite subgroup of $\tcx$ with cardinality $\abs{\cp}$.  This will be called a \emph{node set}  It then follows that for all $\vk \in \bbK$ and $\vt \in \cp$,
\begin{align} 
\nonumber
0 & = \frac{1}{\abs{\cp}} \sum_{\vx \in \cp} [\varphi(\vx,\vk) - \varphi(\vx \oplus \vt,\vk)]  = \frac{1}{\abs{\cp}} \sum_{\vx \in \cp} [\me^{2 \pi \sqrt{-1} \vk \otimes \vx} - \me^{2 \pi \sqrt{-1} \vk \otimes (\vx \oplus \vt)}]\\
\nonumber
& = \frac{1}{\abs{\cp}} \sum_{\vx \in \cp} [\me^{2 \pi \sqrt{-1} \vk \otimes \vx} - \me^{2 \pi \sqrt{-1} \{(\vk \otimes \vx) + (\vk \otimes\vt)\}}] \quad \text{by } \eqref{distrib}\\
\label{sumeq}
& = [1 - \me^{2 \pi \sqrt{-1} \vk \otimes\vt)}] \frac{1}{\abs{\cp}} \sum_{\vx \in \cp} \me^{2 \pi \sqrt{-1} \vk \otimes \vx}.
\end{align}

Define the \emph{dual set} corresponding to $\cp$ as  
\begin{equation*}
\cp^{\perp} = \{\vk \in \bbK : \vk \otimes \vx = 0 \ \forall \vx \in \cp\}.
\end{equation*}
The distributive property, \eqref{distrib}, implies that dual set is a subgroup of $\bbK$.  By the equality \eqref{sumeq} above it follows that the average of a basis function, $\varphi(\cdot,\vk)$, over the points in a node set is either one or zero, depending on whether $\vk$ is in the dual set or not.
\begin{equation*}
 \frac{1}{\abs{\cp}} \sum_{\vx \in \cp} \me^{2 \pi \sqrt{-1} \vk \otimes \vx} = \bbone_{\cp^{\perp}}(\vk) = \begin{cases} 1 , & \vk \in \cp^{\perp}\\
 0,  & \vk \in \bbK \setminus \cp^{\perp}.
 \end{cases}
\end{equation*}
A \emph{shifted} node set is constructed by adding the same point $\vDelta \in \cx$ to each element in the node set: 
\begin{equation*}
\cp_{\vDelta} = \{ \vx + \vDelta : \vx \in \cp\}.
\end{equation*}
\begin{align*}
\frac{1}{\abs{\cp_{\vDelta}}} \sum_{\vx \in \cp_{\vDelta}} \me^{2 \pi \sqrt{-1} \vk \otimes \vx} 
& = \frac{1}{\abs{\cp}} \sum_{\vx \in \cp} \me^{2 \pi \sqrt{-1} \vk \otimes (\vx \oplus \vDelta)} = \frac{1}{\abs{\cp}} \sum_{\vx \in \cp} \me^{2 \pi \sqrt{-1} [(\vk \otimes \vx) + (\vk \otimes \vDelta)]}\\
&= \me^{2 \pi \sqrt{-1} \vk \otimes \vDelta} \bbone_{\cp^{\perp}}(\vk) = \begin{cases} \me^{2 \pi \sqrt{-1} \vk \otimes \vDelta} , & \vk \in \cp^{\perp}\\
 0,  & \vk \in \bbK \setminus \cp^{\perp}.
 \end{cases}
\end{align*}

\subsection{Discrete Transforms}
Define the discrete transform of a function, $f$, over the shifted node set $\cp_{\Delta}$ as 
\begin{align}
\label{tfdef}
\tf(\vk) 
&:= \frac{1}{\abs{\cp_{\vDelta}}} \sum_{\vx \in \cp_{\vDelta}} \me^{-2 \pi \sqrt{-1} \vk \otimes \vx} f(\vx) \\
\nonumber
&= \frac{1}{\abs{\cp_{\vDelta}}} \sum_{\vx \in \cp_{\Delta}} \left[\me^{-2 \pi \sqrt{-1} \vk \otimes \vx}\sum_{\vl \in \bbK} \hf(\vl) \me^{2 \pi \sqrt{-1} \vl \otimes \vx} \right] \\
\nonumber
& = \sum_{\vl  s\in \bbK} \hf(\vl)  \frac{1}{\abs{\cp_{\vDelta}}} \sum_{\vx \in \cp_{\vDelta}}  \me^{2 \pi \sqrt{-1} (\vl \ominus \vk) \otimes \vx} \\
\nonumber
& = \sum_{\substack{\vl \in \bbK \\ \vl \ominus \vk \in \cp^{\perp}}} \hf(\vl) \me^{2 \pi \sqrt{-1} (\vl \ominus \vk) \otimes \vDelta} \\
\nonumber
&= \sum_{\vm \in \cp^{\perp}} \hf(\vk\oplus\vm) \me^{2 \pi \sqrt{-1} \vm \otimes \vDelta}, \\
&= \hf(\vk) + \sum_{\vm \in \cp^{\perp}\setminus \vzero} \hf(\vk\oplus\vm) \me^{2 \pi \sqrt{-1} \vm \otimes \vDelta}, \qquad \forall \vk \in \bbK. \label{tfassum}
\end{align}
It is seen here that the discrete transform $\tf(\vk)$ is equal to the integral transform $\hf(\vk)$, defined in \eqref{Fourierdef}, plus the \emph{aliasing} terms corresponding to $\hf(\vl)$ where $\vl$ and $\vk$ differ (in the $\ominus$ sense) by a nonzero element of the dual set.

Notice that the dual nets can be used to form cosets of wavenumbers.  Let 
\begin{equation*}
\cp^{\perp}_{\vk} = \{ \vl \in \bbK : \vl \ominus \vk \in \cp^{\perp}\} \qquad \forall \vk \in \bbK.
\end{equation*}
This means that $\cp^{\perp}_{\vzero}=\cp^{\perp}$.  There are $\abs{\cp}$ distinct cosets.  Then \eqref{tfassum} above implies that 
\begin{equation}
\label{tfassumb}
\tf(\vk) = \sum_{\vl \in \cp^{\perp}_{\vk}} \hf(\vl) \me^{2 \pi \sqrt{-1} (\vl \ominus \vk) \otimes \vDelta}.
\end{equation}

\subsection{Nested Node Sets and Their Corresponding Nested Dual Sets}
Now consider the situation where there is a sequence of nested sets,
\[
\cp_0 = \{\vzero\} \subset \cp_{1} \subset \cp_{2} \subset \cdots, \qquad \abs{\cp_s} = b^s
\]
Furthermore, assume that each set equals the previous plus multiples of one element:
\begin{equation*}
\cp_s = \{ \vx \oplus \vt : \vx \in \cp_{s-1},\ \vt \in \{\vzero, \vz_s, \vz_s\oplus\vz_s, \ldots \} \}, \qquad s \in \naturals
\end{equation*}
where $\vz_1, \vz_2, \ldots \in \tcx$ is some fixed sequence.  According to this definition of nested sets, the dual sets are nested in the opposite direction,
\begin{equation*}
\cp_0^\perp = \bbK \supset \cp_{1}^\perp \supset \cp_{2}^\perp \supset \cdots.
\end{equation*}
Furthermore, the equivalence classes also obey this nesting:
\begin{equation*}
\cp^{\perp}_{s,\vk} = \{ \vl \in \bbK : \vl \ominus \vk \in \cp^{\perp}_s\}, \qquad \cp_{0,\vk}^\perp = \bbK \supset \cp_{1,\vk}^\perp \supset \cp_{2,\vk}^\perp \supset \cdots \quad \forall \vk \in \bbK.
\end{equation*}

\section{Error Estimate}

\subsection{Wavenumber Map}

Now we are going to map the non-negative numbers into the space of all wavenumbers using the dual sets.  For every $\kappa \in \natzero$, we assign a wavenumber $\vk(\kappa) \in \bbK$ iteratively according to the following constraints: 
\begin{itemize}

\item Let $\vk(0)= \vzero$.

\item For any $s,\lambda \in \naturals$, $\kappa=0, \ldots, b^s-1$, assign $\vk(\kappa)$ and $\vk(\kappa+ \lambda b^s)$ such that $\cp^{\perp}_{s,\vk(\kappa)}=\cp^{\perp}_{s,\vk(\kappa+ \lambda b^s)}$.
\end{itemize}

This wavenumber map allows us to introduce a shorthand notation that facilitates the later analysis:
\begin{align*}
\hf_{\kappa} & =\hf(\vk(\kappa)), \qquad \kappa \in \natzero,\\
\tf_{s,\kappa}& = \tf(\vk(\kappa)) \\
& = \frac{1}{b^s} \sum_{\vx \in \cp_{s,\vDelta}} \me^{-2 \pi \sqrt{-1} \vk(\kappa) \otimes \vx} f(\vx), \qquad \kappa=0, \ldots, b^s-1, \quad s \in \naturals,\\
\intertext{ as defined in \eqref{tfdef} based on the shifted nodeset $\cp_{s,\vDelta}$,}
\cp^{\perp}_{s,\kappa}&=\cp^{\perp}_{s,\vk(\kappa)}, \qquad \kappa=0, \ldots, b^s-1.
\end{align*}
According to \eqref{tfassumb}, it follows that 
\begin{align}
\nonumber
\tf_{s,\kappa} &= \sum_{\vl \in \cp^{\perp}_{s,\kappa}} \hf(\vl) \me^{2 \pi \sqrt{-1} (\vl \ominus \vk(\kappa)) \otimes \vDelta} \\
\nonumber
&= \sum_{\lambda=0}^{\infty} \hf_{\kappa+\lambda b^{s}} \me^{2 \pi \sqrt{-1} (\vk(\kappa+\lambda b^{s}) \ominus \vk(\kappa)) \otimes \vDelta} \\
& = \hf_{\kappa} + \sum_{\lambda=1}^{\infty} \hf_{\kappa+\lambda b^{s}} \me^{2 \pi \sqrt{-1} (\vk(\kappa+\lambda b^{s}) \ominus \vk(\kappa)) \otimes \vDelta}.
\label{tfassumc}
\end{align}
We want to use $\tf_{s,\kappa}$ to estimate $\hf_{\kappa}$ if $s$ is signficantly larger than $\lfloor \log_b(\kappa) \rfloor$.

\subsection{Sums of Series Coefficients and Their Bounds}
Consider the following sums of the series coefficients defined for $r,s \in \naturals$, $r \le s$:
\begin{equation}
S(r) =  \sum_{\kappa=b^{r-1}}^{b^{r}-1} \bigl \lvert \hf_{\kappa} \bigr \rvert, \quad 
\hS(r,s)  = \sum_{\kappa=b^{r-1}}^{b^{r}-1} \sum_{\lambda=1}^{\infty} \bigl \lvert \hf_{\kappa+\lambda b^{s}}\bigr\rvert, \quad 
\tS(r,s) = \sum_{\kappa=b^{r-1}}^{b^{r}-1} \bigl \lvert \tf_{s,\kappa}\bigr\rvert.
\end{equation}
These first two quantities, which involve the true series coefficients, cannot be observed, but the third one, which involves the discrete transform coefficients, can easily be observed.

We now make critical assumptions that $\hS(r,s)$ and $S(s)$ can be bounded above in terms of $S(r)$, provided that $r$ is large enough.  Fix $r_* \in \naturals$.  The assumptions are the following:
\begin{equation}
S(s) \le \omega(s-r) S(r), \quad \hS(r,s) \le \homega(s-r) S(r), \qquad r,s \in \naturals, \ r_* \le r \le s,
\end{equation}
for some functions $\omega$ and $\homega$ with $\lim_{s \to \infty} \omega(s) = \lim_{s \to \infty} \homega(s) = 0$.  

The reason for enforcing these assumptions only  for $r \ge r_*$ is that for small $r$, one might have $S(r)$ coincidentally small, since it only involves $b^r$ coefficients, while $S(s)$ or $\hS(r,s)$ is large.  If $S(s)$ is large compared to $S(r)$ for some $s > r$, it means that the true series coefficients for the integrand are large for some large wavenumbers.  If $\hS(r,s)$ is large compared to $S(r)$ for some $s > r$, it means that the obserbed discrete series coefficients may not correspond well to the true coefficients.

Under this assumption, for $r, s \in \naturals$, $r_* \le r \le s$, it is possible to bound the sum of the true coefficients, $S(r)$, in terms of the observed sum of the discrete coefficients, $\tS(r,s)$, as follows:
\begin{align*}
S(r) &= \sum_{\kappa=b^{r-1}}^{b^{r}-1} \bigl \lvert \hf_{\kappa}\bigr\rvert= \sum_{\kappa=b^{r-1}}^{b^{r}-1} \abs{\tf_{s,\kappa} - \sum_{\lambda=1}^{\infty} \hf_{\kappa+\lambda b^{s}} \me^{2 \pi \sqrt{-1} (\vl(\kappa+\lambda b^{s}) \ominus \vk(\kappa)) \otimes \vDelta}}\\
&\le \sum_{\kappa=b^{r-1}}^{b^{r}-1} \bigl \lvert \tf_{s,\kappa} \bigr\rvert + \sum_{\kappa=b^{r-1}}^{b^{r}-1} \sum_{\lambda=1}^{\infty} \bigl \lvert \hf_{\kappa+\lambda b^{s}}\bigr\rvert = \tS(r,s) + \hS(r,s) \\
&\le \tS(r,s) + \homega(s-r) S(r) \\
S(r) & \le \frac{\tS(r,s)}{1 - \homega(s-r)} \qquad \text{provided that } \homega(s-r) < 1.
\end{align*}

Using this upper bound, one can then conservatively bound the error of integration using the shifted node set $\cp_{s,\vDelta}$.  For for $r, s \in \naturals$, $r_* \le r \le s$, it follows that 
\begin{align*}
\MoveEqLeft{\abs{\int_{\cx} f(\vx) \, \dif \vx - \frac{1}{b^s} \sum_{\vx \in \cp_{s,\vDelta}} f(\vx) }}\\
&= \abs{\hf(\vzero) - \tf(\vzero)} = \abs{\hf_0 - \tf_{s,0}} = \abs{\sum_{\lambda=1}^{\infty} \hf_{\lambda b^{s}} \me^{2 \pi \sqrt{-1} \vl(\lambda b^{s}) \otimes \vDelta}}\\
&\le \sum_{\lambda=1}^{\infty} \bigl \lvert \hf_{\lambda b^{s}} \bigr \rvert \\
&\le \sum_{\kappa=b^{s}}^{\infty} \bigl \lvert \hf_{\kappa} \bigr \rvert = \sum_{r'=s+1}^{\infty} \sum_{\kappa=b^{r'-1}}^{b^{r'}-1} \bigl \lvert \hf_{\kappa} \bigr \rvert = \sum_{r'=s+1}^{\infty} S(r')\\
&\le \sum_{r'=s+1}^{\infty} \omega(r'-r) S(r) =   \sum_{r'=1}^{\infty} \omega(r'+s-r) S(r) =  \Omega(s-r) S(r)\\
& \le \frac{\tS(r,s)\Omega(s-r)}{1 - \homega(s-r)}.
\end{align*}
where 
\[
\Omega(\delta)=\sum_{r'=1}^{\infty} \omega(r'+ \delta), \qquad \delta \in \natzero.
\]
Assuming that $\Omega(0)$ is finite, $\lim_{\delta \to \infty} \Omega(\delta) = 0$.

This error bound suggests the following algorithm.  Choose $\delta \in \natzero$ such that $\homega(s-r)<1$ and set 
\[
\fudge = \frac{\Omega(\delta)}{1 - \homega(\delta)}.
\]
Define $r_j=r_*+j-1$ and $s_j=r_j+\delta$.  Given a tolerance $\varepsilon$, and an integrand $f$, do the following:  for $j=1, 2, \ldots$ check whether
\[
\fudge \tS(r_j,s_j) \le \varepsilon.
\]
If so, we're done.  If not, increment $j$ by one and repeat.

\bibliographystyle{model1b-num-names.bst}
\bibliography{FJH22,FJHown22}
\end{document}