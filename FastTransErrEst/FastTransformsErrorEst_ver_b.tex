\documentclass[]{elsarticle}
\setlength{\marginparwidth}{0.5in}
\usepackage{amsmath,amssymb,amsthm,natbib,mathtools,bbm,extraipa,accents,graphicx}
\input FJHDef.tex

\newcommand{\fudge}{\fC}
\newcommand{\dtf}{\textit{\doubletilde{f}}}
\newtheorem{lem}{Lemma}
\theoremstyle{definition}
\newtheorem{defin}{Definition}
\newtheorem{algo}{Algorithm}

\begin{document}

\begin{frontmatter}

\title{Error Estimation for Cubature Based on Digital Nets}
\author{Fred J. Hickernell}
\address{Room E1-208, Department of Applied Mathematics, Illinois Institute of Technology,\\ 10 W.\ 32$^{\text{nd}}$ St., Chicago, IL 60616}
\begin{abstract} 
\end{abstract}

\begin{keyword}
%% keywords here, in the form: keyword \sep keyword

%% MSC codes here, in the form: \MSC code \sep code
%% or \MSC[2008] code \sep code (2000 is the default)

\end{keyword}
\end{frontmatter}

\section{Bases and Node Sets}

\subsection{Group-Like Structures}
Consider the half open $d$-dimensional unit cube, $\cx:=[0,1)^d$, on which the functions of interest are to be defined. Suppose that there exists a commutative unidal structure on $\cx$, i.e., there exists a commutative addition operation $\oplus:\cx \times \cx \to \cx$ with identity element $\vzero$ (the zero vector): 
\[
\vx \oplus \vt = \vt \oplus \vx, \quad \vx \oplus \vzero =\vx \qquad \forall \vx,\vt \in \cx.
\]
Every $\vx \in \cx$ is assumed to have a unique additive inverse,  denoted $\ominus \vx$, and $\vx \ominus \vt$ means $\vx \oplus (\ominus \vt)$.  Thus, $\vx \ominus \vx = \vzero$.  Associativity is not assumed, and so there may exist $\vt \in \cx$, $\vt \ne \ominus \vx$, such that $\vx \oplus \vt = \vzero$.  This means that $\cx$ might not be a group.  

However, it is assumed that for some subsets of $\tcx \subseteq \cx$ associativity also holds, i.e., 
\begin{equation} \label{assocprop}
\vx \oplus (\vt \oplus \vu) = (\vx \oplus \vt) \oplus \vu \qquad \forall \vx,\vt,\vu \in \tcx.
\end{equation}
If such a subset $\tcx$ is closed under $\oplus$, then $\tcx$ is a commutative group.

Let $\bbK$ denote some subset of the $d$-dimensional vector of integers that contains $\vzero$.  Important examples are the set of integer vectors, $\integers^d$, and the set of non-negative integer vectors, $\natzero^d$.  The set $\bbK$ is used to index the series expressions for the functions to be integrated.  Suppose also that there exists an Abelian group structure on $\bbK$, with the additive operation $\oplus$.  

Moreover, assume that there exists a bilinear operator $\ip{\cdot}{\cdot}: \bbK \times \cx \to [0,1)$ that has a distributive property:
\begin{subequations} \label{distrib}
\begin{gather}
\ip{\vk}{\vzero} = \ip{\vzero}{\vx} = 0 \qquad \forall \vk \in \bbK, \ \vx \in \cx,\\
\vk \otimes (\vx \oplus \vt) = (\vk\otimes\vx) + (\vk\otimes\vt) \pmod 1 \qquad \forall \vk \in \bbK, \ \vx, \vt \in \tcx, \\
(\vk \oplus \vl) \otimes \vx = (\vk\otimes\vx) + (\vl\otimes\vx) \pmod 1 \qquad \forall \vk,\vl \in \bbK, \vx \in \cx,
\end{gather}
for all $\tcx$ for which that \eqref{assocprop} holds.
\end{subequations}

\subsection{Examples of Group-Like Structures}
The general notation introduced in the previous subsection and continued in the subsections below is intended to include the algebra behind both \emph{integration lattices} and \emph{digital nets}.  This subsection defines these two special kinds of operators $\oplus$, $\ominus$ and $\otimes$.

Integration lattices are sets that are closed under addition and subtraction modulo one.  In this setting $\bbK=\integers^d$, and
\begin{gather*}
\vx \oplus \vt = \vx + \vt \pmod 1, \qquad \ominus \vx = -\vx \pmod 1 \qquad \forall \vx, \vt \in \cx, \\
\vk \oplus \vl = \vk + \vl, \qquad \ominus \vk = -\vk \qquad \forall \vk, \vl \in \bbK, \\
\ip{\vk}{\vx} = \vk^T \vx \pmod 1 \qquad \forall \vx \in \cx, \vk \in \bbK.
\end{gather*}
All the properties of the previous section can be shown to hold.  Specifically, associativity, \eqref{assocprop}, and the distributive property, \eqref{distrib}, hold for $\tcx = \cx=[0,1)^d$, so $\cx$ is a group.  

The digital net setting deals with $b$-ary expansions of $\cx$, where $b$ is prime, and $\bbK=\naturals_0^d$.  Let $\bbF_b:=\{0, \ldots, b-1\}$.  For any $x \in [0,1)$ and any $k \in \natzero$, let $\ax$ and $\ak$ denote the sequence of digits of their respective proper binary expansions, i.e., 
\begin{gather*}
\ax=(x_1, x_2, \ldots ) \in \bbF_b^\infty \ \iff \ x=\sum_{\ell=1}^\infty x_\ell b^{-\ell}, \\
\ak=(k_0, k_1, \ldots ) \in \bbF_b^\infty \ \iff \ k=\sum_{\ell=0}^\infty k_\ell b^{\ell}.
\end{gather*}
Let $\vx=(x_1, \ldots, x_d)$, and let $\vk=(k_1, \ldots, k_d)$.  Then the operations are defined in terms of the $\ax_j$ and $\ak_j$.
\begin{gather*}
\vx = \left(\sum_{\ell=1}^{\infty} x_{j\ell} b^{-\ell}\right)_{j=1}^d, \qquad \ominus \vx = \left(\sum_{\ell=1}^{\infty} [-x_{j\ell} \bmod b] b^{-\ell}\right)_{j=1}^d \qquad \forall \vx \in \cx  \\ 
\vx \oplus \vt = \left(\sum_{\ell=1}^{\infty} [x_{j\ell} + t_{j\ell} \bmod b] b^{-\ell}\right)_{j=1}^d \qquad \forall \vx, \vt \in \cx, \\
\vk =  \left(\sum_{\ell=0}^{\infty} k_{j\ell} b^{\ell}\right)_{j=1}^d, \qquad \ominus \vk = \left(\sum_{\ell=0}^{\infty} [-k_{j\ell} \bmod b] b^{\ell}\right)_{j=1}^d \qquad \forall \vk \in \bbK, \\
\vk \oplus \vl = \left(\sum_{\ell=0}^{\infty} [k_{j\ell} + l_{j\ell} \bmod b] b^{\ell}\right)_{j=1}^d \qquad \forall \vk, \vl \in \bbK, \\
\ip{\vk}{\vx} = \frac 1b \sum_{j=1}^{d} \sum_{\ell=0}^{\infty} k_{j\ell}x_{j,\ell+1}  \pmod 1 \qquad \forall \vx \in \cx, \vk \in \bbK.
\end{gather*}
Here, $\tcx$ is any subset of $\cx$ for which any $\vt, \vx \in \cx$ satisfy 
\[
\forall j \in \{1, \ldots, d\},\ \ell \in \natzero,\  \exists \ell'>\ell \text{ such that } x_{j\ell} + t_{j\ell} \bmod b \ne b-1.
\]


\subsection{Fourier Series}

The integrands are assumed to belong to some subset of $\cl_2(\cx)$, the space of square integrable functions.  The $\cl_2$ inner product is defined as 
\[
\ip[2]{f}{g} = \int_{\cx} f(\vx) \overline{g(\vx)} \, \dif \vx.
\]
Let $\{\varphi(\cdot,\vk) \in \cl_2(\cx) : \vk \in \bbK\}$ be some complete orthonormal \emph{basis} for $\cl_2(\cx)$. In particular, let 
\[
\varphi(\vx,\vk)  = \me^{2 \pi \sqrt{-1} \ip{\vk}{\vx}}, \qquad \vk \in \bbK, \vx \in \cx.
\]
Then any function in $\cl_2$ may be written in series form as
\begin{equation} \label{Fourierdef}
f(\vx) = \sum_{\vk \in \bbK} \hf(\vk) \varphi(\vx,\vk), \quad \text{where } \hf(\vk) = \ip[2]{f}{\varphi(\cdot,\vk)},
\end{equation}
and the inner product of two functions in $\cl_2$ is the $\ell_2$ inner product of their series coefficients:
\[
\ip[2]{f}{g} = \sum_{\vk \in \bbK} \hf(\vk)\overline{\hg(\vk)} =: \ip[2]{\bigl(\hf(\vk)\bigr)_{\vk \in \bbK}}{\bigl ( \hg(\vk)\bigr )_{\vk \in \bbK}}.
\]

\subsection{Node Sets and Their Dual Sets}
Suppose that there exists a sequence of points in $\cx$, denoted $\cp_\infty =\{\vt_i\}_{i=0}^{\infty}$ that satisfies the associativity property and is closed under $\oplus$, and so is an abelian group.  We consider $\cp_\infty$ together with the field $\natzero$ to be a module.  Furthermore, $\cp_\infty$ is assumed to satisfy the following properties:
\begin{subequations} \label{cpinfvector}
\begin{gather}
\{\vt_1, \vt_b, \vt_{b^2}, \ldots\} \text{ is linearly independent under the field } \bbF_b, \\
\vt_0=\vzero, \qquad 
\vt_i = \sum_{\ell=0}^{\infty} i_\ell \vt_{b^\ell}, \qquad \text{where }\ai=(i_0, i_1, i_2, \ldots ) \in \bbF_b^\infty,\\
\text{where} \quad a \vt_{b^\ell} \text{ denotes } \underbrace{\vt_{b^\ell} \oplus \cdots \oplus \vt_{b^\ell}}_{a \text{ times}} \text{ for all } a \in \natzero, \\
b \vt_{b^{m}} \in \cp_m \qquad \text{for } m\in \natzero, \\
\cp_m := \{\vt_i\}_{i=0}^{b^m-1} \text{ is a subgroup of } \cp_\infty \text{ for } m\in \natzero, \\
\cp_m \text{ is a subgroup of } \cp_\ell \text{ for } \ell=m+1, m+2, \ldots, \ m\in \natzero, \\
\ip{\vk}{\vt_i} = 0 \quad \forall \vk \in \bbK \ \implies \ i=0.
\end{gather}
\end{subequations}
Any $\cp_m$ defined as above is called a \emph{node set}.

For $m \in \natzero$ define the \emph{dual set} corresponding to $\cp_m$ as  
\begin{align*}
\cp^{\perp}_m &= \{\vk \in \bbK : \ip{\vk}{\vt_i} = 0, \ i=0, \ldots, b^m-1\} \\
&= \{\vk \in \bbK : \ip{\vk}{\vt_{2^{\ell}}} = 0, \ \ell=0, \ldots, m-1\}.
\end{align*}
By this definition $\cp^{\perp}_{0}=\bbK$.  The distributive property, \eqref{distrib}, implies that the dual set $\cp^{\perp}_m$ is a subgroup of the dual set $\cp^{\perp}_\ell$ for all $\ell=0, \ldots, m-1$.  

For all $\vk \in \bbK$ and $\vx \in \cp$, it follows that
\begin{align*} 
\nonumber
0 & = \frac{1}{b^m} \sum_{i=0}^{b^m-1} [\varphi(\vt_i,\vk) - \varphi(\vt_i \oplus \vx,\vk)]  
= \frac{1}{b^m} \sum_{i=0}^{b^m-1} [\me^{2 \pi \sqrt{-1} \ip{\vk}{\vt_i}} - \me^{2 \pi \sqrt{-1} \ip{\vk}{\vt_i \oplus \vx}}]\\
\nonumber
& = \frac{1}{b^m} \sum_{i=0}^{b^m-1} [\me^{2 \pi \sqrt{-1} \ip{\vk}{\vt_i}} - \me^{2 \pi \sqrt{-1} \{\ip{\vk}{\vt_i}+\ip{\vk}{\vx}\}}] \quad \text{by } \eqref{distrib}\\
\label{sumeq}
& = [1 - \me^{2 \pi \sqrt{-1} \ip{\vk}{\vx})}] \frac{1}{b^m} \sum_{i=0}^{b^m-1}  \me^{2 \pi \sqrt{-1} \ip{\vk}{\vt_i}}.
\end{align*}
By this equality it follows that the average of a basis function, $\varphi(\cdot,\vk)$, over the points in a node set is either one or zero, depending on whether $\vk$ is in the dual set or not.
\begin{equation*}
\frac{1}{b^m} \sum_{i=0}^{b^m-1}  \me^{2 \pi \sqrt{-1} \ip{\vk}{\vt_i}} = \bbone_{\cp_m^{\perp}}(\vk) = \begin{cases} 1 , & \vk \in \cp_m^{\perp}\\
 0,  & \vk \in \bbK \setminus \cp_m^{\perp}.
 \end{cases}
\end{equation*}

The next goal is to enumerate cosets of wavenumbers, and do so in a way that facilitates calculation of the discrete transform.  These cosets, denoted $\cp_{m,\nu}^{\perp}$ are defined for $m\in \natzero$, and $\nu \in \{0, \ldots, b^m-1\}$.  Moreover, as shown in Lemma \ref{cosetlem}, the dual sets are identical to the $\cp_{m,0}^{\perp}$.

\begin{defin} \label{noodefalgo} Let $\cp_{0,0}^{\perp}=\cp_{0}^{\perp}=\bbK$.  For $m=0, 1, 2, \ldots$, 
\begin{enumerate}
\renewcommand{\labelenumi}{\alph{enumi})}
\item Define $\cp_{m+1,0}^{\perp}=\{\vk \in \cp_{m,0}^{\perp} : \ip{\vk}{\vt_{b^{m}}}=0\}$.
\item Define $\cp_{m+1,b^{m}}^{\perp}=\{\vk \in \cp_{m,0}^{\perp} : \ip{\vk}{\vt_{b^{m}}}=1/b\}$.  
\item If $m\ge 1$, for all $\ell=0, \ldots, m-1$, define 
\[
\cp_{m+1,b^{\ell}}^{\perp}=\{\vk \in \cp_{m,b^{\ell}}^{\perp} : \ip{\vk}{\vt_{b^{m}}}<1/b\}.
\]
\item For all $\nu \in \{2, \ldots, b^{m+1}-1\} \setminus \{b, b^2, \ldots, b^m\}$, with $\anu=(\nu_0, \nu_1, \ldots)$ define 
\[
\cp_{m+1,\nu}^{\perp}= \{\nu_0\vk_0 \oplus \cdots \oplus \nu_{m} \vk_{m}  : \vk_\ell \in \cp_{m+1,b^{\ell}}^{\perp},\  \ell=0, \ldots, m\}.
\]
\end{enumerate}
\end{defin}

\begin{lem} \label{cosetlem} The above procedure uniquely defines $\cp_{m,\nu}^{\perp}$  for all $\nu \in \{0, \ldots, b^m-1\}$ and all $m \in \natzero$.  Moreover, for all $m \in \natzero$, the following are true.
\begin{enumerate}
\renewcommand{\labelenumi}{\alph{enumi})}

\item The vector $\vzero$ always lies in $\cp_{m,0}^{\perp}$.

\item The sets $\cp_{m,\nu}^{\perp}$ are disjoint and their union is $\bbK$.

\item If $\cp_{m,\nu \boxplus \lambda}^{\perp} = \cp_{m,\nu}^{\perp} \oplus \cp_{m,\lambda}^{\perp} := \{\vk \oplus \vl :  \vk \in \cp_{m,\nu}^{\perp},\ \vl \in \cp_{m,\lambda}^{\perp}\}$.

\item For $m>0$ and any $\vk \in \cp_{m,\ve_\ell}$, define $w_{\ell,m}=\ip{\vk}{\vt_{b^m}}$.  This is also uniquely defined.  Moreover,
\begin{equation} \label{nuwisum}
\vk\in \cp_{m,\vnu} \text{ and } \ai=(i_1, i_2, \ldots) \implies{} \ip{\vk}{\vt_i} = \sum_{\ell,m=1}^{\infty} \nu_\ell w_{\ell,m} i_m \pmod 1.
\end{equation}

\end{enumerate}
\end{lem}





The next goal is to enumerate cosets of wavenumbers, and do so in a way that facilitates calculation of the discrete transform.  These cosets, denoted $\cp_{m,\vnu}^{\perp}$ are defined for $m\in \natzero$, and $\vnu \in \bbF_b^m$.  For flexibility of notation, $\vnu$ may have more than $m$ entries, but the definition of $\cp_{m,\vnu}^{\perp}$ only depends on $\nu_1, \ldots, \nu_m$. In the definition below $\ve_m$ denotes the vector with one in the $m^{\text{th}}$ entry and zeroes in all other entries.  Moreover, as shown in Lemma \ref{cosetlem}, the dual sets are identical to the $\cp_{m,\vzero}^{\perp}$.

\begin{defin} \label{noodefalgo} Let $\cp_{0,\vnu}^{\perp}=\cp_{0}^{\perp}=\bbK$.  For $m=0, 1, 2, \ldots$, 
\begin{enumerate}
\renewcommand{\labelenumi}{\alph{enumi})}
\item Define $\cp_{m+1,\vzero}^{\perp}=\{\vk \in \cp_{m,\vzero}^{\perp} : \ip{\vk}{\vt_{b^{m}}}=0\}$.
\item Define $\cp_{m+1,\ve_{m+1}}^{\perp}=\{\vk \in \cp_{m,\vzero}^{\perp} : \ip{\vk}{\vt_{b^{m}}}=1/b\}$.  
\item If $m\ge 1$, for all $\ell=1, \ldots, m$, define 
\[
\cp_{m+1,\ve_{\ell}}^{\perp}=\{\vk \in \cp_{m,\ve_\ell}^{\perp} : \ip{\vk}{\vt_{b^{m}}}<1/b\}.
\]
\item For all $\vnu \in \bbF_b^m \setminus \{\vzero, \ve_1, \ldots, \ve_m\}$, define 
\[
\cp_{m+1,\vnu}^{\perp}= \{\nu_1\vk_1 \oplus \cdots \oplus \nu_{m+1} \vk_{m+1}  : \vk_\ell \in \cp_{m+1,\ve_{\ell}}^{\perp},\  \ell=1, \ldots, m+1\}.
\]
\end{enumerate}
\end{defin}

\begin{lem} \label{cosetlem} The above procedure uniquely defines $\cp_{m,\vnu}^{\perp}$  for all $\vnu \in \bbF_b^m$ and all $m \in \natzero$.  Moreover, for all $m \in \natzero$, the following are true.
\begin{enumerate}
\renewcommand{\labelenumi}{\alph{enumi})}

\item The vector $\vzero$ always lies in $\cp_{m,\vzero}^{\perp}$.

\item The sets $\cp_{m,\vnu}^{\perp}$ are disjoint and their union is $\bbK$.

\item If $\cp_{m,\vnu + \vlambda \bmod b}^{\perp} = \cp_{m,\vnu}^{\perp} \oplus \cp_{m,\vlambda}^{\perp} := \{\vk \oplus \vl :  \vk \in \cp_{m,\vnu}^{\perp},\ \vl \in \cp_{m,\vlambda}^{\perp}\}$.

\item For $m>0$ and any $\vk \in \cp_{m,\ve_\ell}$, define $w_{\ell,m}=\ip{\vk}{\vt_{b^m}}$.  This is also uniquely defined.  Moreover,
\begin{equation} \label{nuwisum}
\vk\in \cp_{m,\vnu} \text{ and } \ai=(i_1, i_2, \ldots) \implies{} \ip{\vk}{\vt_i} = \sum_{\ell,m=1}^{\infty} \nu_\ell w_{\ell,m} i_m \pmod 1.
\end{equation}

\end{enumerate}
\end{lem}




\begin{proof} The proof proceeds by induction.  First consider the case of $m=0$, in which case $\cp_{m,\vnu}^{\perp}$ is uniquely defined as $\bbK$, and a)--??) hold automatically and ?) is vacuous. 

Now suppose that Lemma \ref{cosetlem} holds for $m$.  We show that it also holds for $m+1$.  By the induction hypothesis, $\vzero \in \cp_{m,\vzero}^{\perp}$, and since $\ip{\vzero}{\vt_{b^{m}}}=0$ by \eqref{distrib}, it follows that $\vzero \in \cp_{m+1,\vzero}^{\perp}$, which corresponds to a).

There must be some $\vk \in  \cp_{m,\vzero}^{\perp}$ for which $\ip{\vzero}{\vt_{b^{m}}}\ne 0$.  Otherwise,  












The term $w_{1,1} = \ip{\vk}{\vt_{1}}$ is well-defined because if 
\[
1=\hnu_1(\vk)=\hnu_1(\vk')=b \ip{\vk}{\vt_{1}} \bmod b = b \ip{\vk'}{\vt_{1}} \bmod b,
\]
then it follows that 
\[
nn
\]
 definition of  we define $w_{1,1}$ Suppose that 
{\bf We will finish the proof later.}
\end{proof}


Now define a map $\hvnu: \bbK \to \bbF_b^\infty$ denoted $\hvnu(\vk)=(\hnu_1(\vk), \hnu_2(\vk), \hnu_3(\vk), \ldots )$ as follows:
\[
\hvnu(\vk)=\vnu \ \iff \ \vk \in \cp_{m,\vnu}^{\perp}, \qquad m \in \natzero.
\]
According to this definition it follows that $\cp_{m}^{\perp}=\cp_{m,\vzero}^{\perp}$.  Moreover, $\cp_{0,\vzero}^{\perp}=\cp_{0}^{\perp}=\bbK$, by convention.  Defining $\hvnu(\vk)$ for all $\vk \in \bbK$ is equivalent to defining $\cp_{m,\vnu}^{\perp}$ for all $m \in \natzero$ and all $\vnu \in \bbF_b$.  It is shown that
\begin{equation} \label{nuwisum}
\hvnu(\vk)=(\nu_1, \nu_2, \ldots),\  \ai=(i_1, i_2 \ldots) \implies{} \ip{\vk_i}{\vt_i} = \sum_{\ell,m=1}^{\infty} \nu_\ell w_{\ell,m} i_m \pmod 1,
\end{equation}
where the $w_{\ell,m}$ are defined below.  

A \emph{shifted} node set is constructed by adding the same point $\vDelta \in \cx$ to each element in the node set: 
\begin{equation*}
\cp_{m,\vDelta} = \{ \vt_i \oplus \vDelta : i=0, \ldots, b^m -1\}.
\end{equation*}
It then follows that 
\begin{align*}
\frac{1}{b^m} \sum_{\vx \in \cp_{m,\vDelta}} \me^{2 \pi \sqrt{-1} \ip{\vk}{\vx}}
& = \frac{1}{b^m} \sum_{i=0}^{b^m-1} \me^{2 \pi \sqrt{-1} \ip{\vk}{\vt_i \oplus \vDelta}}\\
& = \frac{1}{b^m} \sum_{i=0}^{b^m-1} \me^{2 \pi \sqrt{-1} [\ip{\vk}{\vt_i} + \ip{\vk} {\vDelta}]}\\
&= \me^{2 \pi \sqrt{-1} \ip{\vk}{\vDelta}} \bbone_{\cp_m^{\perp}}(\vk) = \begin{cases} \me^{2 \pi \sqrt{-1} \ip{\vk}{\vDelta}} , & \vk \in \cp_m^{\perp}\\
 0,  & \vk \in \bbK \setminus \cp_m^{\perp}.
 \end{cases}
\end{align*}

\subsection{Discrete Transforms}
Define the discrete transform of a function, $f$, over the shifted node set $\cp_{\Delta}$ as 
\begin{align}
\label{tfdef}
\tf(\vk) 
&:= \frac{1}{b^m} \sum_{\vx \in \cp_{m,\vDelta}} \me^{-2 \pi \sqrt{-1} \ip{\vk}{\vx}} f(\vx) \\
\nonumber
&= \frac{1}{b^m}  \sum_{\vx \in \cp_{m,\Delta}} \left[\me^{-2 \pi \sqrt{-1} \ip{\vk}{\vx}}\sum_{\vl \in \bbK} \hf(\vl) \me^{2 \pi \sqrt{-1} \ip{\vl}{\vx}} \right] \\
\nonumber
& = \sum_{\vl \in \bbK} \hf(\vl)  \frac{1}{b^m}  \sum_{\vx \in \cp_{m,\vDelta}}  \me^{2 \pi \sqrt{-1} \ip{\vl \ominus \vk}{\vx}} \\
\nonumber
& = \sum_{\substack{\vl \in \bbK \\ \vl \ominus \vk \in \cp_m^{\perp}}} \hf(\vl) \me^{2 \pi \sqrt{-1} \ip{\vl \ominus \vk}{\vDelta}}= \sum_{\vl \in \cp_{m,\hvnu(\vk)}^{\perp}} \hf(\vl) \me^{2 \pi \sqrt{-1} \ip{\vl \ominus \vk}{\vDelta}} \\
\nonumber
&= \hf(\vk) + \sum_{\vl \in \cp_{m,\hvnu(\vk)}^{\perp}\setminus\vk} \hf(\vl) \me^{2 \pi \sqrt{-1} \ip{\vl \ominus \vk}{\vDelta}}\\
&= \hf(\vk) + \sum_{\vl \in \cp^{\perp}\setminus \vzero} \hf(\vk\oplus\vl) \me^{2 \pi \sqrt{-1} \ip{\vl}{\vDelta}}, \qquad \forall \vk \in \bbK. \label{tfassum}
\end{align}
It is seen here that the discrete transform $\tf(\vk)$ is equal to the integral transform $\hf(\vk)$, defined in \eqref{Fourierdef}, plus the \emph{aliasing} terms corresponding to $\hf(\vl)$ where $\vl \in \cp_{m,\hvnu(\vk)}^{\perp}\setminus \vk$.


\subsection{Computation of the Discrete Transform}
The discrete transform defined in \eqref{tfdef} may also be expressed as
\begin{align}
\nonumber
\tf(\vk) 
&= \frac{1}{b^m} \sum_{i=0}^{b^m-1} \me^{-2 \pi \sqrt{-1} \ip{\vk}{\vt_i\oplus \vDelta}} f(\vt_i\oplus \vDelta) \\
\nonumber
&= \frac{\me^{-2 \pi \sqrt{-1} \ip{\vk}{\vDelta}}}{b^m} \sum_{i=0}^{b^m-1} \me^{-2 \pi \sqrt{-1} \ip{\vk}{\vt_i}} f(\vt_i\oplus \vDelta).
\end{align}
Letting $y_i=f(\vt_i\oplus \vDelta)$, 
\[
Y_{m,0}(i_1,\ldots, i_m) = y_i, \qquad i=i_1 + i_2 b + \cdots + i_m b^{m-1},
\]
and invoking Lemma \ref{cosetlem}, for any $\vk \in \cp_{m,\vnu}^{\perp}$ one may write
\begin{align}
\nonumber
\tf(\vk) &= \me^{-2 \pi \sqrt{-1} \ip{\vk}{\vDelta}}  Y_{m,m}(\nu_1, \ldots, \nu_m) , \\
\nonumber
\MoveEqLeft{Y_{m,m}(\nu_1, \ldots, \nu_m)}\\ 
\nonumber
& : = \frac{1}{b^m} \sum_{i=0}^{b^m-1} \me^{-2 \pi \sqrt{-1} \ip{\vk}{\vt_i}} y_i \\
\nonumber
& = \frac{1}{b^m} \sum_{i_m=0}^{b-1} \cdots \sum_{i_1=0}^{b-1} \me^{-2 \pi \sqrt{-1} \sum_{r=1}^{m} \sum_{\ell=1}^{r} \nu_\ell w_{\ell,r} i_r} Y_{m,0}(i_1,\ldots, i_m) \\
\nonumber
& = \frac{1}{b} \sum_{i_m=0}^{b-1}\me^{-2 \pi \sqrt{-1} \sum_{\ell=1}^{m} \nu_\ell w_{\ell,m} i_m}  \cdots \\
&\qquad \qquad \frac{1}{b} \sum_{i_1=0}^{b-1} \me^{-2 \pi \sqrt{-1} \sum_{\ell=1}^{1} \nu_\ell w_{\ell,1} i_1} Y_{m,0}(i_1,\ldots, i_m) \\
\nonumber
\end{align}

This sum can be computed recursively:
\begin{multline*}
Y_{m,r}(\nu_1, \ldots, \nu_r,i_{r+1}, \ldots, i_m) \\
= \frac{1}{b} \sum_{i_r=0}^{b-1} \me^{-2 \pi \sqrt{-1} i_r \sum_{\ell=1}^{r} \nu_\ell w_{\ell,r} } Y_{m,r-1}(\nu_1, \ldots, \nu_{r-1},i_{r}, \ldots, i_m)
\end{multline*}



Inspired by \eqref{cosetlem} we define the following:

\section{Error Estimation and an Automatic Algorithm}

\subsection{Wavenumber Map}

Now we are going to map the non-negative numbers into the space of all wavenumbers using the dual sets.  For every $\kappa \in \natzero$, we assign a wavenumber $\vk(\kappa) \in \bbK$ iteratively according to the following constraints: 
\begin{itemize}

\item Let $\vk(0)= \vzero$.

\item For any $s,\lambda \in \naturals$, $\kappa=0, \ldots, b^s-1$, assign $\vk(\kappa)$ and $\vk(\kappa+ \lambda b^s)$ such that $\cp^{\perp}_{s,\vk(\kappa)}=\cp^{\perp}_{s,\vk(\kappa+ \lambda b^s)}$.
\end{itemize}

This wavenumber map allows us to introduce a shorthand notation that facilitates the later analysis:
\begin{align*}
\hf_{\kappa} & =\hf(\vk(\kappa)), \qquad \kappa \in \natzero,\\
\tf_{s,\kappa}& = \tf(\vk(\kappa)) \\
& = \frac{1}{b^s} \sum_{\vx \in \cp_{s,\vDelta}} \me^{-2 \pi \sqrt{-1} \vk(\kappa) \otimes \vx} f(\vx), \qquad \kappa=0, \ldots, b^s-1, \quad s \in \naturals,\\
\intertext{ as defined in \eqref{tfdef} based on the shifted nodeset $\cp_{s,\vDelta}$,}
\cp^{\perp}_{s,\kappa}&=\cp^{\perp}_{s,\vk(\kappa)}, \qquad \kappa=0, \ldots, b^s-1.
\end{align*}
According to \eqref{tfassumb}, it follows that 
\begin{align}
\nonumber
\tf_{s,\kappa} &= \sum_{\vl \in \cp^{\perp}_{s,\kappa}} \hf(\vl) \me^{2 \pi \sqrt{-1} (\vl \ominus \vk(\kappa)) \otimes \vDelta} \\
\nonumber
&= \sum_{\lambda=0}^{\infty} \hf_{\kappa+\lambda b^{s}} \me^{2 \pi \sqrt{-1} (\vk(\kappa+\lambda b^{s}) \ominus \vk(\kappa)) \otimes \vDelta} \\
& = \hf_{\kappa} + \sum_{\lambda=1}^{\infty} \hf_{\kappa+\lambda b^{s}} \me^{2 \pi \sqrt{-1} (\vk(\kappa+\lambda b^{s}) \ominus \vk(\kappa)) \otimes \vDelta}.
\label{tfassumc}
\end{align}
We want to use $\tf_{s,\kappa}$ to estimate $\hf_{\kappa}$ if $s$ is signficantly larger than $\lfloor \log_b(\kappa) \rfloor$.

\subsection{Sums of Series Coefficients and Their Bounds}
Consider the following sums of the series coefficients defined for $r,s \in \naturals$, $r \le s$:
\begin{equation}
S(r) =  \sum_{\kappa=b^{r-1}}^{b^{r}-1} \bigl \lvert \hf_{\kappa} \bigr \rvert, \quad 
\hS(r,s)  = \sum_{\kappa=b^{r-1}}^{b^{r}-1} \sum_{\lambda=1}^{\infty} \bigl \lvert \hf_{\kappa+\lambda b^{s}}\bigr\rvert, \quad 
\tS(r,s) = \sum_{\kappa=b^{r-1}}^{b^{r}-1} \bigl \lvert \tf_{s,\kappa}\bigr\rvert.
\end{equation}
These first two quantities, which involve the true series coefficients, cannot be observed, but the third one, which involves the discrete transform coefficients, can easily be observed.

We now make critical assumptions that $\hS(r,s)$ and $S(s)$ can be bounded above in terms of $S(r)$, provided that $r$ is large enough.  Fix $r_* \in \naturals$.  The assumptions are the following:
\begin{equation}
S(s) \le \omega(s-r) S(r), \quad \hS(r,s) \le \homega(s-r) S(r), \qquad r,s \in \naturals, \ r_* \le r \le s,
\end{equation}
for some functions $\omega$ and $\homega$ with $\lim_{s \to \infty} \omega(s) = \lim_{s \to \infty} \homega(s) = 0$.  

The reason for enforcing these assumptions only  for $r \ge r_*$ is that for small $r$, one might have $S(r)$ coincidentally small, since it only involves $b^r$ coefficients, while $S(s)$ or $\hS(r,s)$ is large.  If $S(s)$ is large compared to $S(r)$ for some $s > r$, it means that the true series coefficients for the integrand are large for some large wavenumbers.  If $\hS(r,s)$ is large compared to $S(r)$ for some $s > r$, it means that the obserbed discrete series coefficients may not correspond well to the true coefficients.

Under this assumption, for $r, s \in \naturals$, $r_* \le r \le s$, it is possible to bound the sum of the true coefficients, $S(r)$, in terms of the observed sum of the discrete coefficients, $\tS(r,s)$, as follows:
\begin{align*}
S(r) &= \sum_{\kappa=b^{r-1}}^{b^{r}-1} \bigl \lvert \hf_{\kappa}\bigr\rvert= \sum_{\kappa=b^{r-1}}^{b^{r}-1} \abs{\tf_{s,\kappa} - \sum_{\lambda=1}^{\infty} \hf_{\kappa+\lambda b^{s}} \me^{2 \pi \sqrt{-1} (\vl(\kappa+\lambda b^{s}) \ominus \vk(\kappa)) \otimes \vDelta}}\\
&\le \sum_{\kappa=b^{r-1}}^{b^{r}-1} \bigl \lvert \tf_{s,\kappa} \bigr\rvert + \sum_{\kappa=b^{r-1}}^{b^{r}-1} \sum_{\lambda=1}^{\infty} \bigl \lvert \hf_{\kappa+\lambda b^{s}}\bigr\rvert = \tS(r,s) + \hS(r,s) \\
&\le \tS(r,s) + \homega(s-r) S(r) \\
S(r) & \le \frac{\tS(r,s)}{1 - \homega(s-r)} \qquad \text{provided that } \homega(s-r) < 1.
\end{align*}

Using this upper bound, one can then conservatively bound the error of integration using the shifted node set $\cp_{s,\vDelta}$.  For for $r, s \in \naturals$, $r_* \le r \le s$, it follows that 
\begin{align*}
\MoveEqLeft{\abs{\int_{\cx} f(\vx) \, \dif \vx - \frac{1}{b^s} \sum_{\vx \in \cp_{s,\vDelta}} f(\vx) }}\\
&= \abs{\hf(\vzero) - \tf(\vzero)} = \abs{\hf_0 - \tf_{s,0}} = \abs{\sum_{\lambda=1}^{\infty} \hf_{\lambda b^{s}} \me^{2 \pi \sqrt{-1} \vl(\lambda b^{s}) \otimes \vDelta}}\\
&\le \sum_{\lambda=1}^{\infty} \bigl \lvert \hf_{\lambda b^{s}} \bigr \rvert \\
&\le \sum_{\kappa=b^{s}}^{\infty} \bigl \lvert \hf_{\kappa} \bigr \rvert = \sum_{r'=s+1}^{\infty} \sum_{\kappa=b^{r'-1}}^{b^{r'}-1} \bigl \lvert \hf_{\kappa} \bigr \rvert = \sum_{r'=s+1}^{\infty} S(r')\\
&\le \sum_{r'=s+1}^{\infty} \omega(r'-r) S(r) =   \sum_{r'=1}^{\infty} \omega(r'+s-r) S(r) =  \Omega(s-r) S(r)\\
& \le \frac{\tS(r,s)\Omega(s-r)}{1 - \homega(s-r)}.
\end{align*}
where 
\[
\Omega(\delta)=\sum_{r'=1}^{\infty} \omega(r'+ \delta), \qquad \delta \in \natzero.
\]
Assuming that $\Omega(0)$ is finite, $\lim_{\delta \to \infty} \Omega(\delta) = 0$.

This error bound suggests the following algorithm.  Choose $\delta \in \natzero$ such that $\homega(s-r)<1$ and set 
\[
\fudge = \frac{\Omega(\delta)}{1 - \homega(\delta)}.
\]
Define $r_j=r_*+j-1$ and $s_j=r_j+\delta$.  Given a tolerance $\varepsilon$, and an integrand $f$, do the following:  for $j=1, 2, \ldots$ check whether
\[
\fudge \tS(r_j,s_j) \le \varepsilon.
\]
If so, we're done.  If not, increment $j$ by one and repeat.


$\dtf$ $a^{\dtf}$ $a_{\dtf}$

\section{The Discrete Transform in Its Gory Detail}

We stick with base $2$ for simplicity.  For any $i\in \natzero$ let $\ai \in \{0,1\}^\infty$ be the vector denoting its binary digits, i.e., 
\[
i=i_0+2i_1+2^2i_2+ \cdots \quad \iff \quad \ai=(i_0,i_1,i_2, \ldots ).
\]
Let $\{\vt_0, \vt_1, \ldots\}$ be a sequence of points with the property that 
\[
t_i= i_1 \vt_1 +i_2 \vt_2 + i_3 \vt_4 + \cdots + i_\ell \vt_{2^{\ell-1}} + \cdots, \quad \text{where } \ai=(i_0,i_1,i_2, \ldots ).
\]

For any $\vk \in \bbK$, define $\nu: \bbK \to  \{0,1\}^\infty$ recursively as follows. Letting $\akappa=\nu(\vk)$, 
\[
\kappa_1 = 2\ip{\vk}{\vt_1} \pmod 2
\]  


\bibliographystyle{model1b-num-names.bst}
\bibliography{FJH22,FJHown22}
\end{document}