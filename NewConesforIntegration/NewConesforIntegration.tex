\documentclass[]{elsarticle}
%\setlength{\marginparwidth}{0.5in}
\usepackage{amsmath,amssymb,amsthm,mathtools,bbm,booktabs,array,tikz,pifont,comment,multirow,url,graphicx}
\input FJHDef.tex

%Requires ApproxUnivariate_i.tex, univariate_integration_i.tex, ConesPaperSpikyquad.eps, ConesPaperFlukyquad.eps

\DeclareMathOperator{\Var}{Var}
\DeclareMathOperator{\INT}{INT}
\DeclareMathOperator{\APP}{APP}
\DeclareMathOperator{\lin}{lin}
\DeclareMathOperator{\up}{up}
\DeclareMathOperator{\lo}{lo}
\DeclareMathOperator{\fix}{fix}
\DeclareMathOperator{\err}{err}
\DeclareMathOperator{\maxcost}{maxcost}
\DeclareMathOperator{\mincost}{mincost}
\newcommand{\herr}{\widehat{\err}}
\newcommand{\occ}{\overline{\cc}}
\newcommand{\oci}{\overline{\ci}}

\newtheorem{theorem}{Theorem}
\newtheorem{prop}[theorem]{Proposition}
\newtheorem{lem}{Lemma}
\newtheorem{cor}{Corollary}
\theoremstyle{definition}
\newtheorem{algo}{Algorithm}
\newtheorem{condit}{Condition}
%\newtheorem{assump}{Assumption}
\theoremstyle{remark}
\newtheorem{rem}{Remark}
\newcommand{\Fnorm}[1]{\abs{#1}_{\cf}}
\newcommand{\Ftnorm}[1]{\abs{#1}_{\tcf}}
\newcommand{\Gnorm}[1]{\norm[\cg]{#1}}
\newcommand{\flin}{f_{\text{\rm{lin}}}}

\journal{Journal of Complexity}

\begin{document}

\begin{frontmatter}

\title{Another Cone for Integration}
\author{Fred J. Hickernell} \ead{hickernell@iit.edu}
\address{Room E1-208, Department of Applied Mathematics, Illinois Institute of Technology,\\ 10 W.\ 32$^{\text{nd}}$ St., Chicago, IL 60616}

\begin{abstract} 

\end{abstract}

\begin{keyword}
adaptive \sep automatic \sep cones \sep function recovery \sep guarantee \sep integration \sep quadrature
%% keywords here, in the form: keyword \sep keyword

\MSC[2010] 65D05 \sep 65D30 \sep 65G20
%% MSC codes here, in the form: \MSC code \sep code
%% or \MSC[2008] code \sep code (2000 is the default)

\end{keyword}
\end{frontmatter}

\section{Introduction}

In \cite{HicEtal14b} we considered the problem of integration and the cone of integrands
\begin{equation}\label{origcone}
\cc_{\tau}:=\{f\in \cv^{1}:\Var(f')\leq\tau\norm[1]{f'-f(1)+f(0)}\},
\end{equation}
where the total variation and the $\cl_p$ norms are defined as
\begin{gather*}
\Var(f) := \sup_{\substack{n \in \naturals\\ 0 = x_0 < x_1 < \cdots < x_{n} =1}} \sum_{i=1}^n \abs{f(x_i)-f(x_{i-1})}, \\
\norm[p]{f}:= \begin{cases} \displaystyle \left[\int_0^1 \abs{f(x)}^p \, \dif x \right]^{1/p}, & 1 \le p < \infty,\\[1ex]
\displaystyle  \sup_{0 \le x \le 1} \abs{f(x)}, & p=\infty,
\end{cases}
\\
\cv^{k}: =\cv^{k}[0,1]=\{f\in C[0,1]: \Var(f^{(k)}) < \infty \}.
\end{gather*}
We derived an algorithm \cite[Algorithm 4]{HicEtal14b} that was guaranteed for integrands in $\cc_{\tau}$.  In this note we consider another algorithm and other cones.  

First we recall some notation and results from \cite{HicEtal14b}.  For all $n \in \mathcal{I}:=\{0,2,3,\ldots\}$ we have the linear spline.  By convention $A_0(f)=0$, and for $n>0$, 
\begin{subequations} \label{linearspline}
\begin{equation}
x_{i,n}:=x_{i}:=\frac{i-1}{n-1}, \qquad i=1, \ldots, n,
\end{equation}
\begin{multline}
A_{n}(f)(x):=(n-1) \left[ f(x_{i})(x_{i+1}-x) +f(x_{i+1})(x-x_{i}) \right] \\ \text{for }x_{i} \leq x \leq x_{i+1}.
\end{multline}
\end{subequations}
The cost of each function value is one and so the cost of  $A_n$ is $n$. The dependence of the nodes, $x_i$ on $n$ is often suppressed for simplicity.  Integrating the linear spline gives us the trapezoidal rule based on $n-1$ trapezoids:
\begin{equation*}
    T_{n}(f) := \int_0^1 A_n(f) \, \dif x
    =\frac{1}{2n-2}[f(x_{1})+2f(x_{2})+\cdots+2f(x_{n-1})+f(x_{n})]
\end{equation*}
for $n\ge 2$ and $T_0(f)=0$.

The error of the trapezoidal rule has the following upper bound \cite[(7.15)]{BraPet11a}:
\begin{equation} \label{errbda}
\abs{\int_0^1 f(x) \, dx - T_n(f)} \le \frac{\Var(f')}{8(n-1)^2} \qquad n \in \ci\setminus \{0\}.
\end{equation}
For any $n\in \ci$, let $\cj_n = \{m \in \naturals : (n-1)/(m-1) \in \naturals \}$.  This means that $T_n$ integrates exactly any function that is a linear spline using $m$ nodes for $m\in \cj_n$. This implies that  
\begin{align} 
\nonumber
\abs{\int_0^1 f(x) \, dx - T_n(f)} &= \abs{\int_0^1 [f(x)-A_m(f)(x)] \, dx - T_n(f- A_m(f))}\\
\label{errbdb}
& \le \frac{\Var(f'-A_m(f)')}{8(n-1)^2} \qquad \forall m \in \cj_n, \ n \in \ci\setminus \{0\}.
\end{align}

The variation of the first derivative of $f$ is bounded below by the variation of the first derivative of of the linear spline of $f$.  For all $f \in \cv^1$ it follows that
\begin{multline} 
\Var(f') \ge F_n(f) :=\Var(A_n(f)') \\
 = \begin{cases} 0, & n=0,2, \\
\displaystyle (n-1)\sum_{i=1}^{n-2} \bigabs{f(x_{i}) - 2 f(x_{i+1})+f(x_{i+2})}, 
& n\ge 3. 
\end{cases}
\label{Fnormalg}
\end{multline}
Also note that
\begin{equation} \label{Fnbounda}
F_m(f) = F_m(A_n(f)) \le \Var(A_n(f)') = F_n(f) \qquad \forall m \in \cj_n.
\end{equation}
The bound in  further implies that 
\begin{multline} \label{Fnboundb}
F_n(f) \le \Var(f')  \le \Var(f'-A_m(f)') + \Var(A_m(f)) \\
= \Var(f'-A_m(f)') + F_m(f).
\end{multline}

Another useful fact from ??? is that 
\begin{equation} \label{boundweaknorm}
\norm[1]{f'-f(1)+f(0)} \le \tF_n(f) + \tildeh(n) \Var(f') \qquad \forall f \in \cv^1 
\end{equation}


\section{New Cone, New Algorithm}
Let $\oci$ be some non-empty subset of $\{(\ell,m,n) \in \{2,3,\ldots\}^3 : \ell \in \cj_m, \ m \in \cj_n\}$, and let $\otau:\oci \to (0,\infty)$ be some given function.   The new cone considered here is defined as
\begin{multline} \label{Fcone}
\occ_{\otau}:= \{ f \in \cv^1 :  \Var(f'-A_m(f)') \le F_n(f-A_m(f)) + \otau(\ell,m,n) \norm[1]{f' - A_{\ell}(f)'}) \\ \forall (\ell,m,n) \in \oci \}.
\end{multline}
Note that for all $(\ell,m,n) \in \oci$
\begin{align*}
\norm[1]{f' - A_{\ell}(f)'}) & = \norm[1]{f'-A_m(f)'+A_m(f)' - A_\ell(f)'} \\
& \le  \norm[1]{f'-A_m(f)'} + \tF_m(f-A_\ell(f)) \\
& \le  \tF_m(f-A_m(f)) + \tildeh(m)\Var(f'-A_m(f)') + \tF_m(f-A_\ell(f)) \\
& = \tildeh(m)\Var(f'-A_m(f)') + \tF_m(f-A_\ell(f)) \qquad \text{by \eqref{boundweaknorm}}
\end{align*}

For all $f \in \occ_{\otau}$ it then follows that
\begin{align*}
\Var(f'-A_m(f)') & \le F_n(f-A_m(f)) + \otau(\ell,m,n) \tF_\ell(f)\\
& \le F_n(f-A_m(f)) + \otau(\ell,m,n) [\tildeh(m)\Var(f'-A_m(f)') \\
&\qquad \qquad + \tF_m(f-A_\ell(f))] \\
\Var(f'-A_m(f)') &\le  \frac{F_n(f-A_m(f)) + \otau(\ell,m,n) \tF_m(f-A_\ell(f))}{1-\otau(\ell,m,n) \tildeh(m)}, \\
&\qquad \qquad \text{provided }\otau(\ell,m,n) \tildeh(m) < 1.
\end{align*}

Here $\htau : \hci \to [0,\infty)$ is some specified function that defines the cone, and $\hci=\{N_{\min}, N_{\min}+1, \ldots \}$, where $N_{\min}$ is some integer no smaller than $3$.  
\begin{algo}[New Cone Adaptive Univariate Integration] \label{newconealgo}
Let the sequence of algorithms $\{T_n\}_{n\in \mathcal{I}}$ $\{F_n\}_{n\in \mathcal{I}}$, and $\hcc_{\htau}$ be as described above.  Set $i=1$, and let $n_1=N_{\min}$. For any error tolerance $\varepsilon$ and input function $f$, do the following:
\begin{description}
\item[Step 1.\ Bound {$\Var(f')$} and check for convergence.] Compute $F_{n_i}(f)$ in \eqref{Fnormalg}.  Check whether $n_i$ is large enough to satisfy the error tolerance, i.e.
    \begin{equation*}
      \htau(n_i) F_{n_i}(f) \le 8 (n_i-1)^2 \varepsilon.
    \end{equation*}
If this is true, then return $T_{n_i}(f)$ and terminate the algorithm.   

\item[Step 2.\ Increase the number of trapezoids.]  If the above condition is false, choose $n_{i+1}=2n_i$, increment $i$, and go to Step 1.
\end{description}
\end{algo}



\section{New Cone, New Algorithm}
The new cone considered here is defined as
\begin{equation} \label{Fcone}
\hcc_{\htau}:= \{ f \in \cv^1 :  \min_{m \in \cj_n} \Var(f'-A_m(f)') \le \htau(n) F_n(f) \ \forall n \in \hci \},
\end{equation}
Here $\htau : \hci \to [0,\infty)$ is some specified function that defines the cone, and $\hci=\{N_{\min}, N_{\min}+1, \ldots \}$, where $N_{\min}$ is some integer no smaller than $3$.  
\begin{algo}[New Cone Adaptive Univariate Integration] \label{newconealgo}
Let the sequence of algorithms $\{T_n\}_{n\in \mathcal{I}}$ $\{F_n\}_{n\in \mathcal{I}}$, and $\hcc_{\htau}$ be as described above.  Set $i=1$, and let $n_1=N_{\min}$. For any error tolerance $\varepsilon$ and input function $f$, do the following:
\begin{description}
\item[Step 1.\ Bound {$\Var(f')$} and check for convergence.] Compute $F_{n_i}(f)$ in \eqref{Fnormalg}.  Check whether $n_i$ is large enough to satisfy the error tolerance, i.e.
    \begin{equation*}
      \htau(n_i) F_{n_i}(f) \le 8 (n_i-1)^2 \varepsilon.
    \end{equation*}
If this is true, then return $T_{n_i}(f)$ and terminate the algorithm.   

\item[Step 2.\ Increase the number of trapezoids.]  If the above condition is false, choose $n_{i+1}=2n_i$, increment $i$, and go to Step 1.
\end{description}
\end{algo}


\section{The New Cone's Relationship to Other Cones}

The cone defined in \eqref{Fcone} makes Algorithm \ref{newconealgo} work.  In this section we show that it contains and is contained in other cones that might be more intuitive.  One family of cones of interest is defined by replacing $F_n(f)$ by $\Var(f')$ in \eqref{Fcone}:
\begin{equation}\label{varcone}
\tcc_{\ttau}:=\{f\in \cv^{1}:\Var(f'-A_n(f)')\leq\ttau(n)\Var(f'), \ n \in  \ci\},
\end{equation}
where $\ttau:\ci \to [0,2]$ is non-increasing.  Another family of cones is related to \eqref{origcone} and is defined as
\begin{equation}\label{origconefamily}
\cc_{\otau}:=\{f\in \cv^{1}:\Var(f'-A_n(f)')\leq\otau(n)\norm[1]{f'-A_n(f)'}, \ n \in  \ci\},
\end{equation}
where $\otau:\ci \to [0,\infty]$.  Under this definition $\cc_{\tau}$ corresponds to defining $\otau(2)=\tau$, $\otau(n)=\infty$ for $n>2$.

To facilitate the comparison of  $\hcc_{\htau}$,  $\tcc_{\ttau}$, and $\cc_{\otau}$ we note several inequalities. For all $f \in \cv^1$,
\begin{gather}
\label{varfprimeineq}
\Var(f') \le \Var(f'-A_n(f)') + \Var(A_n(f)') = \Var(f'-A_n(f)') + F_n(f), \\
\label{varfprimeminineq}
\Var(f'-A_n(f)') \le \Var(f')+ \Var(A_n(f')) = \Var(f')+ F_n(f).
\end{gather} prove the following lemma.
From \eqref{Fnormalg} and \eqref{varfprimeminineq} it follows that 
\begin{equation} \label{twoineq}
\Var(f'-A_n(f)') \le 2\Var(f') \qquad \forall f \in \cv^1,
\end{equation}
which is why $\ttau(n) \le 2$ for all $n$. Moreover, if $\ttau(n) = 2$ for all $n \in \ci$, then $\tcc_{\ttau}=\cv^1$.

\begin{theorem} \label{twoinclusiontildethm} Given the function $\htau:\hci \to [0,\infty)$, suppose that $\ttau_j : \ci \to [0,2]$, $j=1,2$ satisfy the inequality
\begin{equation} \label{inequaltilde}
\ttau_1(n) \le \frac{\htau(n)}{1+\htau(n)} \le \min(2,\htau(n)) \le \ttau_2(n) \qquad \forall  n \in \hci.
\end{equation}
It follows that $\tcc_{\ttau_1} \le \hcc_{\htau} \subseteq \tcc_{\ttau_2}$.
\end{theorem}

\begin{proof} First suppose that $f \in \tcc_{\ttau_1}$ where $\ttau_1$ satisfies inequality \eqref{inequaltilde}.  It then follows that
\begin{align*}
\Var(f'-A_n(f)') & = (1+\htau(n)) \Var(f'-A_n(f)') - \htau(n) \Var(f'-A_n(f)')\\
& \le [1+\htau(n)] \ttau_1(n) \Var(f') - \htau(n) \Var(f'-A_n(f)') \quad \text{by \eqref{varcone}}\\
& \le \htau(n)[ \Var(f') - \Var(f'-A_n(f)')] \qquad \text{by \eqref{inequaltilde}}\\
& \le \htau(n)F_n(f) \qquad \text{by \eqref{varfprimeineq}}.
\end{align*}
Thus, $\tcc_{\ttau_1} \le \hcc_{\htau}$.  Now suppose that $f \in \hcc_{\htau}$.  It follows by \eqref{Fnormalg} and \eqref{twoineq} that $f \in \tcc_{\ttau_2}$.
\end{proof}

To prove the relationship between the cones defined in \eqref{Fcone} and \eqref{origconefamily} the following bound is needed.   

\begin{lem} \label{FnFtnlem}  For all $n \in \ci$ and all $ f \in \cv^1$ it follows that 
\begin{equation}
F_n(f) \le 2(n-1)\norm[1]{f'-A_n(f)'}. \label{Fnupbd}
\end{equation}
\end{lem}
\begin{proof}
For all $f \in \cv^1$ we use the triangle inequality:
\begin{align*} 
\nonumber
F_{n}(f) &= (n-1)\sum_{i=1}^{n-2} \bigabs{f(x_{i}) - 2 f(x_{i+1})+f(x_{i+2})}\\
\nonumber
&\le (n-1)\sum_{i=1}^{n-2} \biggabs{f(x_{i}) - f(x_{i+1})+\frac{f(1)-f(0)}{n-1}}\\
\nonumber
&\qquad \qquad + (n-1)\sum_{i=1}^{n-2} \biggabs{- f(x_{i+1})+f(x_{i+2}) -\frac{f(1)-f(0)}{n-1}}\\
\nonumber
&\le 2 (n-1) \sum_{i=1}^{n-1}\biggabs{f(x_{i+1})-f(x_{i}) - \frac{f(1)-f(0)}{n-1}}\\
\nonumber
&=2(n-1) \bignorm[1]{A_n(f)'-f(1)+f(0)} \\
&\le 2(n-1) \bignorm[1]{f'-f(1)+f(0)} . 
\end{align*}
\end{proof}

\begin{theorem}  Given the function $\htau: \hci \to [0,\infty)$, suppose that $\otau_j : \ci \to [0,\infty)$, $j=1,2$ satisfy the inequality
\begin{equation} \label{inequalbar}
\min\{ 2(n-1) \htau(n) : n \in \hci  \} ?? \otau_?(n) 
\end{equation}
It follows that $\cc_{\otau_1} \le \hcc_{\htau} \subseteq \cc_{\otau_2}$.  
\end{theorem}
\begin{proof}  For all $f \in \hcc_{\htau}$ it follows from \eqref{Fnupbd} that
\begin{equation*}
\Var(f') \le \htau(n) F_n(f) = 2(n-1) \htau(n) \bignorm[1]{f'-f(1)+f(0)} \qquad \forall n \ge N_{\min}. 
\end{equation*}
Applying the definition of $\tau$ completes the proof.
\end{proof}

Now we define a cone that is contained in $\hcc_{\htau}$.  Let 
\begin{equation}\label{smallconeinteg}
\tcc_{\ttau}:=\{f\in \cv^{1}:\Var(f')\leq\ttau(n)\norm[1]{f'-f(1)+f(0)} \ \forall n\ge 3  \},
\end{equation}



\begin{theorem}  For any non-increasing $\htau: \ci \to (1,\infty)$, let  
\[
\tau = \min\{ 2(n-1) \htau(n) : n \ge  N_{\min}  \}.
\]
It follows that $\hcc_{\htau} \subseteq \cc_{\tau}$.  
\end{theorem}
\begin{proof}  For all $f \in \hcc_{\htau}$ it follows from \eqref{Fnupbd} that
\begin{equation*}
\Var(f') \le \htau(n) F_n(f) = 2(n-1) \htau(n) \bignorm[1]{f'-f(1)+f(0)} \qquad \forall n \ge N_{\min}. 
\end{equation*}
Applying the definition of $\tau$ completes the proof.
\end{proof}


\section*{References}
\bibliographystyle{model1b-num-names.bst}
%\bibliographystyle{elsarticle-num-names}
\bibliography{FJH22,FJHown22}
\end{document}
