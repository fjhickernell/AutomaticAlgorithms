\documentclass[final]{elsarticle}
\setlength{\marginparwidth}{0.5in}
\usepackage{amsmath,amssymb,amsthm,natbib,mathtools,graphicx}
\input FJHDef.tex

\newcommand{\sphere}{\mathbb{S}}
%\newcommand{\cc}{\mathcal{C}}
\newcommand{\cq}{\mathcal{Q}}
\newcommand{\bbW}{\mathbb{W}}
%\newcommand{\tP}{\widetilde{P}}
\newcommand{\bg}{{\bf g}}
\newcommand{\bu}{{\bf u}}
\newcommand{\bbu}{\bar{\bf u}}
\newcommand{\bv}{{\bf v}}
\newcommand{\bbv}{\bar{\bf v}}
\newcommand{\bw}{{\bf w}}
\newcommand{\bbw}{\bar{\bf w}}
\newcommand{\hv}{\hat{v}}
\DeclareMathOperator{\MSE}{MSE}
\DeclareMathOperator{\RMSE}{RMSE}
\DeclareMathOperator{\rnd}{rnd}
\DeclareMathOperator{\abso}{abs}
\DeclareMathOperator{\rel}{rel}
\DeclareMathOperator{\nor}{nor}
\DeclareMathOperator{\err}{err}
\DeclareMathOperator{\lin}{lin}
%\DeclareMathOperator{\std}{std}
\DeclareMathOperator{\prob}{prob}
\DeclareMathOperator{\trunc}{trc}
\DeclareMathOperator{\third}{third}
%\DeclareMathOperator{\fourth}{fourth}
\newtheorem{theorem}{Theorem}
\newtheorem{prop}[theorem]{Proposition}
\newtheorem{lem}[theorem]{Lemma}
\theoremstyle{definition}
\newtheorem{algo}{Algorithm}
\newtheorem{condit}{Condition}
%\newtheorem{assump}{Assumption}
\theoremstyle{remark}
\newtheorem{rem}{Remark}
\DeclareMathOperator{\sMC}{sMC}
\DeclareMathOperator{\aMC}{aMC}
\DeclareMathOperator{\size}{size}
\DeclareMathOperator{\guile}{guile}
\DeclareMathOperator{\scale}{scale}
\DeclareMathOperator{\fix}{non}
\DeclareMathOperator{\up}{up}
\DeclareMathOperator{\lo}{lo}

\journal{Journal of Complexity}

\begin{document}

\begin{frontmatter}

\title{The Complexity of Deterministic Automatic Algorithms}
\author{Yuhan Ding}
\author{Nicholas Clancy}
\author{Caleb Hamilton}
\author{Fred J. Hickernell}
\author{Yizhi Zhang}
\address{Room E1-208, Department of Applied Mathematics, Illinois Institute of Technology,\\ 10 W.\ 32$^{\text{nd}}$ St., Chicago, IL 60616}
\begin{abstract}
\end{abstract}

\begin{keyword}
%% keywords here, in the form: keyword \sep keyword

%% MSC codes here, in the form: \MSC code \sep code
%% or \MSC[2008] code \sep code (2000 is the default)

\end{keyword}
\end{frontmatter}

\section{Introduction}
Function recovery and integration are two fundamental examples of numerical problems that arise often in practice.  It is desirable to have \emph{automatic} algorithms for solving these problems, i.e., the algorithm should decide adaptively how many and which pieces of function data are needed and then use those data to construct an approximate solution.  The error of this approximation should be guaranteed not to exceed the prescribed tolerance, $\varepsilon$, and the number of data required by the algorithm should be reasonable, given the assumptions made about the problem and the error tolerance required.

Most existing theory starts with a Banach space, $\cg$, of input functions defined on some set $\cx$, and having a semi-norm, $\norm[\cg]{\cdot}$.  The definition of $(\cg,\norm[\cg]{\cdot})$ contains assumptions about smoothness, periodicity or other qualities of the input functions.  The mathematical problem of interest is defined by a solution operator $S:\cg \to \ch$, where $\ch$ is some other Banach space with its norm $\norm[\ch]{\cdot}$.  For integration, $\ch=\reals$, and for function approximation, $\ch$ is some superset of $\cg$, for example, $\cl_{\infty}(\cx)$. One then shows that there exists an algorithm, $A$, that provides an approximate solution differing from the true solution by no  more than $\varepsilon$.  Furthermore, the number of function data needed, $\cost(A)$, is bounded in terms of $\varepsilon$, typically as follows:
\begin{equation} \label{traditionerr}
\sup_{\substack{f \in \cg\\ \norm[\cg]{f} \le \sigma}} \norm[\ch]{S(f)-A(f)} \le \varepsilon, \qquad \cost(A)\le C_{\up}\left(\frac{\sigma}{\varepsilon}\right)^{p}.
\end{equation}
Here it is necessarily assumed that the algorithm is exact if the semi-norm of the input function vanishes, i.e., $S(f)=A(f)$ if $\norm[\cg]{f}=0$. Algorithm $A$ is said to be optimal if any algorithm satisfying the error criterion above must have a cost of at least $\Order((\sigma/\varepsilon)^{p})$.

A practical drawback of the analysis summarized above is that the definition of the algorithm, at least as far as its cost is concerned, depends a priori on an upper bound on $\norm[\cg]{f}$. Specifically, the theory is derived for functions in a \emph{ball}, $\cb_\sigma=\{ f \in \cg : \norm[\cg]{f} \le \sigma\}$, and the algorithm needs to know the radius, $\sigma$. Automatic algorithms try to estimate $\sigma$ so that the number of function data needed can be determined adaptively.  This is the approach taken here.  Unfortunately, there is a lack of theory to rigorously justify the estimate of $\sigma$ and the resulting adaptive algorithm.  A general framework for doing so is developed here and some illustrative examples are provided. The key is to look at functions in a \emph{cone} instead of a ball.

A simple example of the kind of practical problem addressed here is computing $\int_0^1 f(x) \, \dif x$ for all integrands whose second derivatives are absolutely integrable, i.e., $\norm[1]{f''}$.  The trapezoidal rule computes an approximation with error no greater than $\varepsilon$ using $\Order(\sqrt{\norm[1]{f''}/\varepsilon})$ function values.  However, except for relatively simple integrands this norm of the second derivative, $\norm[1]{f''}$, is not available.  After evaluating $f$, at a number of points in $[0,1]$, one may estimate the norm of the first derivative, $\norm[1]{f'}$, by the corresponding norm of the piecewise linear spline for $f$, and the error can be bounded rigorously in terms of $\norm[1]{f''}$.  If one assumes that $\norm[1]{f''}$ is not arbitrarily larger than $\norm[1]{f'}$ (a cone condition), then the reliable numerical estimate of $\norm[1]{f'}$ can be used as a surrogate for $\norm[1]{f''}$, and one can then reliable estimate the integral to the desired precision.  This example is discussed in detail in Section \ref{integsec}.

\section{General Problem Definition}

\subsection{Problems and Algorithms} The function approximation, integration, or other problem to be solved is defined by a \emph{solution operator} $S:\cg \to \ch$, where $\cg$ is a Banach space of possible input functions defined on $\cx$ with semi-norm $\norm[\cg]{\cdot}$, and $\ch$ is some other Banach of possible outputs or solutions with norm $\norm[\ch]{\cdot}$. It is assumed that the $\cg_0 = \{f \in \cg :  \norm[\cg]{f}=0\}$, the subspace of $\cg$ containing functions with vanishing $\cg$-semi-norm, has finite dimension. The solution operator is assumed to have a scale property, i.e., 
\[
S(cf) = cS(f) \qquad \forall c\ge 0.
\]
Examples include the following:
\begin{align*}
\text{Integration:} \quad & S(f) = \int_{\cx} f(\vx) \, \rho(\vx) \, \dif \vx, \quad \rho \text{ is fixed,}\\
\text{Function Recovery:} \quad & S(f) = f, \\
\text{Poisson's Equation:} \quad & S(f) = u, \quad \text{where } \begin{array}{c} -\Delta u(\vx) = f(\vx), \ \vx \in \cx, \\ u(\vx)=0 \ \forall \vx \in \partial \cx, \text{ and}\end{array} \\
\text{Optimization:} \quad & S(f) = \min_{\vx \in \cx} f(\vx).
\end{align*}
The first three examples above are linear problems, which automatically satisfy the scale property for $S$, but so does the last example, which is a nonlinear problem.

The goal is to find an algorithm $A:\cg \to \ch$ for which $S(f) \approx A(f)$. Following the definition of algorithms described in \cite[Section 3.2]{TraWasWoz88}, the algorithm takes the form of some function of data derived from the input function:
\begin{equation}
\label{algoform}
A(f) =  \phi(\vL(f)), \quad \vL(f) = \left(L_1(f), \ldots, L_m(f)\right) \qquad \forall f \in \cg.
\end{equation}
Here the $L_i \in \Lambda$ are real-valued functions defined on $\cg$ with the following scale property:
\begin{equation}
\label{dataassump}
L(cf) = cL(f) \qquad \forall f \in \cg, \ c \in \reals, \ L \in \Lambda.
\end{equation}
One popular choice for $\Lambda$ is the set of all function values, $\Lambda^{\std}$, i.e., $L_i(f) = f(\vx_i)$ for some $\vx_i \in \cx$.  Another common choice is the set of all bounded linear functionals, $\Lambda^{\lin}$.  In general, $m$ may depend on $f$ and the choice of $L_i$ may depend on $L_1(f), \ldots, L_{i-1}(f)$.  In this article, all algorithms are assumed to be deterministic.  There is no random element.

\subsection{Non-Adaptive Algorithms}

The set $\ca_{\fix}(\cg,\ch,S,\Lambda)$ contains algorithms as just described for which the choice of the $L_i$ and the number of function data used by the algorithm, $m$, are both assumed to be independent of the input function, i.e., these algorithms are non-adaptive.  Furthermore, any $A \in \ca_{\fix}(\cg,\ch,S,\Lambda)$ is assumed to satisfy the following scale properties
\begin{equation}
\label{algoscale}
\vL(cf) = c \vL(f), \quad 
\phi(c\vy) = c\phi(\vy), \quad A(cf) = cA(f) \qquad \forall c \ge 0, \ \vy \in \reals^m.
\end{equation}

The cost of a non-adaptive algorithm, $A \in  \ca_{\fix}(\cg,\ch,S,\Lambda)$, is fixed and is defined as the sum of the costs of all the function data:
\begin{equation} \label{costfix}
\cost(A) = \$(\vL) = \$(L_1) + \cdots +\$(L_m),
\end{equation}
where $\$:\Lambda \to (0,\infty)$, and $\$(L)$ is the cost of acquiring the datum $L(f)$. The cost of $L$ may be the same for all $L \in \Lambda$.  Alternatively, it might be a vary with the choice of $L$.  E.g., if $f$ is a function of the infinite sequence real numbers, $(x_1, x_2, \ldots)$, the cost of evaluating the function with arbitrary values of the first $d$ coordinates, $L(f)=f(x_1, \ldots, x_d, 0, \ldots)$, might be $d$.  This cost model has been used by \cite{HicMGRitNiu09a,KuoEtal10a,NiuHic09a,NiuHic09b,PlaWas11a} for integration problems and \cite{Was13a,WasWoz11a,WasWoz11b} for function approximation problems.

The error of a non-adaptive algorithm $A  \in \ca_{\fix}(\cg,\ch,S,\Lambda)$ is defined  as
\begin{equation} \label{errdefworst}
\err(A,\cg,\ch,S)
= \min\{ \delta \ge 0 : \norm[\ch]{S(f) -  A(f)} \le \delta \norm[\cg]{f} \ \forall f \in \cg \},
\end{equation}
When the problem has real-valued solutions, i.e., $\ch=\reals$, one may also define a one sided error criterion:
\begin{equation}\label{errpmdefworst}
\err_{\pm}(A,\cg,\reals,S) = 
\min \{ \delta \ge 0 : \pm[S(f) -  A(f)] \le \delta \norm[\cg]{f} \ \forall f \in \cg \} 
\end{equation}
Since $\norm[\cg]{\cdot}$ may be a semi-norm, but not a norm, a finite error in any of the above definitions assumes that the algorithm is exact, i.e., $S(f)=A(f)$, for all $f$ with $\norm[\cg]{f}=0$.

The above error criteria are normalized, meaning that the absolute error, $\norm[\ch]{S(f) -  A(f)}$ is measured with respect to the $\cg$-semi-norm of the input function. The complexity of a problem for this set of algorithms, $\ca_{\fix}(\cg,\ch,S,\Lambda)$, is defined as the cost of the cheapest algorithm that satisfies the specified error tolerance, $\varepsilon$:
\begin{multline} \label{fixcostcomplex}
\comp(\varepsilon,\ca_{\fix}(\cg,\ch,S,\Lambda)) \\
= \inf\left\{\cost(A) : \err(A,\cg,\ch,S) \le \varepsilon, \ A \in \ca_{\fix}(\cg,\ch,S,\Lambda) \right \}.
\end{multline}
Here the infimum of an empty set is defined to be $\infty$.  This means that to guarantee that $\norm[\ch]{S(f) -  A(f)} \le \varepsilon$, one needs an algorithm with a cost of at least 
\[
\comp(\varepsilon/\norm[\cg]{f},\ca_{\fix}(\cg,\ch,S,\Lambda)).
\]
This cost does not decrease as either $\varepsilon$ decreases or $\norm[\cg]{f}$ increases.

Suppose that there is a sequence of nonadaptive algorithms indexed by their cost, and which converge to the true answer:
\begin{subequations} \label{algseqdef}
\begin{gather} 
\{A_n\}_{n \in \ci}, \qquad A_n  \in \ca_{\fix}(\cg,\ch,S,\Lambda), \\
\lim_{\substack{n \to \infty\\ n \in \ci}} \err(A_n,\cg,\ch,S) = 0, \qquad \cost(A_n) = n,  
\end{gather}
\end{subequations}
where the countable, non-negative-valued index set, 
\begin{equation} \label{indexdef}
\ci=\{n_1, n_2, \ldots\} \quad \text{with } n_i < n_{i+1}, \quad \text{satisfies } \sup_i \frac{n_{i+1}}{n_i} <\infty. 
\end{equation} 
This sequence of algorithms is called \emph{nearly optimal} for the problem $(\cg,\ch,S,\Lambda)$ if it essentially tracks the minimum cost algorithms, namely,
\begin{equation} \label{nearoptdef}
\sup_{0 < \varepsilon \le 1} \frac{\min\{n \in \ci : \err(A_n,\cg,\ch,S) \le \varepsilon\} \varepsilon^p}{\comp(\varepsilon,\ca_{\fix}(\cg,\ch,S,\Lambda))} <\infty, \qquad \forall p>0.
\end{equation}
The sequence is called optimal if above inequality holds fo $p=0$.  A nearly optimal sequence of algorithms may differ in its cost from an optimal algorithm by powers of $\log(\varepsilon^{-1})$, for example.


\subsection{Automatic, Adaptive Algorithms}

Non-adaptive algorithms, $A \in \ca_{\fix}(\cg,\ch,S,\Lambda)$ need an upper bound on $\norm[\cg]{f}$ to guarantee that they meet the prescribed error tolerance for the input function $f$.  Automatic algorithms attempt to estimate $\norm[\cg]{f}$ and then determine the number of function data needed to meet the error tolerance.  Such automatic, adaptive algorithms are now defined, somewhat differently from the non-adaptive algorithms above.  However, in practice automatic algorithms use non-adaptive algorithms as building blocks.

Practical automatic algorithms in $\ca(\cg,\ch,S,\Lambda)$ take the form of ordered pairs of functions
\[
(A,W): \cg \times (0,\infty)\times (0,\infty] \to \ch \times \{\text{false},\text{true}\},
\]
for which $S(f) \approx A(f;\varepsilon,N_{\max})$.  Here $\varepsilon \in (0,\infty)$ is a user-supplied error tolerance, $N_{\max} \in (0,\infty]$ is a user-supplied maximum cost budget, and $W(f;\varepsilon,N_{\max})$ is a Boolean warning flag that is false if the algorithm completed its calculations without attempting to exceed the cost budget, and is true otherwise.  

As in \eqref{algoform}, the algorithm takes the form of some function of function data: $A(f;\varepsilon,N_{\max}) = \phi\left(\vL(f);\varepsilon,N_{\max}\right)$.
Now, however, the algorithm is allowed to be adaptive. The choice of $L_2$ may depend on the value of $L_1(f)$, the choice of $L_3$ may depend on $L_1(f)$ and $L_2(f)$, etc.  The number of function data used by the algorithm, $m$, may also be determined adaptively. The choice of how many and which function data to use depends on $\varepsilon$ and $N_{\max}$.  Thus, $\vL(c\vy)$ might not equal $c\vL(\vy)$ since the length of the information vector depends on the data recorded.  The goal of the algorithm is to make $\norm[\ch]{S(f) - A(f;\varepsilon,N_{\max})} \le \varepsilon$, but this is not a requirement of the definition.

The cost of the algorithm for a specified input function is defined analogously to \eqref{costfix} as the sum of the costs of all function data.
\[
\cost(A,f;\varepsilon,N_{\max}) = \$(\vL) = \$(L_1) + \cdots +\$(L_m).
\]
Because of the potentially adaptive nature of the algorithm, namely, that $m$ may depend on $f$, it follows that the cost may depend on $f$ as well as $A$. The input parameter $N_{\max}$ tells the algorithm to ensure that $\cost(A,f;\varepsilon,N_{\max}) \le N_{\max}$ for all $f$ and $\varepsilon$.  This is a practical consideration since the user does not want to wait indefinitely for an answer.  

The cost of the algorithm is expected to scale with the $\cg$-semi-norm of the integrand.  This means that the cost of an algorithm generally increases as $\norm[\cg]{f}$ increases.  The cost of the algorithm may also depend on $\cn$, the subset of $\cg$ where the functions of interest lie.  To represent this idea one defines
\begin{equation*}
\cost(A,\cn,\varepsilon,N_{\max},\sigma)
= \sup \{ \cost(A,f;\varepsilon,N_{\max}) : f \in \cn, \ \norm[\cg]{f} \le \sigma \} .
\end{equation*}
Here the set $\cn$ is allowed to depend on the algorithm inputs, $\varepsilon$ and $N_{\max}$, but not on the unknown $\sigma$.

For automatic algorithms, returning an approximation with the desired error is not enough.  One also wants the algorithm to be confident that the answer is correct.  A successful algorithm for $\cn \subseteq \cg$, denoted $(A,W) \in \ca(\cn,\ch,S,\Lambda)$, is one that meets the prescribed error tolerance and does not raise the warning flag.  Specifically, success is defined as
\begin{multline*}
\success(A,W,\cn,\varepsilon,N_{\max}) \\
= \begin{cases} \text{true} & \text{if } \displaystyle \norm[\ch]{S(f)-A(f;\varepsilon,N_{\max})} \le \varepsilon \ \& \ W(f;\varepsilon,N_{\max})=\text{false} \quad \forall  f \in \cn, \\
\displaystyle \text{false} & \text{otherwise}.
\end{cases}
\end{multline*}
The above are absolute error criteria for success.  One might also define relative error criteria instead, but finding successful algorithms for relative error is a non-trivial exercise and will be considered in future work.

The complexity of a problem is defined as the cost of the cheapest successful algorithm with $\cg$-semi-norm no greater than $\sigma$:
\begin{multline} \label{complexdef}
\comp(\varepsilon,\ca(\cn,\ch,S),\Lambda,N_{\max},\sigma) \\
 = \inf\left\{\cost(A,\cn,\varepsilon,N_{\max},\sigma) : \success(A,W,\cn,\varepsilon,N_{\max}) = \text{true}, \right .\\
\left.  (A,W) \in \ca(\cn,\ch,S,\Lambda) \right \}.
\end{multline}
Here the infimum of an empty set is defined to be $\infty$.

The set of non-adaptive algorithms, $\ca_{\fix}(\cg,\ch,S,\Lambda)$, defined in the previous subsection is a subset of the automatic algorithms $\ca(\cg,\ch,S,\Lambda)$.  Algorithms in $\ca_{\fix}(\cg,\ch,S,\Lambda)$ are not affected by the error tolerance $\varepsilon$ and do not recognize a cost budget $N_{\max}$.  Moreover, the warning flag for an algorithm in $\ca_{\fix}(\cg,\ch,S,\Lambda)$ is always returned as false.  Whereas the non-adaptive algorithms are inherently impractical by themselves, they are vital components of automatic, adaptive algorithms.

\subsection{Cones of Functions} \label{conesubsec} All algorithms can be fooled by some input functions, even if these functions are sufficiently smooth.  Error analysis such as that outlined in \eqref{traditionerr} rules out fooling functions with large error by restricting the size of  $\norm[\cg]{f}$.  

As mentioned above, it is often difficult to know how large $\norm[\cg]{f}$ is a priori and so practical automatic algorithms try to estimate it.  The framework described here rules out fooling functions whose $\cg$-semi-norms cannot be estimated reliably.  This is done by considering $\cf$, a subspace of $\cg$, with its own semi-norm $\norm[\cf]{\cdot}$.   The semi-norm $\norm[\cf]{\cdot}$ is considered to be stronger than $\norm[\cg]{\cdot}$ in the following sense:
\begin{subequations} \label{Fspacecond}
\begin{equation} \label{Fspacecondstrong}
\min_{f_0 \in \cf_0} \norm[\cg]{f - f_0} \le C_{\cf} \norm[\cf]{f} \qquad \forall f \in \cf,
\end{equation}
where $\cf_0=\{f \in \cf : \norm[\cf]{f}=0\}$ is a finite dimensional subspace of $\cf$.  Moreover, it is assumed that any $f \in \cg$ with zero $\cg$-semi-norm must also lie in $\cf$ and have zero $\cf$-semi-norm:
\begin{equation} \label{Fspacecondzero}
\cg_0 \subseteq \cf_0.
\end{equation}
\end{subequations}

Given $\tau>0$, let $\cc_{\tau} \subset \cf$ denote a \emph{cone} of functions whose $\cf$-semi-norms are no greater than $\tau$ times their $\cg$-semi-norms:
\begin{equation} \label{conedef}
\cc_{\tau}=\{f \in \cf : \norm[\cf]{f} \le \tau \norm[\cg]{f} \}.
\end{equation}
For any $f \in \cc_{\tau}$ both $\norm[\cf]{f}$ and $\norm[\cg]{f}$ must be finite, but they can be arbitrarily large.  There is no need to assume an upper bound on their sizes, but it is possible to obtain reliable upper bounds for both $\norm[\cf]{f}$ and $\norm[\cg]{f}$ by sampling $f$.  An upper bound on $\norm[\cg]{f}$ for $f \in \cc_{\tau}$ can be computed in terms of the data $\vL(f)=(L_1(f), \ldots, L_m(f))$ because $\norm[\cf]{\cdot}$ is a stronger semi-norm than $\norm[\cg]{\cdot}$ in the sense of \eqref{Fspacecondstrong}, and because $\norm[\cf]{f}$ is no larger than a multiple of $\norm[\cg]{f}$ (see Lemma \ref{Gnormlem} below). This upper bound on $\norm[\cg]{f}$ then automatically implies an upper bound on $\norm[\cf]{f}$ from the definition of the cone. These reliable bounds on both $\norm[\cf]{f}$ and $\norm[\cg]{f}$ may be used to obtain a bound on the error of the algorithm for estimating $S(f)$  (see Theorem \ref{TwoStageDetermThm} below).

\subsection{Results that One Wishes to Prove}  The previous subsections define the problem to be approximated and the notation describing the difficulty of the problem and the efficiency of the algorithms.  This subsection summarizes the results that are proved in general in the next section and illustrated for specific cases in the following sections.

\begin{enumerate}

\renewcommand{\labelenumi}{\roman{enumi}.}

\item \emph{Upper bound on the complexity.}
One wishes to bound the complexity of solving the problem successfully, $\comp(\cn,\varepsilon,N_{\max},\sigma,\Lambda)$, in terms of some power of $\varepsilon/\sigma$ for $\cn$ suitably defined as a subset of the cone $\cc_{\tau}$.  This is done in Theorem \ref{TwoStageDetermThm}.

\item \emph{An algorithm that achieves the upper bound.}  Upper bounds on the complexity are sometimes found in a non-constructive way.  However, it is desirable to identify an explicit successful algorithm, $(A,W) \in \ca(\cn,\ch,S,\Lambda)$, that achieves these upper bounds.  This is also done in Theorem \ref{TwoStageDetermThm}.

\item \emph{Penalty for not knowing the $\cf$- and $\cg$-semi-norms of $f$.} The optimal successful algorithm must find an upper bound on $\norm[\cf]{f}$ or $\norm[\cg]{f}$ rather than assuming such an upper bound.  One hopes that the extra cost relative to the situation of knowing a priori bounds on these semi-norms is not too great.  A positive result is shown in Theorem \ref{TwoStageDetermThm}.

\item \emph{Lower bound on the complexity.}  The difficulty of the problem is provided by lower bounds on the complexity.  These are given in Theorem \ref{complowbd}.

\end{enumerate}

\section{General Theorems}

This section provides rather general theorems about the complexity of automatic algorithms.  In some sense, these theorems are roadmap or an outline because their assumptions are non-trivial and take effort to be be verified for specific problems of interest.  On the other hand, the assumptions are reasonable as is demonstrated in the later sections where concrete cases are are discussed.  

\subsection{Bounding the $\cg$-Semi-Norm}

As mentioned in Section \ref{conesubsec} automatic algorithms require a reliable upper bound on $\norm[\cg]{f}$ for all $f$ in the cone $\cc_{\tau}$. This can be obtained using a non-adaptive algorithm $G \in \ca_{\fix}(\cf,\reals_+,\norm[\cg]{\cdot},\Lambda)$, provided that one has error bounds, $\err_{\pm}(G,\cf,\reals_+,\norm[\cg]{\cdot})$, as defined in $\eqref{errpmdefworst}$.  These upper and lower error bounds imply 
\begin{multline*}
-\err_{-}(G,\cf,\reals_{+},\norm[\cg]{\cdot}) \norm[\cf]{f} \le \norm[\cg]{f}-G(f) \le \err_{+}(G,\cf,\reals_{+},\norm[\cg]{\cdot}) \norm[\cf]{f} \\ \forall f \in \cf.
\end{multline*}
Noting that $\norm[\cf]{f} \le \tau \norm[\cg]{f}$ for all $f$ in the cone $\cc_{\tau}$ implies the lemma below. 

\begin{lem} \label{Gnormlem} Any nonadaptive algorithm $G \in \ca_{\fix}(\cf,\reals_+,\norm[\cg]{\cdot},\Lambda)$ yields an approximation to the $\cg$-semi-norm of functions in the cone $\cc_{\tau}$ with the following upper and lower error bounds:
\begin{equation*}
\frac{G(f)}{1 + \tau \err_{-}(G,\cf,\reals_+,\norm[\cg]{\cdot})} \le \norm[\cg]{f} \le \frac{G(f)}{1 - \tau \err_{+}(G,\cf,\reals_+,\norm[\cg]{\cdot})} \qquad \forall f \in \cc_{\tau}.
\end{equation*}
The upper bound assumes that  
\begin{equation} \label{Gtaucond}
\err_{+}(G,\cf,\reals,\norm[\cg]{\cdot}) < 1/\tau
\end{equation}
\end{lem}

\subsection{Two-Stage, Automatic Algorithms}

Computing an approximate solution to the problem $S: \cc_{\tau} \to \ch$, e.g., integration or function approximation, depends on non-adaptive algorithms. Suppose that there is a sequence of such of algorithms, $\{A_n\}_{n \in \ci}$, with $A_n  \in \ca_{\fix}(\cg,\ch,S,\Lambda)$, indexed by their cost as defined in \eqref{algseqdef}, and for which upper error bounds are known for both the spaces $\cg$ and $\cf$:
\begin{equation}\label{algseqerrbd}
%\err(A_n,\cg,\ch,S) \le C_{1} n^{-p_1}, \qquad \err(A_n,\cf,\ch,S) \le C_{2} n^{-p_2}, 
\err(A_n,\cg,\ch,S) \le h(n), \qquad \err(A_n,\cf,\ch,S) \le \tildeh(n), 
\end{equation}
for some \emph{non-increasing} functions $h$ and $\tildeh$.  The definitions of these errors in \eqref{errdefworst} then implies upper bounds on the error of $A_n(f)$ in terms of the $\cg$-semi-norm of $f$:
\begin{align} \nonumber
\norm[\ch]{S(f) -  A_n(f)} &\le \min(\err(A_n,\cg,\ch,S)\norm[\cg]{f},\err(A_n,\cf,\ch,S)\norm[\cf]{f}) \\
\label{Anerrbound}
&\le \min(h(n),\tau \tildeh(n))\norm[\cg]{f} \qquad \forall f \in \cc_{\tau}.
\end{align}

\begin{algo} \label{twostagedetalgo} {\bf (Automatic, Adaptive, Two-Stage).} Let $\cf$, $\cg$, and $\ch$ be Banach spaces as described above, let $S$ be the solution operator, let $\varepsilon$ be a positive error tolerance, and let $N_{\max}$ be the maximum cost allowed.  Let $\tau$ be a fixed positive number, and let $G \in \ca_{\fix}(\cf,\reals_+,\norm[\cg]{\cdot},\Lambda)$ be an algorithm as described in Lemma \ref{Gnormlem} and satisfying  \eqref{Gtaucond}.
Moreover, let  $\{A_n\}_{n \in \ci}$, $A_n  \in \ca_{\fix}(\cg,\ch,S,\Lambda)$, be a sequence of algorithms as described in  \eqref{algseqdef} and \eqref{algseqerrbd}.  Given an input function $f$, do the following:

\begin{description} 

\item[Stage 1.\ Estimate {$\norm[\cg]{f}$}.] First compute $G(f)$ at a cost of $n_{G} = \cost(G)$.   Define the inflation factor 
\begin{equation}\label{norminflate}
\fC =\frac{1}{1 - \tau \err_{+}(G,\cf,\reals_+,\norm[\cg]{\cdot})} \ge 1.
\end{equation}
Then $\fC G(f)$ provides a reliable upper bound on $\norm[\cg]{f}$.  

\item [Stage 2.\ Estimate {$S(f)$}.] Choose the sample size need to approximate $S(f)$, namely, $n_A=N_{A}(\varepsilon/(\fC G(f)))$, where 
\begin{equation} \label{Nmindef}
N_{A}(a)= \min\left\{ n \in \ci : \min(h(n),\tau \tildeh(n)) \le a \right\}, \quad a \in (0,\infty).
\end{equation}
If $n_A \le N_{\max}-n_G$, then $S(f)$ may be approximated within the desired error tolerance and within the cost budget.  Set the warning flag, $W$, to false. Otherwise, recompute $n_A$ to be within budget, $n_A = \tN_{\max} := \max\{n \in \ci : n\le N_{\max} -  n_G\}$, and set the warning flag, $W$ to true.  Compute $A_{n_A}(f)$ as the approximation to $S(f)$.
\end{description}

Return the result $(A_{n_A}(f),W)$, at a total cost of $n_G+n_A$.  
\end{algo}



\begin{theorem}  \label{TwoStageDetermThm}  Let  $\cf$, $\cg$, $\ch$, $\varepsilon$, $N_{\max}$, $\tN_{\max}$, $\fC$, and $\tau$ be given as described in Algorithm \ref{twostagedetalgo}, and assume that $\cf$ satisfies \eqref{Fspacecond}.  Let 
\begin{equation} \label{normdeflate}
\fc =1 + \tau \err_{-}(G,\cf,\reals_+,\norm[\cg]{\cdot})  \ge 1.
\end{equation}
Let $\cc_\tau$ be the cone of functions defined in \eqref{conedef} whose $\cf$-semi-norms are no larger than $\tau$ times their $\cg$-semi-norms.  Let
\begin{align} 
\nonumber
\cn &= \left \{ f \in \cc_\tau : N_{A}\left(\frac{\varepsilon}{\fC \fc \norm[\cg]{f}} \right) \le \tN_{\max} \right\} \\
\label{nicefdef}
&= \left \{ f \in \cc_\tau : \norm[\cg]{f} \le \frac{\varepsilon}{\fC \fc \min(h(\tN_{\max}),\tau \tildeh(\tN_{\max}))} \right\}
\end{align}
be a subset of the cone $\cc_\tau$ that lies inside a $\cg$-semi-norm ball of rather large radius (since $\min(h(\tN_{\max}),\tau \tildeh(\tN_{\max}))$ is assumed to be tiny).  Then it follows that Algorithm \ref{twostagedetalgo} is successful for all functions in this set of \emph{nice} functions $\cn$,  i.e.,  $\success(A,W,\cn,\varepsilon,N_{\max}) = 1$.  Moreover, the cost of this algorithm is bounded above in terms of the $\cg$-semi-norm of the input function as follows:
\begin{equation} \label{auto2stagedetcost}
\cost(A,\cn,\varepsilon,N_{\max},\sigma)
\le n_G+ N_{A}\left(\frac{\varepsilon}{\fC \fc \sigma} \right).
\end{equation}
The upper bound on the cost of this specific algorithm provides an upper bound on the complexity of the problem, $\comp(\varepsilon,\ca(\cn,\ch,S,\Lambda),N_{\max},\sigma)$.  

Consider the limit of infinite cost budget, i.e., $N_{\max} \to \infty$.  If the sequence of algorithms $\{A_n\}_{n \in \ci}$, $A_n \in\ca_{\fix}(\cg,\ch,S,\Lambda)$  is nearly optimal for the problems $(\cg,\ch,S,\Lambda)$ and $(\cf,\ch,S,\Lambda)$ as defined in \eqref{nearoptdef}, then Algorithm \ref{twostagedetalgo} does not incur a significant penalty for not knowing $\norm[\cg]{f}$ a priori, i.e., for all $p>0$,
\begin{equation} \label{penalty}
\sup_{0 < \varepsilon/\sigma \le 1} \frac{\cost(A,\cc_{\tau},\varepsilon,\infty,\sigma)} {\comp(\varepsilon/\sigma,\ca_{\fix}(\cj,\ch,S,\Lambda))} \left(\frac{\varepsilon}{\sigma}\right)^p <\infty, \qquad \cj \in \{\cf,\cg\}.
\end{equation}
\end{theorem}

\begin{proof} The definition of $\fC$ in \eqref{norminflate} implies that the true $\cg$-semi-norm of $f$ is bounded above by $\fC G(f)$ according to Lemma \ref{Gnormlem}.  The upper bound on the error of the sequence of algorithms $\{A_n\}_{n \in \ci}$ in \eqref{Anerrbound} then implies that 
\[
\norm[\ch]{S(f) -  A_n(f)} \le \min(h(n),\tau \tildeh(n)) \fC G(f) \qquad \forall f \in \cc_{\tau}.
\]
This error upper bound may be made no greater than the error tolerance, $\varepsilon$, by choosing the algorithm cost, $n$, to satisfy the condition in Stage 2. of Algorithm \ref{twostagedetalgo}, provided that this can be done within the maximum cost budget.  In this case, the algorithm is successful, as claimed in the theorem.

To ensure that the algorithm does not attempt to overrun the cost budget, one must limit the $\cg$-semi-norm of the input function.  The definition of  $\fc$ in \eqref{normdeflate} implies that $G(f) \le \fc\norm[\cg]{f}$ according to Lemma \ref{Gnormlem}. This means that for any function, $f$, with actual $\cg$-semi-norm $\sigma=\norm[\cg]{f}$, the upper bound on its $\cg$-semi-norm computed via Lemma \ref{Gnormlem} is no greater than $\fC \fc \sigma$.  Thus, after using $N_G$ samples to estimate $\norm[\cg]{f}$, functions in $\cn$ as defined in \eqref{nicefdef} never need more than $N_{\max} - n_G$ additional samples to estimate $S(f)$ with the desired accuracy.  This establishes that Algorithm \ref{twostagedetalgo} must be successful for all $f \in \cn$.  It furthermore establishes an upper bound on the cost of the algorithm as given in \eqref{auto2stagedetcost}.

Now consider the penalty for not knowing $\norm[\cg]{f}$ in advance.  If the sequence of nonadaptive algorithms, $\{A_n\}_{n \in \ci}$, used to construct Algorithm \ref{twostagedetalgo} are nearly optimal for solving the problem on both $\cf$ and $\cg$, as defined in \eqref{nearoptdef}, then it follows that for $\cj \in \{\cf,\cg\}$,
\begin{multline*}
\sup_{0 < \varepsilon/\sigma \le 1} \frac{\cost(A,\cn,\varepsilon,\infty,\sigma)} {\comp(\varepsilon/\sigma,\ca_{\fix}(\cj,\ch,S,\Lambda))} \left(\frac{\varepsilon}{\sigma}\right)^{p} \\
= \sup_{0 < \varepsilon/\sigma \le 1} \frac{\cost(A,\cn,\varepsilon,\infty,\sigma)}{\min\{n \in \ci : \err(A_n,\cj,\ch,S) \le \varepsilon/\sigma\}} \left(\frac{\varepsilon}{\sigma}\right)^{p/2}  \\
 \times \sup_{0 < \varepsilon/\sigma \le 1} \frac{\min\{n \in \ci : \err(A_n,\cj,\ch,S) \le \varepsilon/\sigma\}} {\comp(\varepsilon/\sigma,\ca_{\fix}(\cj,\ch,S,\Lambda))} \left(\frac{\varepsilon}{\sigma}\right)^{p/2}
\end{multline*} 
The first of these suprema is finite by comparing the convergence rates of the sequence algorithms, $\{A_n\}_{n \in \ci}$, in \eqref{algseqerrbd} with the cost of the automatic algorithm given by \eqref{auto2stagedetcost}. The second of these suprema is finite for all $p>0$ by the near optimality of $\{A_n\}_{n \in \ci}$.  
\end{proof}

There are a several remarks that may facilitate understanding of this result.

\begin{rem} There are three main conditions to be checked for this theorem to hold.
\begin{enumerate}
\renewcommand{\labelenumi}{\roman{enumi}.}
\item An algorithm, $G$, to approximate the semi-norm in the larger space,  $\norm[\cg]{\cdot}$,  must be identified and its error must be bounded.
\item Both the error functions $h$ and $\tildeh$, for the sequence of nonadaptive algorithms, $\{A_n\}_{n \in \ci}$, must be computed explicitly for the the automatic algorithm to be defined.  
\item The near optimality of this sequence of nonadaptive algorithms must be verified to ensure that there is no significant penalty for not having an a priori upper bound on $\norm[\cg]{f}$.
\end{enumerate}
Sections \ref{integsec} and \ref{approxsec} provide concrete examples where these conditions are checked.
\end{rem}

\begin{rem} If $\tildeh$ is unknown, then one may take $\tildeh=\infty$, and the algorithm still satisfies the error tolerance with a cost upper bound given in \eqref{auto2stagedetcost}.  The optimality result in \eqref{penalty} then only holds for $\cg$, and not $\cf$.  The analogy holds if $h$ is unknown.  However, at least one of these two functions $h$ or $\tildeh$, must be known for this theorem to have a meaningful result.
\end{rem}

\begin{rem} The cost of Algorithm \ref{twostagedetalgo}, as given by \eqref{auto2stagedetcost}, depends on $\sigma$, which is essentially the $\cg$-semi-norm of the input function, $f$.  Thus, if $\sigma$ is smaller, the cost will correspondingly be smaller.  Moreover, $\sigma$ is not an input parameter for the algorithm.  Rather the algorithm reliably estimates $\norm[\cg]{f}$, and then adjusts the number of samples used (the cost) accordingly.
\end{rem}

\begin{rem}
The definition of the set of algorithms for which the Algorithm \ref{twostagedetalgo} is guaranteed to work, $\cn$, depends somewhat on $\norm[\cg]{f}$, but only because of the practical constraint of a cost budget of $N_{\max}$.  This dependence disappears if one lifts this constraint by taking $N_{\max} \to \infty$.  The primary constraint determining the success of the algorithm is that $f \in \cc_{\tau}$.
\end{rem}

\begin{rem} Instead of choosing $\tau$ as an input parameter for Algorithm \ref{twostagedetalgo}, one may alternatively choose the inflation factor $\fC >1$.  This then implies that 
\begin{equation} \label{taufromnC}
\tau = \left(1 - \frac{1}{\fC}\right)\frac{1}{\err_{+}(G,\cf,\reals_+,\norm[\cg]{\cdot})},
\end{equation}
which is equivalent to \eqref{norminflate}.
\end{rem}

\begin{rem} It is observed in the examples of Sections \ref{integsec} and \ref{approxsec} that for the sequence of algorithms $\{A_n\}_{n \in \ci}$
\begin{equation}
\tau \tildeh(n) \le h(n) \quad \forall n \in \ci.
\end{equation}
or equivalently, $\min(h(n),\tau \tildeh(n))=\tau\tildeh(n)$.  This then simplifies Algorithm \ref{twostagedetalgo} in the computation of the sample size for $A$ in \eqref{Nmindef} and also in Theorem \ref{TwoStageDetermThm} in the definition of $\cn$ in \eqref{nicefdef} and the upper bound on the cost in \eqref{auto2stagedetcost}. 
\end{rem}


\subsection{Embedded Nonadaptive Algorithms $\{A_n\}_{n \in \ci}$}

Suppose that the sequence of nonadaptive algorithms, 
\[
\{A_n\}_{n \in \ci} = \{A_{n_1}, A_{n_2}, \ldots \}, 
\]
are \emph{embedded}, i.e., $A_{n_{i+1}}$ uses all of the data used by $A_{n_{i}}$ for $i=1, 2, \ldots$.  An example would be a sequence of composite trapezoidal rules for integration that uses a number of trapezoids that is an increasing power of two. Furthermore, it is supposed that the data used by $G$, the algorithm used to estimate the $\cg$-semi-norm of $f$, is the same data used by $A_{n_1}$, and so $n_1=n_G$.  Then the total cost of Algorithm \ref{twostagedetalgo} can be reduced; it is simply $n_A$, as given in Stage 2, instead of $n_G+n_A$.  Moreover, $\tN_{\max}$ may then be taken to be $N_{\max}$, and the cost bound of the automatic algorithm in \eqref{auto2stagedetcost} does not need the term $n_G$.

Again suppose that $\{A_n\}_{n \in \ci}$, $A_n  \in \ca_{\fix}(\cg,\ch,S,\Lambda)$, consists of algorithms as described in  \eqref{algseqdef} and \eqref{algseqerrbd}, but some of which are embedded in others.  Specifically, suppose that for some fixed $r > 1$, for all $n \in \ci$, there exists at least one $\tn \in \ci$ with $n < \tn \le rn$, such that the data for $A_n$ is embedded in the data for $A_{\tn}$.  An example would be all possible composite trapezoidal rules for integration that use trapezoids of equal widths. Suppose also that there exists a sequence of algorithms for approximating the $\cg$-sem-inorm, $\{G_n\}_{n \in \ci}$, $G_n  \in \ca_{\fix}(\cg,\reals_+,\norm[\cg]{\cdot},\Lambda)$, such that for each $n \in \ci$, $A_n$ and $G_n$ use exactly the same data. Moreover, define the following terms analogously to \eqref{norminflate} and \eqref{normdeflate} for all $n \in \ci$:
\begin{equation} \label{frakCn}
\fC_{n} =\frac{1}{1 - \tau \err_{+}(G_n,\cf,\reals_+,\norm[\cg]{\cdot})}, \qquad
\fc_n =1 + \tau \err_{-}(G_n,\cf,\reals_+,\norm[\cg]{\cdot}).
\end{equation}
It is assumed that $\err_{\pm}(G_n,\cf,\reals_+,\norm[\cg]{\cdot})$ are non-increasing functions of $n$, which implies that $\fC_{n}$ and $\fc_n$ do not increase as $n$ increases. These embedded algorithms suggest the following iterative algorithm.

\begin{algo} \label{multistagealgo}  Let the Banach spaces $\cf$, $\cg$, and $\ch$, the solution operator $S$, and the error tolerance $\varepsilon$, the maximum cost budget $N_{\max}$, and the positive constant $\tau$ be as described in Algorithm \ref{twostagedetalgo}. Let the sequences of algorithms, $\{A_n\}_{n \in \ci}$ and  $\{G_n\}_{n \in \ci}$ be as described above.  Set $i=1$.  Let $n_1$ be the smallest number in $n \in \ci$ satisfying $\err_+(G_n,\cf,\norm[\cg]{\cdot},\reals_+) \le 1/\tau$. For any input function $f \in \cf$, do the following:
\begin{description}

\item [Stage 1. Estimate $\| f \| _{\cg}$.] Compute $G_{n_i}(f)$ and $\fC_{n_i} \le \infty$ as defined in \eqref{frakCn}.  

\item [Stage 2. Check for Convergence.] Check whether $n_i$ is large enough to satisfy the error tolerance, i.e., 
\begin{equation} \label{multistageconv}
\min(h(n_i),\tau \tildeh(n_i))\fC_{n_i} G_{n_i}(f) \le \varepsilon.
\end{equation}
If this is true, then set $W$ to be false, return $(A_{n_i}(f),W)$ and terminate the algorithm.

\item[Stage 3. Compute $n_{i+1}$.]  Otherwise, if the inequality above fails to hold, compute $\fc_i$ according to \eqref{normdeflate} using $G_{n_i}$. Choose $n_{i+1}$ as the smallest number exceeding $n_i$ and not less than $N_{A}(\varepsilon \fc_{n_i}/G_{n_i}(f))$ such that $A_{n_{i}}$ is embedded in $A_{n_{i+1}}$. If $n_{i+1} \le N_{\max}$, increment $i$ by $1$, and return to Stage 1.  

Otherwise, if $n_{i+1} > N_{\max}$, choose $n_{i+1}$ to be the largest number not exceeding $N_{\max}$ such that $A_{n_{i}}$ is embedded in $A_{n_{i+1}}$, and set $W$ to be true. Return $(A_{n_{i+1}}(f),W)$ and terminate the algorithm.
\end{description}  
\end{algo}

This iterative algorithm is guaranteed to converge also, and its cost can be bounded.  The following theorem is analogous to Theorem \ref{TwoStageDetermThm}.

\begin{theorem}  \label{MultiStageThm}  Let  $\cf$, $\cg$, $\ch$, $\varepsilon$, $N_{\max}$, $\tau$, and $n_1$ be given as described in Algorithm \ref{multistagealgo}. Assume that $n_1 \le N_{\max}$. Let $r$ be the number described in the paragraph preceding that algorithm.  Define 
\[
\tN_A(a) = \min\left\{ n \in \ci : \min(h(n),\tau \tildeh(n))\fC_n\fc_n \le a \right\}, \quad a \in (0,\infty).
\]
Let $\cc_\tau$ be the cone of functions defined in \eqref{conedef} whose $\cf$-semi-norms are no larger than $\tau$ times their $\cg$-semi-norms.  Let
\begin{align} 
\nonumber
\cn &= \left \{ f \in \cc_\tau :r\tN_{A}\left(\frac{\varepsilon}{\norm[\cg]{f}} \right) \le N_{\max} \right\} \\
\label{nicefdefmulti}
&= \left \{ f \in \cc_\tau : \norm[\cg]{f} \le \frac{\varepsilon}{\fC_{N_{\max}/r} \fc_{N_{\max}/r} \min(h(N_{\max}/r),\tau \tildeh(N_{\max}/r))} \right\}
\end{align}
be the nice subset of the cone $\cc_\tau$.  Then it follows that Algorithm \ref{multistagealgo} is successful for all functions in $\cn$,  i.e.,  $\success(A,W,\cn,\varepsilon,N_{\max}) = 1$.  Moreover, the cost of this algorithm is bounded above in terms of the $\cg$-semi-norm of the input function as follows:
\begin{equation} \label{automultistagedetcost}
\cost(A,\cn,\varepsilon,N_{\max},\sigma)
\le \max \left(n_1, r\tN_{A}\left(\frac{\varepsilon}{\sigma} \right) \right).
\end{equation}
The upper bound on the cost of this specific algorithm provides an upper bound on the complexity of the problem, $\comp(\varepsilon,\ca(\cn,\ch,S,\Lambda),N_{\max},\sigma)$.  

Consider the limit of infinite cost budget, i.e., $N_{\max} \to \infty$.  If the sequence of algorithms $\{A_n\}_{n \in \ci}$, $A_n \in\ca_{\fix}(\cg,\ch,S,\Lambda)$  is nearly optimal for the problems $(\cg,\ch,S,\Lambda)$ and $(\cf,\ch,S,\Lambda)$ as defined in \eqref{nearoptdef}, then Algorithm \ref{multistagealgo} does not incur a significant penalty for not knowing $\norm[\cg]{f}$ a priori, i.e., for all $p>0$,
\begin{equation*} \label{multistagepenalty}
\sup_{0 < \varepsilon/\sigma \le 1} \frac{\cost(A,\cc_{\tau},\varepsilon,\infty,\sigma)} {\comp(\varepsilon/\sigma,\ca_{\fix}(\cj,\ch,S,\Lambda))} \left(\frac{\varepsilon}{\sigma}\right)^p <\infty, \qquad \cj \in \{\cf,\cg\}.
\end{equation*}
\end{theorem}

\begin{proof} Let $n_1, \ldots, n_{j}$ be the sequence of $n_i$ generated by Algorithm \ref{multistagealgo}, $j$ being the number of the iterate where the algorithm either 
\begin{enumerate}
\renewcommand{\labelenumi}{\roman{enumi})}
\item terminates because the convergence criterion, \eqref{multistageconv}, is satisfied for $i=j$, or 

\item terminates with a warning because \eqref{multistageconv} is not satisfied for $i=j$, but the the proposed $n_{j+1}$ exceeds the cost budget, $N_{\max}$. 

\end{enumerate}
Here $j$ may be any positive integer.  The design of Algorithm \ref{multistagealgo} guarantees that $n_1 < \cdots < n_j$.  It is shown that under the hypotheses of this theorem, the algorithm does terminate without a warning and the error tolerance is satisfied.  

First, consider possibility i).  Note that by Lemma \ref{Gnormlem} and by \eqref{frakCn}, it follows that
\begin{equation}\label{twosidedGineq}
\frac{G_{n}(f)}{\fc_n} \le \norm[\cg]{f} \le \fC_{n}G_{n}(f) \qquad \forall n \in \ci.
\end{equation}
Since \eqref{multistageconv} is satisfied, it follows that $\norm[\ch]{S(f)-A_{n_j}(f)} \le \varepsilon$ by the same argument as given in the proof of Theorem \ref{TwoStageDetermThm}.  In this case the algorithm terminates without warning and the approximate value is within the required tolerance.

If $j=1$, then the cost of the algorithm is $n_1$.  If $j>1$, then the convergence criterion was not satisfied for $i=j-1$. Thus, it follows that $n_{j-1} < \tN_A(\varepsilon/\norm[\cg]{f})$, since $h(n), \tildeh(n), \fC_n$, and $\fc_n$ all do not decrease as $n$ increases. 
If $n_{j-1} \ge N_{A}(\varepsilon \fc_{n_{j-1}}/G_{n_{j-1}}(f))$, then Stage 3 chooses $n_{j}$ to be the smallest element of $\ci$ that exceeds $n_{j-1}$ and for which $A_{n_{j-1}}$ is embedded in $A_{n_j}$.  By the definition of $r$ it follows that
\begin{equation*}
n_{j} \le r  n_{j-1} < r \tN_{A}(\varepsilon/\norm[\cg]{f}).
\end{equation*}
If, on the other hand, $n_{j-1} < N_{A}(\varepsilon \fc_{n_{j-1}}/G_{n_{j-1}}(f))$, then Stage 3 chooses $n_{j}$ to be the smallest element of $\ci$ that is no less than $N_{A}(\varepsilon \fc_{n_{j-1}}/G_{n_{j-1}}(f))$ and for which $A_{n_{j-1}}$ is embedded in $A_{n_j}$.  By the definition of $r$, \eqref{twosidedGineq}, and the definition of $\tN_A$, it follows that
\begin{equation*}
n_{j} < r  N_{A}(\varepsilon \fc_{n_{j-1}}/G_{n_{j-1}}(f)) \le r  N_{A}(\varepsilon/\norm[\cg]{f}) \le r \tN_{A}(\varepsilon/\norm[\cg]{f}).
\end{equation*}
In both cases, the algorithm chooses $n_{j} < r \tN_{A}(\varepsilon/\norm[\cg]{f})$.  Thus, the cost of the algorithm is bounded as in \eqref{automultistagedetcost}.

Second, consider possibility ii), meaning that the convergence criterion, \eqref{multistageconv}, is not satisfied for $i=j$.  Then Stage 3 tries to choose $n_{j+1}$ to satisfy this criterion.  Using similar arguments as in the previous paragraph, it follows that $n_j < \tN_A(\varepsilon/\norm[\cg]{f})$. 
If $n_j \ge N_{A}(\varepsilon \fc_{n_j}/G_{n_j}(f))$, then the proposed $n_{j+1}$ satisfies
\begin{equation*}
n_{j+1} \le r  n_j < r \tN_{A}(\varepsilon/\norm[\cg]{f}).
\end{equation*}
If, on the other hand, $n_j < N_{A}(\varepsilon \fc_{n_j}/G_{n_j}(f))$, then the proposed $n_{j+1}$ satisfies
\begin{equation*}
n_{j+1} < r  N_{A}(\varepsilon \fc_{n_j}/G_{n_j}(f)) \le r  N_{A}(\varepsilon/\norm[\cg]{f}) \le r \tN_{A}(\varepsilon/\norm[\cg]{f}).
\end{equation*}
In both cases, the $n_{j+1}$ proposed by the algorithm satisfies $n_{j+1} < r \tN_{A}(\varepsilon/\norm[\cg]{f})$, which does not exceed the cost budget by the  by the definition of $\cn$.  Thus, possibility ii) cannot happen.

The proof of the optimality of the multistage algorithm follows the same line of argument used to prove the optimality of the two-stage algorithm in Theorem \ref{TwoStageDetermThm}.  This completes the proof.
\end{proof}

\subsection{Lower Complexity Bounds for the Algorithms}
Lower complexity bounds are typically proved by constructing fooling functions.  First, a lower bound is derived for the complexity of problems defined on $\cf$- and $\cg$-semi-norm balls of input functions.  This technique is generally known \cite{???}.  Then it is shown how to extend this idea for the cone $\cc_{\tau}$.  

Consider the Banach spaces $\cf$, $\cg$, $\ch$, and the \emph{linear} solution operator $S: \cg \to \ch$.  Let $\Lambda$ be the set of bounded linear functionals that can be used as data. Suppose that for any $n>0$, and for all $\vL \in \Lambda^m$, satisfying $\$(\vL)\le n$, there exists an $f_1 \in \cf$, depending on $n$ and the $L_i$, with solution norm one, zero data, and bounded $\cf$ and $\cg$ semi-norms, i.e.,
\begin{equation} \label{assumpfone}
\norm[\ch]{S(f_1)} = 1, \quad \vL(f_1)= \vzero, \quad
\norm[\cg]{f_1} \le g(n), \quad \norm[\cf]{f_1} \le \tg(n) \norm[\cg]{f_1}, 
\end{equation}
for some positive, non-decreasing functions $g$ and $\tg$ defined in $(0,\infty)$.  For example, one might have $g(n)=an^p$ and $\tg(n)=bn^q$ with positive $a, b, p$, and $q$.  Since the data for the function $f_1$ are all zero, it follows that $A(f_1)=A(-f_1)$ for any algorithm, $A$, adaptive or not, that is based on the information $\vL(f_1)$.  Then, by the triangle inequality and \eqref{assumpfone} the error for one of the fooling functions $\pm f_1$ must be at least one:
\begin{align*}
\MoveEqLeft{\max(\norm[\ch]{S(f_1)-A(f_1)},\norm[\ch]{S(-f_1)-A(-f_1)})} \\
& = \max(\norm[\ch]{S(f_1)-A(f_1)},\norm[\ch]{-S(f_1)-A(f_1)})\\
& \ge \frac{1}{2} \left[ \norm[\ch]{S(f_1)-A(f_1)}+ \norm[\ch]{S(f_1)+A(f_1)} \right] \\
& \ge \frac{1}{2} \norm[\ch]{[S(f_1)-A(f_1)]+[S(f_1)+A(f_1)]} = \norm[\ch]{S(f_1)}=1.
\end{align*}
Furthermore, applying \eqref{assumpfone}, for $\cj \in \{\cf,\cg\}$, it follows that any nonadaptive algorithm satisfying the error tolerance $\varepsilon$ must have a cost $n$ satisfying the following inequality:
\begin{align*}
\varepsilon & \ge \sup_{f \ne 0} \frac{\norm[\ch]{S(f)-A(f)}}{\norm[\cj]{f}} \\
& \ge \frac{\max(\norm[\ch]{S(f_1)-A(f_1)},\norm[\ch]{S(-f_1)-A(-f_1)})}{\norm[\cj]{f_1}} \\
& \ge \frac{1}{\norm[\cj]{f_1}} 
\ge \begin{cases} \displaystyle \frac{1}{g(n)}, & \cj=\cg,\\[2ex]
 \displaystyle  \frac{1}{g(n)\tg(n)}, & \cj=\cf.
\end{cases}
\end{align*}
This implies lower bounds on the complexity of nonadaptive algorithms, as defined in \eqref{fixcostcomplex}:
\[
\comp(\varepsilon,\ca_{\fix}(\cj,\ch,S,\Lambda)) \ge
\begin{cases} g^{-1}(\varepsilon^{-1}), & \cj=\cg,\\
(g\tg)^{-1}(\varepsilon^{-1}), & \cj=\cf,
\end{cases}
\]
where $g^{-1}$ and $(g\tg)^{-1}$ denote the inverse functions of $g$ and $g\tg$, respectively.
Thus, the cost of solving the problem within error tolerance $\varepsilon$ for input functions in a $\cg$-semi-norm ball of radius $\sigma$ is at least $g^{-1}(\sigma\varepsilon^{-1})$ and for input functions in a $\cf$-semi-norm ball of radius $\sigma$ is at least $(g\tg)^{-1}(\sigma\varepsilon^{-1})$

Turning to the problem of solving functions in the cone $\cc_{\tau}$, the lower bound on the complexity becomes a bit more difficult to derive.  Note that condition \eqref{assumpfone} allows the fooling function $f_1$ to lie \emph{outside} this cone for $bn^q > \tau$.  Thus, when considering the cone of input functions, the fooling function must be modified as described below.

It is assumed that there exists a function $f_0$ with non-zero $\cg$-semi-norm lying in the interior of the cone $\cc_{\tau}$, i.e.,
\begin{equation}
\label{assumpfzero}
\norm[\cg]{f_{0}} > 0, \qquad \norm[\cf]{f_{0}} \le \tau_0 \norm[\cg]{f_{0}}, \qquad \tau_0 < \tau.
\end{equation}
Furthermore, suppose that for each $n>0$, and for all $\vL \in \Lambda^m$, satisfying $\$(\vL)\le n$, there exists $f_1$ as described above in \eqref{assumpfone}. Under these assumptions, one may show the following lower bound on the complexity of solving the problem $S$ for functions in the cone $\cc_{\tau}$.

\begin{theorem} \label{complowbd} Suppose that functions $f_{0}$ and $f_1$ can be found that satisfy conditions \eqref{assumpfone} and \eqref{assumpfzero}.  It then follows that the complexity of the problem, defined by \eqref{complexdef}, assuming infinite cost budget, over the cone of functions $\cc_{\tau}$ is
\begin{multline*}
\comp(\varepsilon,\ca(\cc_{\tau},\ch,S,\Lambda),\infty,\sigma) \\
\ge \min\left(g^{-1}\left(\frac{\sigma(\tau-\tau_0)}{2(2\tau-\tau_0)\varepsilon}\right), (g\tg)^{-1}\left(\frac{\sigma\tau(\tau-\tau_0)}{2(2\tau-\tau_0)\varepsilon}\right) \right).
\end{multline*}
\end{theorem}

\begin{proof} Let $A$ be a successful, possibly adaptive, algorithm for all functions lying in the cone $\cc_{\tau}$.  Given an error tolerance, $\varepsilon$, and a positive $\sigma$, let $f_0$ be a function satisfying \eqref{assumpfzero} and choose 
\begin{subequations}\label{c0c1bumpdef}
\begin{equation} 
\label{c0bumpdef}
c_0 = \frac{\sigma\tau}{\norm[\cg]{f_0} (2\tau - \tau_0)}.
\end{equation} 
Provide the algorithm $A$ with the input function $c_0f_0$, and let $\vL(c_0f_0)=c_0\vL(f_0)$ be the data vector extracted by $A$ to obtain the estimate $A(c_0f_0)$. Let $n=\$(\vL)$ denote the cost of this algorithm for the function $c_0f_0$, and define two fooling functions, $f_{\pm}=c_0f_{0} \pm c_1 f_1$, in terms of the $f_{0}$ and $f_1$ satisfying conditions \eqref{assumpfone} with $c_1$ satisfying
\begin{equation} 
\label{c1bumpdef}
c_1= \frac{(\tau-\tau_0)c_0\norm[\cg]{f_0}}{\norm[\cg]{f_1} [\tg(n) + \tau]} = \frac{\sigma \tau (\tau-\tau_0)}{\norm[\cg]{f_1}(2\tau - \tau_0) [\tg(n) + \tau]}.
\end{equation}
\end{subequations}
These fooling functions must lie inside the cone $\cc_{\tau}$ because
\begin{align*}
\norm[\cf]{f_{\pm}} - \tau  \norm[\cg]{f_{\pm}} & \le  c_{0}\norm[\cf]{f_0} + c_1 \norm[\cf]{f_1} - \tau (c_{0}\norm[\cg]{f_0} - c_1 \norm[\cg]{f_1}) \\
&\qquad \qquad \qquad \qquad \qquad \qquad \text{by the triangle inequality} \\
& \le c_1 [\tg(n)+\tau] \norm[\cg]{f_1} - (\tau - \tau_0) c_{0} \norm[\cg]{f_0} \qquad \text{by \eqref{assumpfone}, \eqref{assumpfzero}} \\
& = 0 \qquad \qquad \text{by \eqref{c0c1bumpdef}}.
\end{align*}
Moreover, both fooling functions have $\cg$-semi-norms no greater than $\sigma$, since
\begin{align*}
\norm[\cg]{f_{\pm}} &\le c_{0} \norm[\cg]{f_0} + c_1 \norm[\cg]{f_1} \\
& = \frac{\sigma \tau}{2\tau - \tau_0}\left[1 + \frac{\tau-\tau_0}{\tg(n) + \tau} \right] \qquad \text{by \eqref{c0c1bumpdef}}\\
&\le \frac{\sigma\tau}{2\tau - \tau_0}\left[1 + \frac{\tau-\tau_0}{\tau} \right] = \sigma.
\end{align*}

Following the argument earlier in this section, it is noted that the data used by algorithm $A$ for both fooling functions is the same, i.e., $\vL(f_{\pm})=\vL(c_0f_0)$, and so $A(f_{\pm})=A(c_0f_0)$.  Consequently, by the same argument used above, 
\[
\varepsilon  \ge  \max(\norm[\ch]{S(f_+)-A(f_+)},\norm[\ch]{S(f_-)-A(f_-)}) \ge c_1 \norm[\ch]{S(f_1)}=c_1.
\]
Since $A$ is successful for these two fooling functions, $c_1$, as defined in \eqref{c0c1bumpdef}, must be no larger than the error tolerance, which implies by \eqref{assumpfone} that 
\begin{align*}
\frac{\sigma \tau(\tau-\tau_0)}{\varepsilon} & \le \frac{\sigma \tau(\tau-\tau_0)}{c_1}  = \norm[\cg]{f_1} (2\tau - \tau_0) [\tg(n) + \tau] \\
&\le (2\tau - \tau_0) g(n) [\tg(n) + \tau)] \\
&\le 2 (2\tau - \tau_0) g(n)\max( \tg(n), \tau).
\end{align*}
Since $A$ is an arbitrary successful algorithm, this inequality provides a lower bound on the cost, $n$, that any such algorithm requires.  This then implies the lower bound on the complexity of the problem.   
\end{proof}

\begin{rem} Now suppose that the sequence of nonadaptive algorithms used to construct the adaptive, automatic Algorithm \ref{twostagedetalgo}, and the fooling functions in Theorem \ref{complowbd} have comparable powers, namely $p_1=p$ and $p_2=p+q$.  It then follows by comparing the upper bound on the cost in Theorem \ref{TwoStageDetermThm} to the lower bound in Theorem \ref{complowbd} that Algorithm \ref{twostagedetalgo} is optimal. 
\end{rem}

%\begin{rem} To derive a lower bound on the complexity of solving the problem over a ball in $\cf$ or $\cg$, rather than the cone, $\cc_{\tau}$, one does \emph{not} need the function $f_0$ to define the fooling functions.  Choosing $f_{\pm}$ as multiples of $f_1$ suffices.  
%\end{rem}

\section{Approximation of One-Dimensional Integrals} \label{integsec}

The first example we consider is univariate integration on the unit interval, $S(f)=\int_{0}^{1}f(x) \, \dif x$.  The two spaces of input functions are Sobolev spaces of different degrees of smoothness:
\begin{equation*}
  \cf=\mathcal{W}^{2,1} \quad \text{and} \quad
  \cg=\mathcal{W}^{1,1},
\end{equation*}
where the general Sobolev spaces with smoothness of degree $k$ are defined as 
\begin{equation} \label{defSobolev}
  \mathcal{W}^{k,p}=\mathcal{W}^{k,p}[0,1]=\{f\in C[0,1]: \|f^{k}\|_{p}<\infty\}.
\end{equation}
The corresponding semi-norms are $\|f\|_{\cf}=\|f''\|_{1}$ and $\|f\|_{\cg}=\|f'\|_{1}$, respectively, and the cone of integrands is defined as  
\begin{equation}\label{coneinteg}
\cc_{\tau}=\{f\in \mathcal{W}^{2,1}:\|f''\|_1\leq\tau\|f'\|_1\}.
\end{equation}
The space of outputs $\ch$ is $\reals$.

The non-adaptive building block to construct the adaptive algorithm is the composite trapezoidal rule based on $n-1$ trapezoids:
\begin{equation*}
    A_{n}(f)
    =\frac{1}{2n}[f(x_1)+2f(x_2)+\cdots+2f(x_{n-1})+f(x_n)], \qquad x_i=\frac{i-1}{n-1},
\end{equation*}
defined for $n \in \mathcal{I}=\{2,3,\ldots\}$, and with $\cost(A_n)=n$.  The algorithm $A_n$ is imbedded in the algorithm $A_{2n-1}$, which uses $2n-2$ trapezoids, so the minimum cost multiple is $r=2$.  
The algorithm for approximating $\|f'\|_{1}$ is the corresponding norm of the linear spline using the same $x_i$ as used in the trapezoidal rule:
\begin{equation}\label{1direst}
    G_n(f)=\sum_{i=0}^{n-1}\left|f(x_{i+1})-f(x_{i})\right|.
\end{equation} 

\subsection{Adaptive Algorithm and Upper Bound on the Cost}

Constructing the adaptive algorithm for integration requires upper bounds on the errors of the $A_n$ and $G_n$.  The errors of the trapezoidal rule for integrands in the spaces $\mathcal{W}^{2,1}$ and $\mathcal{W}^{1,1}$ are bounded in terms of $h(n)=1/[8(n-1)^2]$ and $\tilde{h}(n)=1/[2(n-1)]$, respectively, according to \cite[(7.14) and (7.15)]{BraPet11a}.  Since $G_{n}(f)$ never overestimates $\|f'\|_{1}$, it follows that $h_{\cg_{-}}(n)=0$ and $\mathfrak{c}_n=1$. 

To find an upper bound on $\|f'\|_{1}-G_{n}(f)$, note that for $x_i \le x_{i+1}$
\begin{subequations}
\begin{equation}
f'(x)=(n-1)\left\{[f(x_{x+1})-f(x_{i})]+\int_{x_i}^{x_{i+1}}(n-1)v(t,x)f''(t) \, \dif t \right\}
\end{equation}
where 
\begin{equation}
v(t,x)=\begin{cases}  x_i-t, & x_i\leq t\leq x,\\
t-x_{i+1}, & x< t \leq x_{i+1}.
\end{cases}
\end{equation}
\end{subequations}
This implies the following upper bound on a piece of $\|f'\|_{1}$
\begin{align*}
\MoveEqLeft{\int_{x_i}^{x_{i+1}}|f'(x)| \, \dif x - |f(x_{i+1})-f(x_i)|}\\
& \le (n-1)\int_{x_i}^{x_{i+1}}\int_{x_i}^{x_{i+1}}|v(t,x)||f''(t)| \dif t \, \dif x\\
& \le  (n-1)\int_{x_i}^{x_{i+1}}2(t-x_i)(x_{i+1}-t)|f''(t)| \, \dif t \\
& \le  (n-1) \max_{x_i \le t \le x_{i+1}} \abs{2(t-x_i)(x_{i+1}-t)} \int_{x_i}^{x_{i+1}} |f''(t)| \, \dif t \\
 &  \leq  \frac{1}{2(n-1)}\int_{x_i}^{x_{i+1}} |f''(t)| \, \dif t .
\end{align*}
Applying this inequality over the entire interval, $[0,1]$ leads to 
\begin{align*}
\norm[1]{f'} - G_n(f)  &  = \sum_{i=1}^{n-1} \left \{  \int_{x_i}^{x_{i+1}}|f'(x)|\, \dif x - |f(x_{i+1})-f(x_i)| \right \} \\
& \le \frac{1}{2(n-1)} \sum_{i=1}^{n-1} \int_{x_i}^{x_{i+1}} |f''(t)| \, \dif t = 
\frac{\|f''(t)\|_{1}}{2(n-1)}.
 \end{align*}
According to \eqref{Gerrbds} and , we have $h_{+}(n)=1/[2(n-1)]$ and an inflation factor of 
\begin{equation}\label{factor}
\mathfrak{C}_n =\frac{1}{1 - \tau/(2n-2)} \geq 1 \qquad \text{for } n>1+\tau/2.
\end{equation}

By condition \eqref{hbestverb}, it follows that $\min(\tildeh(n),\tau h(n))=\tau h(n)$, which then simplifies some of the expressions in Algorithm \ref{multistagealgo} and Theorem \ref{MultiStageThm}.  Specifically, the left side of the inequality in \eqref{multistageconv} becomes
\[
\min(\tildeh(n_i),\tau h(n_i))\fC_{n_i} G_{n_i}(f) = \frac{\tau  G_{n_i}(f) } {4(n_i-1)(2n_i-2 -\tau)}.
\]
The function $N_A$ defined in \eqref{Nmindef} is
\[
N_{A}(a)= \min\left\{ n \in \ci : \tau h(n) \le a \right\} = 1+ \left \lceil \sqrt{a/(8\tau)}\right \rceil.
\]
The denominator in the definition of the set of integrands $\cn$ in \eqref{nicefdefmulti} is
\[
\fC_{N_{\max}/r} \fc_{N_{\max}/r} \min(\tildeh(N_{\max}/r),\tau h(N_{\max}/r)) =
\frac{\tau}{2 (N_{\max}-2)(N_{\max}-2 -\tau)}.
\]
The function $\tN_A$ defined in \eqref{tNmindef} is
\begin{align*} \label{tNmindef}
\tN_A(a) &= \min\left\{ n \in \ci : \min(\tildeh(n),\tau h(n))\fC_n\fc_n \le a \right\} \\
&= 1+ \left \lceil \sqrt{\frac{\tau}{8a} + \frac{\tau^2}{16}} +\frac{\tau}{4} \right \rceil \le 1+ \left \lceil\sqrt{\frac{\tau}{8a}} + \frac{\tau}{2} \right \rceil.
\end{align*}
With these preliminaries, Algorithm \ref{multistagealgo} and Theorem \ref{MultiStageThm} may be applied directly to  yield the following automatic, adaptive integration algorithm and its guarantee.

\begin{algo} \label{multistageintegalgo}
Let the error tolerance $\varepsilon$, the maximum cost budget $N_{\text{max}}$ and the cone constant $\tau$ be given inputs. Let the sequence of algorithms $\{A_n\}_{n\in \mathcal{I}}$, $\{G_n\}_{n\in \mathcal{I}}$ be described above. Set $i=1$. Let $n_1=\lceil(\tau+1)/2\rceil+1$. For any input function $f\in \mathcal{W}^{2,1}$, do the following:
\begin{description}
\item[Stage 1.\ Estimate {$\|f'\|_{1}$}.] Compute $G_{n_i}(f)$ in \eqref{1direst}.

\item[Stage 2. Check for convergence.] Check whether $n_i$ is large enough to satisfy the error tolerance, i.e.
    \begin{align*}
     G_{n_i}(f) \le \frac{4\varepsilon(n_i-1)(2n_i-2 - \tau)}{\tau}.
    \end{align*}
    If this is true, then set $W$ to be false, return $(A_{n_i}(f),W)$ and terminate the algorithm. If this is not true, go to Stage 3.

\item[Stage 3. Compute $n_{i+1}$.] Otherwise, if the inequality above fails to hold,
choose
$$
n_{i+1}=1+ (n_i-1)\max\left\{\left\lceil\frac{1}{(n_i-1)}\sqrt{\frac{\tau G_{n_i}(f)}{8\varepsilon}}\right\rceil,2\right\}.
$$
If $n_{i+1} \le N_{\max}$, increment $i$ by $1$, and return to Stage 1.

Otherwise, if $n_{i+1} > N_{\max}$, choose $n_{i+1}$ to be the largest number not exceeding $N_{\max}$ such that $A_{n_{i}}$ is embedded in $A_{n_{i+1}}$, and set $W$ to be true. Return $(A_{n_{i+1}}(f),W)$ and terminate the algorithm.
\end{description}
\end{algo}

\begin{theorem} Let  $\varepsilon$, $N_{\max}$, $\tau$ and $n_1$ be given as described in Algorithm \ref{multistageintegalgo}.  Assume that $n_1 \le N_{\max}$.  Let $\cc_\tau$ be the cone of functions defined in \eqref{coneinteg}.  Let
$$
\cn
= \left \{ f \in \cc_\tau : \norm[1]{f'} \le \frac{2\varepsilon (N_{\max}-2)(N_{\max}-2 -\tau)}{\tau} \right\}
$$
be the nice subset of the cone $\cc_\tau$.  Then it follows that Algorithm \ref{multistageintegalgo} is successful for all functions in $\cn$,  i.e.,  $\success(A,W,\cn,\varepsilon,N_{\max}) = 1$.  Moreover, the cost of this algorithm is bounded above as follows:
\begin{align*}
\cost(A,\cn,\varepsilon,N_{\max},\norm[\infty]{f'})
&\le 2+2 \left \lceil \sqrt{\frac{\tau \norm[1]{f'}}{8\varepsilon} + \frac{\tau^2 }{16}} + \frac{\tau}{4} \right\rceil \\
&\le 2+2 \left \lceil\sqrt{\frac{\tau \norm[1]{f'}}{8\varepsilon}} + \frac{\tau}{2} \right \rceil.
\end{align*}
\end{theorem}

\subsection{Lower Bound on the Computational Cost}
Next, we derive a lower bound on the cost of approximating functions in the cone $\cc_{\tau}$ by constructing fooling functions. Following the arguments of Section \ref{LowBoundSec}, we choose  $f_0(x)=x.$ Then
\[
\norm[1]{f'_0}=1, \qquad \norm[1]{f''_0}=0, \qquad \tau_0=0.
\]
For any $n =2, 3, \ldots$, suppose that the one has the data $L_i(f)=f(\xi_i)$, $i=1, \ldots, n$ for arbitrary $\xi_i$, where $0=\xi_0 \le \xi_1 < \cdots < \xi_n \le \xi_{n+1} = 1$.  There must be some $j=0, \ldots, n$ such that $\xi_{j+1} - \xi_j \ge 1/(n+1)$.  The function $f_{1}$, defined as
$$
f_{1}(x):=\begin{cases} \displaystyle
\frac{30(x-\xi_{j})^{2}(\xi_{j+1}-x)^{2}}{(\xi_{j+1}-\xi_{j})^5} & \xi_{j} \le x \leq \xi_{j+1},\\
0 & \text{otherwise},
\end{cases}
$$
has $\int_0^1 f(x) \, \dif x=1$ and $f_1(\xi_i)=0$ for $i=0, \ldots, n+1$.  Moreover, the norms of the first and second derivatives of $f_1$ are
\begin{align*}
\norm[1]{f'_1}&=\frac{15}{4(\xi_{j+1}-\xi_{j})}\leq \tg(n),\\
\norm[1]{f''_1}&=\frac{20}{\sqrt{3}(\xi_{j+1}-\xi_{j})^2}
=\frac{16\norm[1]{f'_1}}{3\sqrt{3}(\xi_{j+1}-\xi_{j})}
 \leq g(n) \norm[1]{f'_1},
\end{align*}
where the inequality $\xi_{j+1} - \xi_j \ge 1/(n+1)$ is used to define $\tg$ and $g$ as
\[
\tg(n) = \frac{15(n+1)}{4}, \qquad g(n) = \frac{16(n+1)}{3\sqrt{3}}, \qquad (g\tg)(n) = \frac{20(n+1)^2}{\sqrt{3}}.
\]
Using these choices of $f_0$ and $f_1$, along with the corresponding $\tg$ and $g$ above, one may invoke Proposition \ref{optimalprop}, Theorem \ref{complowbd}, and Corollary \ref{optimcor} to obtain the following theorem.

\begin{theorem} \label{complowbdappr} The adaptive trapezoidal algorithm is optimal for integration of functions in $\mathcal{W}^{1,1}$ and $\mathcal{W}^{2,1}$. Assuming an infinite cost budget, the complexity of the function recovery problem problem, over the cone of functions $\cc_{\tau}$ defined in \eqref{coneinteg} is
\begin{align*}
\comp(\varepsilon,\ca(\cc_{\tau},\ch,S,\Lambda),\infty,\norm[1]{f'})
&\ge \min\left(\frac{\norm[1]{f'}}{15 \varepsilon}, \sqrt{\frac{\sqrt{3}\tau\norm[1]{f'}}{80\varepsilon}} \right)-1\\
& \sim \sqrt{\frac{\sqrt{3}\tau\norm[1]{f'}}{80\varepsilon}}  \quad\text{as } \frac{\norm[\infty]{f'}}{\varepsilon} \to \infty.
\end{align*}
Moreover, Algorithm \ref{multistageapproalgo} is optimal for approximating functions in $\cc_{\tau}$.
\end{theorem}

\subsection{Numerical Example}
Consider the test function $f(x)=e^{-(a(x-\mu))^2},$
where $\mu$'s are uniformly distributed on $[-1/\sqrt{2}a,1-1/\sqrt{2}a],$ and $\log_{10} a \sim U[0,4].$
If we have large $a$, we can hardly approximate the function due to the spiky shape of the function. From the numerical results, we can find the success rate for our algorithm. Since $\norm[1]{f'}=-e^{-(a\mu)^2}-e^{-(a(1-\mu))^2}, \ \norm[1]{f''}=-2a(\mu e^{-(a\mu)^2}+(1-\mu)e^{-(a(1-\mu))^2})$, if $f \in \cc_{\tau}$ then we should have
$\norm[1]{f''}/\norm[1]{f'} \leq \tau$. We use Monte Carlo simulation with 1000000 samples to estimate this probability. To obtain the observed success rate, here we use $\tau = 10, 25 , 100$ separately to test $10000$ times with $\varepsilon = 1e-7$.
\begin{table}[h]
\centering
\begin{tabular}{cccc}
$\tau$ &  10 & 25 & 100\\
\toprule
Probability of $f \in \mathfrak{C}_{\tau}$ &  $\leq 45 \%$ &  $\leq 50 \%$  & $\leq 54 \%$ \\
Observed Success Rate & $52\%$ &  $60\%$  & $74\%$ \\
\end{tabular}
\caption{ Comparison between probability of the input function lying in the cone and the empirical success rate of the algorithm}
\end{table}

From the above table, we can find our algorithm works very well, as the observed
success rate is always larger than the expected success rate.


\section{$\cl_{\infty}$ Approximation of Univariate Functions} \label{approxsec}


In this part, we consider the function recovery $S(f)=f$ and $\ch=\cl_{\infty}.$
Then $\norm[\ch]{S(f)-A(f)}=\norm[\infty]{f-A(f)}.$
Let $\cg \subset \mathcal{W}^{1,\infty}[0,1]$ and $\cf \subset \mathcal{W}^{2,\infty}[0,1],$ where
$$\mathcal{W}^{k,\infty}[0,1]=\left\{u \in \cl_{\infty}[0,1]:D^\alpha u \in \cl_\infty[0,1], \forall |\alpha| \leq k \right\}.$$

Given $\tau>0$, then by the \eqref{conedef}, we have $\cc_{\tau} \subset \cf$ as
\begin{equation}\label{taudef}
\cc_{\tau}=\{f \in  \mathcal{W}^{2,\infty}[0,1] : \norm[\infty]{f''}\leq \tau\norm[\infty]{f'}\},
\end{equation}

Suppose one wants to approximate functions in $\cc_{\tau}$ based on function values.  The goal is to find an algorithm, $A$ for which $\norm[\infty]{f-A(f)} \le \varepsilon.$ Here we use piecewise linear interpolation to approximate. Thus $\forall f \in \cc_{\tau},$ consider the sequence of algorithms $\{A_n\}_{n \in \ci},$ where
$\ci=\{2,3,\dots\}$
 and given the data $f(x_i),$ where $ x_i=\frac{i}{n-1}, i=0, \ldots,n-1,$ and $\cost(A_{n})=n.$
For $x_{i-1} \leq x \leq x_i, \ i=1, \ldots, n-1,$ define $A_{n}(f)$
$$A_{n}(f)(x)=f(x_{i-1})+\frac{x-x_{i-1}}{x_i-x_{i-1}}\left(f(x_i)-f(x_{i-1})\right).$$

\subsection{Upper Bound}

%Fix the algorithm $A_n,$ which $\cost(A_n)=n,$
%suppose the data sites $x_0, x_1, \dots, x_{n-1}$ and the corresponding function values. One may estimate $\norm[\infty]{f'}$ in terms of divided differences:
%\begin{equation} \label{estfirstderiv}
%G_{n}(f)= \max_{i=1, \ldots, n-1} (n-1)\abs{f(x_i)-f(x_{i-1})}.
%\end{equation}

% Suppose that $\zeta$ is some point satisfying $\abs{f'(\zeta)} = \norm[\infty]{f'}$ and
%that $x_{i-1}\le \zeta < x_i$.  It then follows that $f(x_i)-f(x_{i-1}) = f'(\eta)/(n-1)$ for some $\eta$ between $x_{i-1}$ and $x_i$.  Thus,
%%\begin{align*}
%%\norm[\infty]{f'} &= \abs{f'(\zeta)}\leq \abs{f'(\eta)} + \abs{f'(\eta)-f'(\zeta) }  \le G_{n}(f) + \abs{ \eta-\zeta}^{s}\sup\limits_{x,t \in [0,1]}\frac{|f'(x)-f'(t)|}{|x-t|^{s}}\nonumber\\
%%& \leq  G_n(f) + \frac{1}{m^{s}} \sup\limits_{x,t \in [0,1]}\frac{|f'(x)-f'(t)|}{|x-t|^{s}}\nonumber\\
%%&\Rightarrow \err_{+}(G_n,\cf,\reals_{+},\norm[\cg]{\cdot}) =\frac{1}{(n-1)^{s}}
%%\end{align*}
%\begin{align*}
%\norm[\infty]{f'} &= \abs{f'(\zeta)}\leq \abs{f'(\eta)} + \abs{f'(\eta)-f'(\zeta) }  \le G_{n}(f) + \abs{\int_\zeta^\eta f''(x)dx }\nonumber\\
%& \leq  G_n(f) + \norm[\infty]{f''}|\eta-\zeta| \leq G_n(f)+ \frac{1}{n-1}\norm[\infty]{f''}\nonumber\\
%&\Rightarrow \err_{+}(G_n,\cf,\reals_{+},\norm[\cg]{\cdot}) =\frac{1}{n-1}.
%\end{align*}


Given the data sites $x_i=(i-1)/(n-1)$, $i=0, \ldots, n$, let the $\cl_\infty$ norm of the $f'$ be approximated by
\begin{equation}\label{estfirstderiv}
G_n(f) = (n-1) \sup_{i=1, \ldots, n-1} \abs{f(x_{i+1})-f(x_i)}
\end{equation}

Note that $G_n(f)$ never overestimates $\norm[\infty]{f'}$. Thus, we have
$h_{-}(n)=0.$
For any $x \in [x_{i}, x_{i+1}]$, note that
\begin{align*}
\MoveEqLeft{\abs{f'(x)} - (n-1) \abs{f(x_{i+1})-f(x_i)}} \\
 & \le \abs{f'(x) - (n-1)[f(x_{i+1})-f(x_i)]} \\
&=\abs{ \int_{x_i}^{x_{i+1}} f''(t) [(n-1)(t-x_i) -  1_{[x,x_{i+1}]}(t)] \, \dif t} \\
& \le \sup_{x_i \le t \le x_{i+1}} \abs{f''(t)} \int_{x_i}^{x_{i+1}} \abs{  (n-1)(t-x_i) -  1_{[x,x_{i+1}]}(t)} \, \dif t \\
&=\sup_{x_i \le t \le x_{i+1}} \abs{f''(t)} \left \{\frac{1}{2(n-1)} - (x-x_i)[1-(n-1)(x-x_i)] \right\} \\
&\le \frac{1}{2(n-1)}\sup_{x_i \le t \le x_{i+1}} \abs{f''(t)}
\end{align*}
Furthermore, this inequality is tight if $f''$ is constant second derivative and $f'$ does not change sign over $[x_{i}, x_{i+1}]$, and $x=x_i$ or $x_{i+1}$.  Applying the above argument for $i=0, \ldots, n-1$ implies that
\[
\norm[\infty]{f'} - G_n(f) \le \frac{\norm[\infty]{f''}}{2(n-1)} \Rightarrow h_{+}(n)=\frac{1}{2(n-1)}
\]
with equality holding when has a constant second derivative and its first derivative does not change sign over $[0,1]$.
Then by \eqref{normdeflate} and \eqref{norminflate}, we can obtain
\begin{equation}\label{inflaapprox}
\fc_n=1, \ \   \fC_n =\frac{1}{1 - \tau /(2(n-1))}.
\end{equation}

Here we use divided difference. Denote
$$f[x,y]:=\frac{f(y)-f(x)}{y-x},f[x,y,z]:=\frac{f[y,z]-f[x,y]}{z-x}.$$
Note that for all $x,y \in [0,1]$ distinct, we have
 $$|f(y)-f(x)|=\left|\int_{x}^{y}f'(t)dt\right| \leq |y-x|\|f'\|_{\infty}.$$
 Therefore we have $|f[x,y]|\leq \|f'\|_{\infty}.$ 
For $x_{i-1} \leq x \leq x_i, \ i=1, \ldots, n-1,$ we  can rewrite
$A_{n}(f)(x)=f\left(x_{i-1}\right)+f\left[x_{i-1},x_{i}\right]\left(x-x_{i-1}\right).$
Then by Eq(2.67) in \cite{walter}, we have
$
f(x)-A_{n}(f)(x)
=f\left[x_{i-1},x_{i},x\right]\cdot (x-x_{i-1})(x-x_i).
$
As we can rewrite
$$|f\left[x_{i-1},x_{i},x\right]|=\left|\frac{f[x_{i},x]-f[x_{i-1},x]}{x_i-x_{i-1}}\right|\leq2(n-1)\norm[\infty]{f'}.$$
And as $x \in \left[x_{i-1},x_{i} \right],$ thus we have
$\left|(x-x_{i-1})(x-x_{i})\right| \leq \frac{1}{4(n-1)^2}.$
Then,
$$
|f(x)-A_{n}(f)(x)|\leq \frac{1}{4(n-1)^2}\cdot 2(n-1)\norm[\infty]{f'}=\frac{1}{2(n-1)}\norm[\infty]{f'}.
$$
Similarly, we can obtain that for $x \in [0,1]$
$$\norm[\infty]{f-A_{n}(f)} \leq \frac{\|f'\|_{\infty}}{2(n-1)} \Rightarrow \err(A_n,\cg,\ch,S) \leq \frac{1}{2}(n-1)^{-1} \Rightarrow \tilde{h}(n)=\frac{1}{2}(n-1)^{-1}.$$

On the other hand, by Eq(2.68) in \cite{walter},
we know
$f\left[x_{i-1},x_{i},x\right]=\frac{1}{2}f''(\xi),$ where $\xi \in [x_{i-1},x_{i}].$
Then we know
$
|f\left[x_{i-1},x_{i},x\right] |\leq \frac{1}{2}\|f''\|_{\infty}.$
And as $x \in \left[x_{i-1},x_{i} \right],$ thus we have
$\left|(x-x_{i-1})(x-x_{i})\right| \leq \frac{1}{4(n-1)^2}.$
Then,
$$
|f(x)-A_{n}(f)(x)|\leq \frac{1}{4(n-1)^2}\frac{\norm[\infty]{f''}}{2}=\frac{1}{8(n-1)^2}\norm[\infty]{f''}.
$$
Hence, we can obtain that for $\forall x \in [0,1]$
$$\norm[\infty]{f-A_{n}(f)} \leq  \frac{1}{8(n-1)^{2}}\norm[\infty]{f''} \Rightarrow \err(A_n,\cf,\ch,S) \leq \frac{1}{8}(n-1)^{-2}
\Rightarrow h(n)=\frac{1}{8}(n-1)^{-2}.$$


%Combine eq.(\ref{errorg}) and eq.(\ref{errorf}), we have the same result as in eq. (\ref{Anerrbound}).
%$$\norm[\ch]{S(f) -  A_n(f)} \le \frac{\min(C_1,C_2\tau n^{p_1-p_2})\norm[\cg]{f}}{n^{p_1}} \qquad \forall f \in \cc_{\tau}.$$

%
%Then by Algorithm \ref{twostagedetalgo}, we have the following algorithm
%\begin{algo} \label{adpatalgo}
%{\bf (Automatic, Adaptive, Two-Stage, Deterministic).}
% Let $\varepsilon$ be a positive error tolerance, and let $N_{\max}$ be the maximum cost allowed.  Let $\tau$ be a fixed positive number and
% let $A_{n}$ is defined as above. Let $n_{0}$ be a number larger than $\tau+2.$ Given an input function $f$, do the following:
%
%\begin{description}
%\item[Stage 1.\ Estimate {$\norm[\infty]{f'}$}.] Compute $G_{n_{0}}(f)$ at a cost of $n_{0} = \cost(A_{n_{0}})$.   Define the inflation factor
%$$
%\fC =\frac{1}{1 - \tau (n_{0}-1)^{-s}} \ge 1.
%$$
%Then $\fC G_{n_{0}}(f)$ provides a reliable upper bound on $\norm[\infty]{f'}$.
%
%\item [Stage 2.\ Estimate {$f$}.] Choose the sample size need to approximate $f$, namely, check whether $N$ is large enough to satisfy the error tolerance, i.e.,
%\[
%\frac{\min(1,\tau (N-1)^{-s})\fC G_{n_{0}}(f)}{2N} \le \varepsilon.
%\]
%If $N \le N_{\max}-n_0$, then $f$ may be approximated within the desired error tolerance and within the cost budget.  Set the warning flag to false, $W=0$. Otherwise, recompute $N$ to be within budget, $N = \tN_{\max} := \max\{N \in \ci : N\le N_{\max} -  N_G\}$, and set the warning flag to true, $W=1$.  Compute $A_N(f)$ as the approximation to $f$.
%
%\end{description}
%
%\end{algo}

And we can find  $h(n) \leq \tilde{h}(n) h_{+}(n), \forall n \in \mathcal{I},$ then $ \tau h(n)< \tilde{h}(n) \ \forall n \in \mathcal{I}.$
Thus we only need to find $n$ such that  \[
\frac{\tau \fC_n G_{n}(f)}{8(n-1)^2} \le \varepsilon.
\]

If we consider about the embedded algorithms, by Algorithm $\ref{multistagealgo},$ we obtain
\begin{algo} \label{multistageapproalgo}
Let the error tolerance $\varepsilon$, the maximum cost budget $N_{\max}$, and the positive constant $\tau$ be as described in  \eqref{taudef}. Let the sequences of algorithms, $\{A_n\}_{n \in \ci}$ and  $\{G_n\}_{n \in \ci}$ be as described above.  Set $i=1$.  Let $n_1$ be the smallest number in $n \in \ci$ satisfying $\err_+(G_n,\cf,\norm[\cg]{\cdot},\reals_+) \le 1/\tau$. For any input function $f \in \cf$, do the following:
\begin{description}

\item [Stage 1. Estimate {$\norm[\infty]{f'} $}.]
Compute $G_{n_i}(f)$ in \eqref{estfirstderiv} and $\fC_n$ in \eqref{inflaapprox}.
\item [Stage 2. Check for Convergence.]
 Check whether $n_i$ is large enough to satisfy the error tolerance, i.e.,
$$
\frac{\tau \fC_{n_i} G_{n_i}(f)}{8(n_i-1)^2} \le \varepsilon.
$$
If this is true, then set $W$ to be false, return $(A_{n_i}(f),W)$ and terminate the algorithm.


\item[Stage 3. Compute $n_{i+1}$.]  Otherwise, if the inequality above fails to hold,
choose $n_{i+1}$ as the smallest number exceeding $n_i$ and not less than $N_{A}(\varepsilon /G_{n_i}(f))$ such that $A_{n_{i}}$ is embedded in $A_{n_{i+1}}$. That is
$$n_{i+1}-1=(n_i-1)\max\left\{\left\lceil\frac{1}{(n_i-1)}\left(\frac{\tau G_{n_i}(f)}{8\varepsilon}\right)^{\frac{1}{2}}\right\rceil,2\right\}.$$
 If $n_{i+1} \le N_{\max}$, increment $i$ by $1$, and return to Stage 1.\\
Otherwise, if $n_{i+1} > N_{\max}$, choose $n_{i+1}$ to be the largest number not exceeding $N_{\max}$ such that $A_{n_{i}}$ is embedded in $A_{n_{i+1}}$, and set $W$ to be true. Return $(A_{n_{i+1}}(f),W)$ and terminate the algorithm.
\end{description}
\end{algo}


Then we can have the complexity of the upper bound.
\begin{theorem}   Let  $\varepsilon$, $N_{\max}$, $\tau$ and $n_1$ be given as described in Algorithm \ref{multistageapproalgo}.
Assume that $n_1 \le N_{\max}$. Let $r$ be the number described in the paragraph preceding that algorithm.
Let $\cc_\tau$ be the cone of functions defined in \eqref{taudef}.
Define
\[
\tN_A(a) = \min\left\{ n \in \ci : \frac{\tau \fC_n }{8(n-1)^2} \le a \right\}, \quad a \in (0,\infty).
\]
Let $\cc_\tau$ be the cone of functions defined in \eqref{taudef}.  Let
$$
\cn = \left \{ f \in \cc_\tau :r\tN_{A}\left(\frac{\varepsilon}{\sigma} \right) \le N_{\max} \right\}
= \left \{ f \in \cc_\tau : \sigma \le \frac{8(N_{\max}/r-1)^2 \varepsilon }{\tau \fC_{N_{\max}/r}  } \right\}
$$
be the nice subset of the cone $\cc_\tau$.  Then it follows that Algorithm \ref{multistageapproalgo} is successful for all functions in $\cn$,  i.e.,  $\success(A,W,\cn,\varepsilon,N_{\max}) = 1$.  Moreover, the cost of this algorithm is bounded above in terms of the $\norm[\infty]{f'}$ of the input function as follows:
\begin{multline}
\cost(A,\cn,\varepsilon,N_{\max},\sigma) \\
\le  \min\left\{ n \in \ci : n \ge \left(\frac{\sigma\tau}{8\varepsilon(1 - \tau/(2(n_{1}-1)))}\right)^{\frac{1}{2}} +1\right\}
\end{multline} where $\sigma=\norm[\infty]{f'}.$
%The upper bound on the cost of this specific algorithm provides an upper bound on the complexity of the problem, $\comp(\varepsilon,\ca(\cn,\ch,S,\Lambda),N_{\max},\sigma)$.  If the sequence of algorithms $\{A_n\}_{n \in \ci}$, $A_n \in\ca_{\fix}(\cg,\ch,S,\Lambda)$  is nearly optimal for the problems $(\cg,\ch,S,\Lambda)$ and $(\cf,\ch,S,\Lambda)$ as defined in \eqref{nearoptdef}, then Algorithm \ref{twostagedetalgo} does not incur a significant penalty for not knowing $\norm[\cg]{f}$ a priori, i.e., for all $p>0$,
%\begin{equation} \label{penalty}
%\sup_{0 < \varepsilon/\sigma \le 1} \frac{\cost(A,\cn,\varepsilon,\infty,\sigma)} {\comp(\varepsilon/\sigma,\ca_{\fix}(\cj,\ch,S,\Lambda))} \left(\frac{\varepsilon}{\sigma}\right)^p <\infty, \qquad \cj \in \{\cf,\cg\}.
%\end{equation}
\end{theorem}

\begin{proof}
Applied Theorem \ref{MultiStageThm}, then we can have the above result.
\end{proof}




\subsection{Lower Bound}
Next, we will derive the lower bound. Suppose that for any $n>0,$ for any algorithm which cost is less than $n.$
We want to build fooling function at all data sites $\xi_{i}, \ i=0,\ldots,n-1.$
There exists at least one $i$ such that $\xi_{i+1}-\xi_{i}\ge\frac{1}{n-1}.$
Then we can define $f_{1}$
$$f_{1}(x):=\left\{\begin{matrix}
A_{1}(x-\xi_{j})^{2}(x-\xi_{j+1})^{2} & \xi_{j}< x \leq \xi_{j+1},\\
0 & \text{otherwise},
\end{matrix}\right.,$$
where $A_{1} \ge 0.$ As we need $\norm[\infty]{f_1}=1,$ we can have $A_{1}=\frac{16}{(\xi_{i+1}-\xi_{i})^4}.$
Then we can obtain
\begin{align*}
\norm[\infty]{f'_1}&=\frac{16}{3\sqrt{3}(\xi_{j+1}-\xi_{j})}\leq \frac{16}{3\sqrt{3}}(n-1) \Rightarrow g(n)=\frac{16}{3\sqrt{3}}(n-1).\\
\norm[\infty]{f''_1}&=\frac{16}{(\xi_{j+1}-\xi_{j})^2}
=\frac{3\sqrt{3}}{(\xi_{j+1}-\xi_{j})}\norm[\infty]{f'_1}
 \leq 3\sqrt{3}(n-1)\norm[\infty]{f'_1}\Rightarrow \tg(n)=3\sqrt{3}(n-1).
\end{align*}

Let $f_0=x.$ Then we have $\norm[\infty]{f'_0}=1, \norm[\infty]{f''_0}=0 \Rightarrow \tau_0=0.$

\begin{theorem} \label{complowbdappr} Suppose that functions $f_{0}$ and $f_1$ can be found that satisfy conditions \eqref{assumpfone} and \eqref{assumpfzero}.  It then follows that the complexity of the problem, defined by \eqref{complexdef}, assuming infinite cost budget, over the cone of functions $\cc_{\tau}$ is
$$
\comp(\varepsilon,\ca(\cc_{\tau},\ch,S,\Lambda),\infty,\sigma)
\ge \min\left(\frac{3\sqrt{3}\sigma}{64 \varepsilon}+1, \left(\frac{\sigma\tau}{64\varepsilon}\right)^{2}+1 \right).
$$
\end{theorem}

\begin{proof}
$g(n), \ \tg(n)$ and $\tau_0$ is obtained above, then by Theorem \ref{complowbd}
we can have the above results.
\end{proof}

\subsection{Numerical Example}

We want to testify the success rate for our algorithm. So we
choose test function $f(x)=e^{-(a(x-\mu))^2},$
where $\mu \sim U[0,1]$ and $\log_{10} a \sim U[0,4].$
And $\mu \sim U[0,1]$ means $\mu$ is uniform on [0,1].
Also for this kind of function, if we have large $a$, we can approximate the function hardly
because the function has a sharper bump. And we know
$$\norm[\infty]{f'}=\sqrt{\frac{2}{e}} a, \ \norm[\infty]{f''}=2a^2.$$
If $f \in \cc_{\tau}$ then we should have
$$a \leq \frac{\tau}{\sqrt{2e}} \Rightarrow \text{Expected Success Rate }= \frac{1}{4}\ln\left(\frac{\tau}{\sqrt{2}}\right) -\frac{1}{8}.$$

We use $\tau = 10, 25 , 100$ separately to test $10000$ times with
$\varepsilon = 1e-7$ to obtain the observed success rate.
\begin{table}[h]
\centering
\begin{tabular}{cccc}
$\tau$ &  10 & 25 & 100\\
\toprule 
Expected Success Rate &  $\leq 16 \%$ &  $\leq 26 \%$  & $\leq 41 \%$ \\
Observed Success Rate & 25$\%$ &  34$\%$  & 49$\%$ \\
\end{tabular}
\caption{ Comparison between expected and observed success rate}
\end{table}

From the above table, we can find our algorithm works very well, as the observed
success rate is always larger than the expected success rate.\\






\section{Addressing Concerns About Automatic Algorithms} \label{overcomesec}

Automatic algorithms are popular, especially for univariate integration problems.  As mentioned in Section \ref{integsec}, MATLAB \cite{TrefEtal12,MAT7.12}, Mathematica \cite{Mat8a}, and the NAG \cite{NAG23} library all have one or more automatic integration routines.  In spite of this popularity certain concerns or even objections have been raised about automatic algorithms.  This section addresses those concerns by showing that the assumptions made in this article violate the assumptions on which those concerns are based.

\subsection{Concerns Raised by Lyness}

In his provocatively titled SIAM Review article, \emph{When Not to Use an Automatic Quadrature Routine} \cite[p.\ 69]{Lyn83}, James Lyness makes the following claim.
\begin{quote}
While prepared to take the risk of being misled by chance alignment of zeros in the integrand function, or by narrow peaks which are ``missed,'' the user may wish to be reassured that for ``reasonable'' integrand functions which do not have these characteristics all will be well. It is the purpose of the rest of this section to demonstrate by example that he cannot be reassured on this point. In fact the routine is likely to be unreliable in a significant proportion of the problems it faces (say $1$ to $5\%$) and there is no way of predicting in a straightforward way in which of any set of apparently reasonable problems this will happen.
\end{quote}

The following is a summary of Lyness's argument using the notation of the present article. Lyness's automatic algorithm consists of a sequence of non-adaptive algorithms $\{A_{n_i}\}_{i=1}^{\infty}$, and a stopping rule that returns $A_{n_{i}}(f)$ as the answer for the first $i$ where $A_{n_{i}}(f)-A_{n_{i-1}}(f)$ is small enough.  To fool this automatic algorithm one constructs an integrand $f_\lambda$, which is parameterized by $\lambda$, such that 
\begin{itemize}
\item $f_\lambda$ is ``reasonable'' for all $\lambda \in [0,1]$,
\item $A_n(f_\lambda)$ is continuous in $\lambda$, and 
\item for some $i$ with moderate $n_i$, $A_{n_{i}}(f_\lambda)-A_{n_{i-1}}(f_\lambda)$ has different signs for $\lambda=0,1$.  
\end{itemize}
It then follows that $A_{n_{i}}(f_{\lambda_*})-A_{n_{i-1}}(f_{\lambda_*})=0$ for some $\lambda_* \in [0,1]$.  Then this algorithm will likely fail for integrands $f_{\lambda}$ with $\lambda$ nearby $\lambda_*$ since the stopping criterion is satisfied, but $n_i$ is not large enough to satisfy the desired error tolerance.

Lyness's model of automatic quadrature algorithms describes many, perhaps most, existing  ones.  His argument that any such algorithm will fail for certain ``reasonable'' integrands is correct.  Algorithm \ref{??} presented in Section \ref{integsec}, however, never fails for ``reasonable'' integrands because the stopping criterion used here is not of the form that Lyness assumes.  The stopping criteria used in Algorithm \ref{??}, as well as for the general Algorithms \ref{twostagedetalgo} and \ref{multistagealgo} are based on estimates of the norms of the input functions, not on whether two different algorithms give the same answer. 

Although he means it as such, Lyness's warning in \cite{Lyn83} should not be interpreted as an objection to automatic algorithms.  It should be an objection to stopping criteria that are based on the value of a single functional, in this case, $A_{n_{i}}-A_{n_{i-1}}$.  What Lyness clearly demonstrates is that it one may easily make a functional vanish even though the error of the algorithm is significant.

\subsection{No Advantage in Adaption}

There are results dating to ???? stating that 

\subsection{Why Cones?}

There are good reasons why the error analysis for automatic algorithms is performed for input functions lying in cones, $\cc_{\tau}$, of the form \eqref{conedef}.  Automatic algorithms proceed by estimating, perhaps conservatively, the error for a particular approximation and then increasing the number of function data until that error estimate does not exceed the error tolerance.  These error estimates are constructed in such a way that if they are reliable for $f$, then they are also reliable for $cf$, where $c$ is any real number.  Thus, if an algorithm is successful for the input function $f$, then it is also successful for the input $cf$, where $c$ is any real number.  Any set that contains $cf$ if it contains $f$ is by definition a cone.  

One might wonder whether the definition of the cone $\cc_{\tau}$ in \eqref{conedef} is too strict.  Ignoring the cost budget, an alternative could be to define $\cc_{\text{success}}$ as the cone of all input functions for which the automatic algorithm successfully approximates the solution within the error tolerance.  Clearly $\cc_{\text{success}}$ contains $\cc_{\tau}$, and quite likely $\cc_{\text{success}}$ also contains functions not in $\cc_{\tau}$.  However, the definition of $\cc_{\text{success}}$ does not provide any insight into the  kind of input functions that the automatic algorithm can successfully handle.  Moreover, given the lower bounds that one may derive, it can be shown in many cases, including those in Sections \ref{integsec} and \ref{approxsec}, that the cost of the automatic algorithm is within a constant multiple of the cost of the best algorithm for $\cc_{\tau}$.

How to choose $\tau$

\section{Further Work} \label{furthersec}





\bibliographystyle{model1b-num-names.bst}
\bibliography{FJH22,FJHown22}
\end{document}

