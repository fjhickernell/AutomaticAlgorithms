The algorithms used in this section on integration and the next section on function recovery are all based on linear splines on $[0,1]$.  The node set and the linear spline algorithm using $n$ function values are defined for $n \in \mathcal{I}=\{2,3,\ldots\}$ as follows:
\begin{subequations} \label{linearspline}
\begin{equation}
x_i=\frac{i-1}{n-1}, \qquad i=1, \ldots, n,
\end{equation}
\begin{multline}
A_{n}(f)(x)=(n-1) \left[ f(x_{i})(x_{i+1}-x) +f(x_{i+1})(x-x_i) \right] \\ \text{for }x_i \leq x \leq x_{i+1}.
\end{multline}
\end{subequations}
The cost of each function value is one and so the cost of  $A_n$ is $n$. The algorithm $A_n$ is imbedded in the algorithm $A_{2n-1}$, which uses $2n-2$ subintervals.  Thus, $r=2$ is the minimum cost multiple as described in Section \ref{adapintrosec}.

The problem to be solved is univariate integration on the unit interval, $S(f)=\INT(f):=\int_{0}^{1}f(x) \, \dif x \in \cg := \reals$.  The fixed cost building blocks to construct the adaptive integration algorithm are the composite trapezoidal rules based on $n-1$ trapezoids:
\begin{equation*}
    T_{n}(f) = \int_0^1 A_n(f) \, \dif x
    =\frac{1}{2n-2}[f(x_1)+2f(x_2)+\cdots+2f(x_{n-1})+f(x_n)].
\end{equation*}

The space of input functions is $\cf=\mathcal{V}^{1}$, the space of functions whose first derivatives have finite variation.  The general definition of these relevant norms and spaces are as follows:
\begin{subequations} \label{defSobolev}
\begin{gather}
\Var(f) := \sup_{\substack{n \in \naturals\\ 0 = x_0 < x_1 < \cdots < x_{n} =1}} \sum_{i=1}^n \abs{f(x_i)-f(x_{i-1})}, \\
\norm[p]{f}:= \begin{cases} \displaystyle \left[\int_0^1 \abs{f(x)}^p \, \dif x \right]^{1/p}, & 1 \le p < \infty,\\[1ex]
\displaystyle  \sup_{0 \le x \le 1} \abs{f(x)}, & p=\infty,
\end{cases}
\\
\cv^{k}: =\cv^{k}[0,1]=\{f\in C[0,1]: \Var(f^{(k)}) < \infty \}, \\
\mathcal{W}^{k,p}=\mathcal{W}^{k,p}[0,1]=\{f\in C[0,1]: \|f^{(k)}\|_{p}<\infty\}.
\end{gather}
\end{subequations}
The stronger semi-norm is $\Fnorm{f}:=\Var(f')$, while the weaker semi-norm is
\[
\Ftnorm{f}:=\norm[1]{f'-A_2(f)'}=\norm[1]{f'-f(1)+f(0)}=\Var(f-A_2(f)),
\]
where that $A_2(f): x \mapsto f(0)(1-x)+f(1)x$ is the linear interpolant of $f$ using the two endpoints of the integration interval. The reason for defining $\Ftnorm{f}$ this way is that $\Ftnorm{f}$ vanishes if $f$ is a linear function, and linear functions are integrated exactly by the trapezoidal rule.  The cone of integrands is defined as
\begin{equation}\label{coneinteg}
\cc_{\tau}=\{f\in \cv^{1}:\Var(f')\leq\tau\|f'-f(1)+f(0)\|_1\}.
\end{equation}

The algorithm for approximating $\norm[1]{f'-f(1)+f(0)}$ is the $\tcf$-semi-norm of the linear spline, $A_n(f)$:
\begin{multline}\label{1direst}
    \tF_n(f):=\Ftnorm{A_n(f)}=\bignorm[1]{A_n(f)'-A_2(f)'} \\
    =\sum_{i=1}^{n-1}\left|f(x_{i+1})-f(x_{i}) - \frac{f(1)-f(0)}{n-1}\right|.
\end{multline}
The variation of the first derivative of the linear spline of $f$, i.e.,
\begin{equation} \label{Fnormalg}
F_n(f) :=\Var(A_n(f)') = (n-1)\sum_{i=1}^{n-2} \bigabs{f(x_i) - 2 f(x_{i+1})+f(x_{i+2})},
\end{equation}
provides a lower bound on $\Var(f')$ for $n \ge 3$, and can be used in the necessary condition that $f$ lies in $\cc_\tau$ as described in Remark \ref{neccondrem}.
The mean value theorem implies that
\begin{align*}
F_n(f) &= (n-1)\sum_{i=1}^{n-1} \bigabs{[f(x_{i+2}) - f(x_{i+1})] - [f(x_{i+1}) - f(x_{i})]} \\
&= \sum_{i=1}^{n-1} \abs{f'(\xi_{i+1}) - f'(\xi_{i})} \le \Var(f'),
\end{align*}
where $\xi_i$ is some point in $[x_i,x_{i+1}]$.


\subsection{Adaptive Algorithm and Upper Bound on the Cost}

Constructing the adaptive algorithm for integration requires an upper bound on the error of $T_n$ and a two-sided bound on the error of $\tF_n$.  Note that $\tF_{n}(f)$ never overestimates $\Ftnorm{f}$ because
\begin{align*}
\Ftnorm{f} & = \bignorm[1]{f'-A_2(f)'}
= \sum_{i=1}^{n-1} \int_{x_i}^{x_{i+1}} \abs{f'(x) - A_2(f)'(x)} \, \dif x \\
& \ge \sum_{i=1}^{n-1} \abs{\int_{x_i}^{x_{i+1}} [f'(x) - A_2(f)'(x)] \, \dif x}=\norm[1]{A_n(f)'-A_2(f)'} = \tF_n(f).
\end{align*}
Thus, $h_{-}(n)=0$ and $\fc_n=\tfc_n=1$.

To find an upper bound on $\Ftnorm{f}-\tF_{n}(f)$, note that
\begin{equation*}
\Ftnorm{f} - \tF_{n}(f) = \Ftnorm{f} - \bigabs{A_n(f)}_{\tcf} \le \bigabs{f-A_n(f)}_{\tcf} = \bignorm[1]{f' -A_n(f)'},
\end{equation*}
since $(f-A_n(f))(x)$ vanishes for $x=0,1$.  Moreover,
\begin{equation} \label{onenormfp}
\bignorm[1]{f' -A_n(f)'} = \sum_{i=1}^{n-1} \int_{x_i}^{x_{i+1}} \abs{f'(x) -(n-1)[f(x_{i+1})-f(x_i)]} \, \dif x.
\end{equation}
Now we bound each integral in the summation.  For $i=1, \ldots, n-1$, let $\eta_i(x) = f'(x) -(n-1)[f(x_{i+1})-f(x_i)]$, and let $p_i$ denote the probability that $\eta_i(x)$ is non-negative:
\[
p_i = (n-1)\int_{x_i}^{x_{i+1}} \bbone_{[0,\infty)} (\eta_i(x)) \, \dif x,
\]
and so $1-p_i$ is the probability that $\eta_i(x)$ is negative.  Since $\int_{x_i}^{x_{i+1}} \eta_i(x) \, \dif x =0$, we know that $\eta_i$ must take on both non-positive and non-negative values.  Invoking the mean value theorem, it follows that
\begin{multline*}
\frac{p_i}{n-1} \sup_{x_i \le x \le x_{i+1}} \eta_i(x) \ge \int_{x_i}^{x_{i+1}} \max(\eta_i(x),0) \, \dif x \\
= \int_{x_i}^{x_{i+1}} \max(-\eta_i(x),0) \, \dif x \le \frac{-(1-p_i)}{n-1} \inf_{x_i \le x \le x_{i+1}} \eta_i(x) .
\end{multline*}
These bounds allow us to derive bounds on the integrals in \eqref{onenormfp}:
\begin{align*}
\MoveEqLeft{\int_{x_i}^{x_{i+1}} \abs{\eta_i(x)} \, \dif x} \\
 &= \int_{x_i}^{x_{i+1}} \max(\eta_i(x),0) \, \dif x + \int_{x_i}^{x_{i+1}} \max(-\eta_i(x),0) \, \dif x\\
&=2(1-p_i) \int_{x_i}^{x_{i+1}} \max(\eta_i(x),0) \, \dif x + 2p_i\int_{x_i}^{x_{i+1}} \max(-\eta_i(x),0) \, \dif x\\
&\le \frac{2p_i(1-p_i)}{n-1} \left[ \sup_{x_i \le x \le x_{i+1}} \eta_i(x) - \inf_{x_i \le x \le x_{i+1}} \eta_i(x) \right]\\
&\le\frac{1}{2(n-1)} \left[ \sup_{x_i \le x \le x_{i+1}} f'(x) - \inf_{x_i \le x \le x_{i+1}} f'(x) \right],
\end{align*}
since $p_i(1-p_i)\le 1/4$.

Plugging this bound into \eqref{onenormfp} yields
\begin{align*}
\bignorm[1]{f'-f(1)+f(0)} - \tF_n(f) &= \Ftnorm{f} - \tF_{n}(f)\\
 & \le \bignorm[1]{f' -A_n(f)'}\\
&\le \frac{1}{2n-2} \sum_{i=1}^{n-1} \left[ \sup_{x_i \le x \le x_{i+1}} f'(x) - \inf_{x_i \le x \le x_{i+1}} f'(x) \right] \\
& \le \frac{\Var(f')}{2n-2} = \frac{\Fnorm{f}}{2n-2},
\end{align*}
and so
\begin{equation*}\label{factor}
h_{+}(n)= \frac{1}{2n-2}, \qquad \mathfrak{C}_n =\frac{1}{1 - \tau/(2n-2)} \qquad \text{for } n>1+\tau/2.
\end{equation*}
Since $\tF_2(f)=0$ by definition, the above inequality for $\Ftnorm{f} - \tF_{2}(f)$ implies that
\begin{equation*} \label{taumininteg}
2\bignorm[1]{f'-f(1)+f(0)} = 2 \Ftnorm{f} \le \Fnorm{f} = \Var(f'), \qquad \tau_{\min}=2.
\end{equation*}

The errors of the trapezoidal rule in terms of the variation and the $\cl_1$-norm of the first derivative of the integrand are given in \cite[(7.15)]{BraPet11a}:
%\begin{subequations} \label{integhhtilde}
\begin{gather*}
\abs{\int_0^1 f(x) \, dx - T_n(f)} \le h(n) \Var(f') \\
h(n)= \frac{1}{8(n-1)^2}, \qquad h^{-1}(\varepsilon) = \left \lceil \sqrt{\frac{1}{8\varepsilon}} \right \rceil +1.
\end{gather*}
%\end{subequations}

Given the above definitions of $h, \fC_n, \fc_n$, and $\tfc_n$, it is now possible to also specify
\begin{subequations} \label{simplifycond}
\begin{gather}
h_1(n) = h_2(n) = \fC_n h(n) = \frac{1}{4(n-1)(2n-2-\tau)}, \\
h_1^{-1}(\varepsilon) = h_2^{-1}(\varepsilon) = 1+ \left \lceil \sqrt{\frac{\tau}{8 \varepsilon} + \frac{\tau^2}{16}} +\frac{\tau}{4} \right \rceil \le 2 + \frac{\tau}{2} + \sqrt{\frac{\tau}{8\varepsilon}}.
\end{gather}
Moreover, the left side of \eqref{multistageconv}, the stoping criterion inequality in the multi-stage algorithm, becomes
\begin{equation}
\tau h(n_i)\fC_{n_i} \tF_{n_i}(f) = \frac{\tau  \tF_{n_i}(f) } {4(n_i-1)(2n_i-2 -\tau)}.
\end{equation}
\end{subequations}
With these preliminaries, Algorithm \ref{multistagealgo} and Theorem \ref{MultiStageThm} may be applied directly to  yield the following adaptive integration algorithm and its guarantee.

\begin{algo}[Adaptive Univariate Integration] \label{multistageintegalgo}
Let the sequence of algorithms $\{A_n\}_{n\in \mathcal{I}}$, $\{\tF_n\}_{n\in \mathcal{I}}$, and $\{F_n\}_{n\in \mathcal{I}}$ be as described above.
Let $\tau\ge2$ be the cone constant. Set $i=1$. Let $n_1=\lceil(\tau+1)/2\rceil+1$. For any error tolerance $\varepsilon$ and input function $f$, do the following:
\begin{description}
\item[Stage 1.\ Estimate {$\norm[1]{f'-f(1)+f(0)}$} and bound {$\Var(f')$}.] Compute $\tF_{n_i}(f)$ in \eqref{1direst} and $F_{n_i}(f)$ in \eqref{Fnormalg}.

\item[Stage 2. Check the necessary condition for $f \in \cc_{\tau}$.] Compute
    \begin{align*}
     \tau_{\min,n_i} =  \frac{F_{n_i}(f)}{\tF_{n_i}(f)+F_{n_i}(f)/(2n_i-2)}.
    \end{align*}
If $\tau \ge \tau_{\min,n_i}$, then go to stage 3.  Otherwise, set $\tau = 2\tau_{\min,n_i}$.  If $n_i \ge (\tau+1)/2$, then go to stage 3.  Otherwise, choose
$$
n_{i+1}=1+ (n_i-1)\left\lceil\frac{\tau+1}{2n_i-2}\right\rceil.
$$
Go to Stage 1.

\item[Stage 3. Check for convergence.] Check whether $n_i$ is large enough to satisfy the error tolerance, i.e.
    \begin{equation*}
     \tF_{n_i}(f) \le \frac{4\varepsilon(n_i-1)(2n_i-2 - \tau)}{\tau}.
    \end{equation*}
If this is true, then return $T_{n_i}(f)$ and terminate the algorithm.   If this is not true, choose
$$
n_{i+1}=1+ (n_i-1)\max\left\{2,\left\lceil\frac{1}{(n_i-1)}\sqrt{\frac{\tau \tF_{n_i}(f)}{8\varepsilon}}\right\rceil\right\}.
$$
Go to Stage 1.
\end{description}
\end{algo}

\begin{theorem} \label{multistageintegthm}
Let $\sigma >0$ be some fixed parameter, and let $\cb_{\sigma}=\{f \in  \mathcal{V}^{1} : \Var(f')\leq \sigma\}$. Let $T \in \ca(\cb_{\sigma}, \reals, \INT, \Lambda^{\std})$ be the non-adaptive trapezoidal rule defined by Algorithm \ref{nonadaptalgo}, and let $\varepsilon>0$ be the error tolerance. Then this algorithm succeeds for $f \in \cb_{\sigma}$, i.e., $\abs{\INT(f) - T(f,\varepsilon)} \le \varepsilon$, and the cost of this algorithm is $\left \lceil \sqrt{\sigma/(8\varepsilon)}\right \rceil + 1$, regardless of the size of $\Var(f')$.

Let $T \in \ca(\cc_{\tau}, \reals, \INT, \Lambda^{\std})$ be the adaptive trapezoidal rule defined by Algorithm \ref{multistageintegalgo}, and let $\tau$, $n_1$, and $\varepsilon$ be the inputs and parameters described there. Let $\cc_\tau$ be the cone of functions defined in \eqref{coneinteg}.  Then it follows that Algorithm \ref{multistageintegalgo} is successful for all functions in $\cc_{\tau}$,  i.e.,  $\abs{\INT(f) - T(f,\varepsilon)} \le \varepsilon$.  Moreover, the cost of this algorithm is bounded below and above as follows:
\begin{multline}
\max \left(\left \lceil\frac{\tau+1}{2} \right \rceil, \left \lceil \sqrt{\frac{ \Var(f')}{8\varepsilon}} \right \rceil \right) +1 \\
\le \max \left(\left \lceil\frac{\tau+1}{2} \right \rceil, \left \lceil \sqrt{\frac{\tau \norm[1]{f'-f(1)+f(0)}}{8\varepsilon}} \right \rceil \right) +1 \\
\le
\cost(T,f;\varepsilon,N_{\max}) \\
\le \sqrt{\frac{\tau \norm[1]{f'-f(1)+f(0)}}{2\varepsilon}} + \tau + 4
\le \sqrt{\frac{\tau \Var(f') }{4\varepsilon}} + \tau + 4.
\end{multline}
The algorithm is computationally stable, meaning that the minimum and maximum costs for all integrands, $f$, with fixed $\norm[1]{f'-f(1)+f(0)}$ or $\Var(f')$ are an $\varepsilon$-independent constant of each other.
\end{theorem}


\subsection{Lower Bound on the Computational Cost}
Next, we derive a lower bound on the cost of approximating functions in the ball $\cb_{\tau}$ and in the cone $\cc_{\tau}$ by constructing fooling functions. Following the arguments of Section \ref{LowBoundSec}, we choose  the triangle shaped function $f_0: x \mapsto 1/2-\abs{1/2-x}$. Then
\begin{gather*}
\Ftnorm{f_0}=\norm[1]{f'_0-f_0(1)+f_0(0)}=\int_0^1 \abs{\sign(1/2-x)} \, \dif x = 1, \\ \Fnorm{f_0}=\Var(f'_0)=2= \tau_{\min}.
\end{gather*}
For any $n \in \natzero$, suppose that the one has the data $L_i(f)=f(\xi_i)$, $i=1, \ldots, n$ for arbitrary $\xi_i$, where $0=\xi_0 \le \xi_1 < \cdots < \xi_n \le \xi_{n+1} = 1$.  There must be some $j=0, \ldots, n$ such that $\xi_{j+1} - \xi_j \ge 1/(n+1)$.  The function $f_{1}$ is defined as a triangle function on the interval $[\xi_j, \xi_{j+1}]$:
$$
f_{1}(x):=\begin{cases} \displaystyle
\frac{\xi_{j+1}-\xi_{j}-\abs{\xi_{j+1}+\xi_{j}-2x}}{8} & \xi_{j} \le x \leq \xi_{j+1},\\
0 & \text{otherwise},
\end{cases}
$$
This is a piecewise linear function whose derivative changes from $0$ to $1/4$ to $-1/4$ to $0$ provided $0 < j < n$, and so $\Fnorm{f_1}=\Var(f'_1)\le 1$. Moreover,
\begin{gather*}
\INT(f)=\int_0^1 f_1(x) \, \dif x = \frac{(\xi_{j+1} - \xi_j)^2}{16} \ge \frac{1}{16(n+1)^2} =: g(n),\\
g^{-1}(\varepsilon)=\left \lceil \sqrt{\frac{1}{16 \varepsilon}} \right \rceil - 1.
\end{gather*}
Using these choices of $f_0$ and $f_1$, along with the corresponding $g$ above, one may invoke Theorems \ref{complowbdball}--\ref{complowbd}, and Corollary \ref{optimcor} to obtain the following theorem.

\begin{theorem} \label{complowbdinteg} For $\sigma>0$ let $\cb_{\sigma}=\{f \in \cv^{1} : \Var(f') \le \sigma\}$.  The complexity of integration on this ball is bounded below as
\begin{equation*}
\comp(\varepsilon,\ca(\cb_{\sigma},\reals,\INT,\Lambda^{\std}),\cb_{s}) \ge \left \lceil \sqrt{\frac{\min(s,\sigma)}{16 \varepsilon}} \right \rceil -1 .
\end{equation*}
Algorithm \ref{nonadaptalgo} using the trapezoidal rule is optimal in the sense of Theorem \ref{optimalprop}.

For $\tau>2$, the complexity of the integration problem over the cone of functions $\cc_{\tau}$ defined in \eqref{coneinteg} is bounded below as
\begin{equation*}
\comp(\varepsilon,\ca(\cc_{\tau},\reals,\INT,\Lambda^{\std}),\cb_{s}) \ge \left \lceil \sqrt{\frac{(\tau-2)s}{32 \tau \varepsilon}} \right \rceil -1 .
\end{equation*}
The adaptive trapezoidal Algorithm \ref{multistageintegalgo} is optimal for integration of functions in $\cc_{\tau}$ in the sense of Corollary \ref{optimcor}.
\end{theorem}

\subsection{Numerical Example}
{\bf FIX this subsection with new $\tcf$- and $\cf$-semi-norms, $A_n$, $T_n$}

Consider the family of test functions defined by
\begin{equation}\label{testfun}
f(x)=
\begin{cases}
\displaystyle  b[(x-z)^2-(x-z-a)|x-z-a|-(x-z+a)|x-z+a|+4a^2], & z-2a\leq x\leq z+2a,\\[2ex]
\displaystyle  0, & \text{otherwise}.
\end{cases}.
\end{equation}
with  $\log_{10}(a) \sim \cu[-4,-1]$, $z \sim \cu[2a,1-2a]$, and $b=4/a^3$ chosen to make $\int_0^1 f(x) \, \dif x = 1$.
Straightforward calculations yield:
\begin{gather*}
\norm[1]{f'-f(1)+f(0)}=1/a, \\
\Var(f')=2/a^2.
\end{gather*}
The probability that $f \in \cc_{\tau}$ is $\min\left(1,\max(0,\left(\log_{10}(\tau/2)-1\right)/3)\right).$
For initial $\tau$ values of $10, 100 , 1000$ we choose a sample of  $10000$ functions and an error tolerance of  $\varepsilon = 10^{-8}$.  The probability that $f$ initially lies in $\cc_{\tau}$ is the smaller number in the second column of Table \ref{integresultstable}, while the larger number is the empirical probability that $f$ eventually lies in $\cc_{\tau}$ after possible increases in $\tau$ made by Stage 2 of Algorithm \ref{multistageintegalgo}.  Algorithm \ref{multistageintegalgo} yields the correct value to within the error tolerance for all $f$ that finally lie inside $\cc_{\tau}$ plus some functions that lie outside $\cc_{\tau}$.  The empirical probability of success is given in Table
\begin{table}[h]
\centering
\begin{tabular}{cccccc}
&&&\multicolumn{2}{c}{Success} & Failure \\
& $\tau$ &  $\Prob(f \in \cc_{\tau}) $ & No Warning & Warning & No Warning \\
\toprule
&$10$ & $0\% \rightarrow  25\% $ & $25\%$ & $0\%$ & $75\%$  \\
Algorithm \ref{multistageintegalgo}
 &$100$ & $23 \% \rightarrow 58\% $ & $56\%$ & $0\%$ & $42\%$ \\
&$1000$ & $57\% \rightarrow 88\% $& $68\%$ & $5\%$ &$11\%$ \\
\midrule
{\tt quad} & & & 8\% & & $92\%$\\
{\tt quadgk} & & & 19\% & & $81\%$\\
{\tt chebfun} & & &29\% & & $71\%$\\
\end{tabular}
\caption{Performance of multi-stage, adaptive Algorithm \ref{multistageintegalgo} for various values of $\tau$, and also of other automatic quadrature algorithms. The test function \eqref{testfun} with random parameters, $a$ and $z$, was used. \label{integresultstable}}
\end{table}

Ten thousand random instances of this test function are integrated by  Algorithm \ref{multistageintegalgo} and three existing automatic algorithms with an absolute error tolerance of $\varepsilon=10^{-8}$.  The observed success rates for these algorithms are shown.  As expected, the success rate of {Algorithm \ref{multistageintegalgo} increases as $\tau$ is increased.  All of the integrands lying inside $\cc_{\tau}$ are integrated to within the error tolerance, and a significant number of integrands lying outside the cone are integrated accurately as well.

