\documentclass[12pt]{amsart}
\setlength{\marginparwidth}{0.5in}
\usepackage{amsmath,amssymb,natbib,graphicx}
\textheight 9in
\textwidth 6.5 in
\hoffset -1 in
\voffset -1 in
\input FJHDef.tex

\newcommand{\sphere}{\mathbb{S}}
\newcommand{\cc}{\mathcal{C}}
\newcommand{\cq}{\mathcal{Q}}
\newcommand{\bbW}{\mathbb{W}}
\newcommand{\tP}{\widetilde{P}}
\newcommand{\bg}{{\bf g}}
\newcommand{\bu}{{\bf u}}
\newcommand{\bbu}{\bar{\bf u}}
\newcommand{\bv}{{\bf v}}
\newcommand{\bbv}{\bar{\bf v}}
\newcommand{\bw}{{\bf w}}
\newcommand{\bbw}{\bar{\bf w}}
\newcommand{\hv}{\hat{v}}
\DeclareMathOperator{\MSE}{MSE}
\DeclareMathOperator{\RMSE}{RMSE}
\DeclareMathOperator{\rnd}{rnd}
\DeclareMathOperator{\abso}{abs}
\DeclareMathOperator{\rel}{rel}
\DeclareMathOperator{\nor}{nor}
\DeclareMathOperator{\err}{err}
\DeclareMathOperator{\prob}{prob}
\DeclareMathOperator{\third}{third}
%\DeclareMathOperator{\fourth}{fourth}
\newtheorem{theorem}{Theorem}
\newtheorem{prop}[theorem]{Proposition}
\DeclareMathOperator{\sMC}{sMC}
\DeclareMathOperator{\aMC}{aMC}
\DeclareMathOperator{\nasty}{nasty}
\begin{document}

\title{When Does the Rectangle Rule Work}
\author{Fred J. Hickernell}
\address{Room E1-208, Department of Applied Mathematics, Illinois Institute of Technology, 10 W.\ 32$^{\text{nd}}$ St., Chicago, IL 60616}
\begin{abstract}
\end{abstract}
\maketitle

\section{General Setting}
Let $\ch$ be a Hilbert space of functions of interest.  Suppose one wishes to bound the value of some bounded, linear functional which is very difficult to compute:
\[
L:\ch \to \reals \qquad \text{what we want to bound}
\]
in terms of some bounded, linear functional which can be computed fairly easily
\[
M:\ch \to \reals \qquad \text{what we can compute}
\]
Furthermore, let these two bounded, linear functionals have representers, $\xi$ and $\zeta$, respectively, i.e., 
\[
L(f) = \ip[\ch]{\xi}{f}, \quad M(f) = \ip[\ch]{\zeta}{f}, \qquad \forall f \in \ch.
\]
We want to bound $\abs{L(f)}$ in terms of $\abs{M(f)}$

Decompose $\xi$ into a term parallel to $\zeta$ and a term perpendicular to $\zeta$:
\[
\xi = a \zeta + \zeta_{\perp}, \qquad \text{where } a := \frac{\ip[\ch]{\xi}{\zeta}}{\norm[\ch]{\zeta}^2}, \quad \zeta_{\perp} := \xi - a \zeta.
\]
Note that $\ip[\ch]{\zeta}{\zeta_{\perp}} = 0$.  Now any $f \in \ch$ may be decomposed as sum of three terms:
\[
f = b \zeta + b_{\perp} \zeta_{\perp} + f_{\perp}, \qquad \text{where } b := \frac{\ip[\ch]{f}{\zeta}}{\norm[\ch]{\zeta}^2}, \quad b_{\perp} := \frac{\ip[\ch]{f}{\zeta_{\perp}}}{\norm[\ch]{\zeta_{\perp}}^2}, \quad f_{\perp} := f - b \zeta -  b_{\perp} \zeta_{\perp}.
\]
Analogously, $\ip[\ch]{\zeta}{f_{\perp}} = \ip[\ch]{\zeta_{\perp}}{f_{\perp}} = 0$.  The bounded linear functionals of $f$ may be epxressed in terms of the coefficients defined above:
\begin{align*}
L(f)&=\ip[\ch]{\xi}{f}= \ip[\ch]{a \zeta + \zeta_{\perp}}{b \zeta + b_{\perp} \zeta_{\perp} + f_{\perp}} = ab \norm[\ch]{\zeta}^2 +  b_{\perp} \norm[\ch]{\zeta_{\perp}}^2 \\
M(f)&=\ip[\ch]{\zeta}{f}= \ip[\ch]{\zeta}{b \zeta + b_{\perp} \zeta_{\perp} + f_{\perp}} = b \norm[\ch]{\zeta}^2 \\
\norm[\ch]{f}^2&= \ip[\ch]{b \zeta + b_{\perp} \zeta_{\perp} + f_{\perp}}{b \zeta + b_{\perp} \zeta_{\perp} + f_{\perp}} =b^2 \norm[\ch]{\zeta}^2 +  b_{\perp}^2 \norm[\ch]{\zeta_{\perp}}^2  \norm[\ch]{\zeta}^2 + \norm[\ch]{f_{\perp}}^2
\end{align*}

Now we derive an upper bound on $\abs{L(f)}/\abs{M(f)}$.  Suppose $\abs{L(f)}/\abs{M(f)} \le C$.  Then $\abs{L(f)}$, what is not known, can be bounded above by $C\abs{M(f)}$, which can be computed.
\[
\abs{\frac{L(f)}{M(f)}} = \frac{\abs{ab \norm[\ch]{\zeta}^2 +  b_{\perp} \norm[\ch]{\zeta_{\perp}}^2}}{\abs{b} \norm[\ch]{\zeta}^2} \le \frac{\abs{ab} \norm[\ch]{\zeta}^2 +  \abs{b_{\perp}} \norm[\ch]{\zeta_{\perp}}^2}{\abs{b} \norm[\ch]{\zeta}^2} = \abs{a} +  \abs{\frac{b_{\perp}}{b}} \frac{\norm[\ch]{\zeta_{\perp}}^2}{\norm[\ch]{\zeta}^2}
\]
Assuming $\zeta_{\perp} \ne 0$, one may easily construct a \emph{nasty} function $f$, i.e., choose $b$ and $b_{\perp}$ such that $\abs{b_{\perp}/b}$ is arbitrarily large.  What we aim to do is find a nastiness criterion so that if the nastiness is bounded, then  $\abs{b_{\perp}/b}$ is bounded above.

We consider nastiness criteria of the form
\[
\nasty(f) = \frac{\ip[P]{f}{f}}{\ip[\ch_2]{f}{f}},
\]
where $\ip[P]{\cdot}{\cdot}$ and $\ip[\ch_2]{\cdot}{\cdot}$ are semi-inner products defined on $\ch$.  Furthermore,
\[
\ip[\ch]{f}{g} = \ip[\ch_1]{f}{g} + \ip[\ch_2]{f}{g}
\]
where $\ch=\ch_1 \oplus \ch_2$, and $\ip[\ch_1]{\cdot}{\cdot}$ and $\ip[\ch_2]{\cdot}{\cdot}$ are the inner products defined on $\ch_1$ and $\ch_2$, respectively.  Thus
\[
0 = \ip[\ch]{f_{\parallel}}{f_{\perp}} = \ip[\ch_1]{f_{\parallel,1}}{f_{\perp,1}} + \ip[\ch_2]{f_{\parallel,2}}{f_{\perp,2}}
\]

Let $f_{\parallel} = b \zeta + b_{\perp} \zeta_{\perp}$, so that $f=f_{\parallel} + f_{\perp}$.  Furthermore, let $f_{\parallel,j}$ and $f_{\perp,j}$, $j=1,2$ be the projections of $f_{\parallel}$ and $f_{\perp}$ into $\ch_j$

bilinear functions.  Note that using the above notation
\begin{align*}
P(f,f)& = P(b \zeta + b_{\perp} \zeta_{\perp} + f_{\perp},b \zeta + b_{\perp} \zeta_{\perp} + f_{\perp}) \\
& = b^2 P(\zeta,\zeta) +  2 b b_{\perp}  P(\zeta,\zeta_{\perp})  + b_{\perp}^2 P(f_{\perp},f_{\perp}) + 2 P(b \zeta + b_{\perp} \zeta_{\perp},f_{\perp}) +  P(f_{\perp},f_{\perp})
\end{align*}

\bibliographystyle{spbasic}
\bibliography{FJH21,FJHown21}
\end{document}
