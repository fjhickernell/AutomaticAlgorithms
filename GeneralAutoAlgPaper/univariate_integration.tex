\newcommand{\R}{\mathbb{R}}

In the following cases, we focus on approximation of upper bounds for truncation errors. The input function is in Sobolev space:
\begin{align*}
  \cf&=W^{2,1}=W^{2,1}[0,1]=\{f\in C^1[0,1]: \|f''\|_1<\infty\}, \text{ and }\\
  \cg&=W^{1,1}=W^{1,1}[0,1]=\{f\in C^1[0,1]: \|f'\|_1<\infty\},
\end{align*}
which have the semi-norm $\|f\|_{\cf}=\|f''\|_{1}$ and $\|f\|_{\cg}=\|f'\|_{1}$ respectively. The space of outputs $\ch$ is the real space $\R$. Given $\tau > 0$, we have the cone $\mathfrak{C}_{\tau} \subset \cf$ defined as $$\mathfrak{C}_{\tau}=\{f\in W^{2,1}:\|f''\|_1\leq\tau\|f'\|_1\}.$$

Define the solution operator $S$ as $S(f)=\int_{0}^{1}f(x)dx$. Consider using the trapezoidal rule as the algorithms $A_n$ which are going to use to approximate the integral. Suppose one wants to approximate the integration of functions in $\mathfrak{C}_{\tau}$ based on function values, the goal is to make the error small, i.e. to find an algorithm A, for which $|\int_{0}^{1}f(x)dx-A(f)|\leq \varepsilon$.

\subsection{Upper Bounds}
 Fix the algorithm $A_n$ with cost$(A_n)=n$. Suppose that the data sites $x_0,x_1,\cdots, x_{n-1}$ and the corresponding function values. One may estimate $\|f'\|_1$ in the following way: $$G_n(f)=\sup_{i\in\{0,\cdots,n-1\}}\left|\frac{f(x_{i+1})-f(x_{i})}{1/(n-1)}\right|.$$ According to the results from [book1], [book2], the upper bound error function can be expressed as $h(n)=1/[2(n-1)]$, $\tilde{h}(n)=1/[8(n-1)^2]$. For those two non-decreasing functions, the upper bounds on the error of $A_n(f)$ in terms of the $\cg$-semi-norm of $f$ is:
\begin{align*}
  \left|\int_{0}^{1}f(x)dx-A_n(f)\right|& \leq \min(h(n),\tilde{h}(n))\|f\|_{\cg}, \quad \forall f \in \mathfrak{C}_{\tau}\\
  &= \min\left(1,\frac{\tau}{4(n-1)}\right)\frac{\|f'\|_{1}}{2(n-1)},\\
  &\leq \frac{\tau\|f'\|_{1}}{8(n-1)^2}, \quad (n\geq \tau \Rightarrow \frac{\tau}{4(n-1)}<1).
\end{align*}

It is shown that in the specific case, the worst error in $W^{2,1}$ will be always smaller than the one in $W^{1,1}$. Therefore, the best worst error will be $\|f'\|_1/[8(n-1)^2]$.

\begin{algo} \label{algintegration}
\begin{description} Let $\varepsilon$ be the positive error tolerance, $N_{\text{max}}$ be the maximum cost budget. Let $\tau$ be a fixed positive constant and the sequence of algorithms $\{A_n\}_{n\in \mathcal{I}}$, $\{G_n\}_{n\in \mathcal{I}}$ be described above.Set $i=1$, let $n_1$ be the smallest number in $\mathcal{I}$ such that $1/(n-1)\leq1/\tau$. Then do the following:

\item[Stage 1.\ Estimate {$\|f''\|_{1}$}.] First compute $G_{n_i}(f)$

\item[Stage 2. Check for convergence.] Check whether $n_i$ is large enough to satisfy the error tolerance, i.e.
    \begin{align*}
      \frac{\tau}{8(n-1)^2}G_{n_i}(f)\leq \varepsilon.
    \end{align*}
    If this is true, then set $W=0$, return $(A_{n_i}(f),W)$ and terminate the algorithm. If this is not true, go to Stage 3.

\item[Stage 3. Compute $n_{i+1}$.] Choose $n_{i+1}$ as the smallest number exceeding $n_i$ and not less than $N_A(\varepsilon/G_{n_i}(f))$ such that $A_n$ is embedded in $A_{n+1}$. That is
    \begin{align*}
      n_{i+1}=(n_i-1)\left\lceil\frac{\sqrt{\frac{\tau G_{n}(f)}{\varepsilon(1 - \tau/ N_G)}}}{N_{G}-1}\right\rceil+1.
    \end{align*}
    If $n_{i+1}\leq N{\text{max}}$, increment $i$ by 1, and return to Stage 1. Otherwise, if $n_{i+1}> N{\text{max}}$, choose $n_{i+1}$ be the largest number not exceeding $N{\text{max}}$ such that $A_n$ is embedded in $A_{n+1}$ and set $W=1$. Return $(A_{n_{i+1}}(f),W)$ and terminate the algorithm.
\end{description}
\end{algo}

\subsection{Lower Bounds}
Now we consider the lower bound. We define the fooling function as $f_{\pm}=c_0f_0\pm c_1f_1$. The first piece of the fooling function is defined by $f_0=x$. In this case,
\begin{align*}
    \|f_0\|_{W^{1,1}}=\|f'_0\|_1=1,\\
    \|f_0\|_{W^{2,1}}=\|f''_0\|_1=0=\tau_0.
\end{align*}

The second part depends on the sample points. Given any $n$ sample points $\{\xi_i\}_{i=1}^{n}$ between $0$ and $1$ such that $0=\xi_0\leq \xi_1<\cdots<\xi_n\leq \xi_{n+1}=1$, let $l$ and $u$ be the two consecutive points with maximum distance between them, i.e. $\exists j\in \{1,2,\cdots,n-1\}$ such that $l=\xi_j, u=\xi_{j+1}$ and $u-l=\max_{1\leq i\leq n}(\xi_{i+1}-\xi_i)$. Then let
$$f_1(x)=\frac{30(x-l^2)(x-u)^2}{(u-l)^5}, \quad l\leq x\leq u.$$ and $f_1(x)=0$ otherwise. Note that $\|S(f_1)\|_{W^{1,1}}=int_{0}^{1}f_1(x)dx=1.$
\begin{align*}
  \|f_1\|_{W^{1,1}}&=\|f'_{1}\|_1=\frac{15}{4(u-l)}\leq \frac{15}{4}(n+1)=g(n),\\
  |f_1\|_{W^{2,1}}&=\|f''_{1}\|_1=\frac{40}{\sqrt{3}(u-l)^2}\leq \frac{40}{\sqrt{3}}(n+1)^2=\frac{32}{3\sqrt{3}}(n+1)g(n)=\tilde{g}(n)g(n).
\end{align*}
We can easily get that $g^{-1}(n)=\frac{4}{15}x-1$ and $(\tilde{g}(g))^{-1}(x)=\sqrt{\frac{\sqrt{3}}{40}x}-1$. So according to the Theorem 3 in the paper, $$\text{comp}\geq \min\left(\frac{4}{15}(\frac{\sigma}{4\varepsilon})-1,\sqrt{\frac{\sqrt{3}}{40}\frac{\sigma\tau}{4\varepsilon}}-1\right)\geq\min\left(\frac{\sigma}{15\varepsilon},\sqrt{\frac{\sqrt{3}\sigma\tau}{160\varepsilon}}\right)-1$$