The first example we consider is univariate integration on the unit interval, $S(f)=\int_{0}^{1}f(x) \, \dif x$.  The two spaces of input functions are Sobolev spaces of different degrees of smoothness:
\begin{equation*}
  \cf=\mathcal{W}^{2,1} \quad \text{and} \quad
  \cg=\mathcal{W}^{1,1},
\end{equation*}
where the general Sobolev spaces with smoothness of degree $k$ are defined as 
\begin{equation} \label{defSobolev}
  \mathcal{W}^{k,p}=\mathcal{W}^{k,p}[0,1]=\{f\in C[0,1]: \|f^{k}\|_{p}<\infty\}.
\end{equation}
The corresponding semi-norms are $\|f\|_{\cf}=\|f''\|_{1}$ and $\|f\|_{\cg}=\|f'\|_{1}$, respectively, which correspond to the total variation of $f'$ and $f$, respectively,  and the cone of integrands is defined as  
\begin{equation}\label{coneinteg}
\cc_{\tau}=\{f\in \mathcal{W}^{2,1}:\|f''\|_1\leq\tau\|f'\|_1\}.
\end{equation}
The space of outputs $\ch$ is $\reals$.

We consider algorithms consider algorithms based on function values, and the cost of a function value is one.  The non-adaptive building blocks to construct the adaptive algorithm are the composite trapezoidal rules based on $n-1$ trapezoids:
\begin{equation*}
    A_{n}(f)
    =\frac{1}{2n}[f(x_1)+2f(x_2)+\cdots+2f(x_{n-1})+f(x_n)], \qquad x_i=\frac{i-1}{n-1},
\end{equation*}
defined for $n \in \mathcal{I}=\{2,3,\ldots\}$, and with $\cost(A_n)=n$.  The algorithm $A_n$ is imbedded in the algorithm $A_{2n-1}$, which uses $2n-2$ trapezoids.  Thus, any trapezoidal rule is embedded in some other trapezoidal rule with cost no more than $r=2$ times the original one.  This is the minimum cost multiple as described just before Algorithm \ref{multistagealgo}.  
The algorithm for approximating $\|f'\|_{1}$ is the corresponding norm of the derivative of the linear spline using the same $x_i$ as used in the trapezoidal rule:
\begin{equation}\label{1direst}
    G_n(f)=\sum_{i=0}^{n-1}\left|f(x_{i+1})-f(x_{i})\right|.
\end{equation} 

\subsection{Adaptive Algorithm and Upper Bound on the Cost}

Constructing the adaptive algorithm for integration requires upper bounds on the errors of the $A_n$ and $G_n$.  The errors of the trapezoidal rule for integrands in the spaces $\mathcal{W}^{2,1}$ and $\mathcal{W}^{1,1}$ are bounded in terms of $h(n)=1/[8(n-1)^2]$ and $\tilde{h}(n)=1/[2(n-1)]$, respectively, according to \cite[(7.14) and (7.15)]{BraPet11a}.  Since $G_{n}(f)$ never overestimates $\|f'\|_{1}$, it follows that $h_{\cg_{-}}(n)=0$ and $\mathfrak{c}_n=1$. 

To find an upper bound on $\|f'\|_{1}-G_{n}(f)$, note that for $x_i \le x_{i+1}$
\begin{subequations} \label{foneasinteg}
\begin{equation}
f'(x)=(n-1)\left\{[f(x_{i+1})-f(x_{i})]+\int_{x_i}^{x_{i+1}}(n-1)v(t,x)f''(t) \, \dif t \right\}
\end{equation}
where 
\begin{equation}
v(t,x)=\begin{cases}  x_i-t, & x_i\leq t\leq x,\\
t-x_{i+1}, & x< t \leq x_{i+1}.
\end{cases}
\end{equation}
\end{subequations}
This implies the following upper bound on a piece of $\|f'\|_{1}$
\begin{align*}
\MoveEqLeft{\int_{x_i}^{x_{i+1}}|f'(x)| \, \dif x - |f(x_{i+1})-f(x_i)|}\\
& \le (n-1)\int_{x_i}^{x_{i+1}}\int_{x_i}^{x_{i+1}}|v(t,x)||f''(t)| \dif t \, \dif x\\
& \le  (n-1)\int_{x_i}^{x_{i+1}}2(t-x_i)(x_{i+1}-t)|f''(t)| \, \dif t \\
& \le  (n-1) \max_{x_i \le t \le x_{i+1}} \abs{2(t-x_i)(x_{i+1}-t)} \int_{x_i}^{x_{i+1}} |f''(t)| \, \dif t \\
 &  \leq  \frac{1}{2(n-1)}\int_{x_i}^{x_{i+1}} |f''(t)| \, \dif t .
\end{align*}
Applying this inequality over the entire interval, $[0,1]$ leads to 
\begin{align*}
\norm[1]{f'} - G_n(f)  &  = \sum_{i=1}^{n-1} \left \{  \int_{x_i}^{x_{i+1}}|f'(x)|\, \dif x - |f(x_{i+1})-f(x_i)| \right \} \\
& \le \frac{1}{2(n-1)} \sum_{i=1}^{n-1} \int_{x_i}^{x_{i+1}} |f''(t)| \, \dif t = 
\frac{\|f''\|_{1}}{2(n-1)}.
 \end{align*}
According to \eqref{Gerrbds} and , we have $h_{+}(n)=1/[2(n-1)]$ and an inflation factor of 
\begin{equation}\label{factor}
\mathfrak{C}_n =\frac{1}{1 - \tau/(2n-2)} \qquad \text{for } n>1+\tau/2.
\end{equation}

By condition \eqref{hbestverb}, it follows that $\min(\tildeh(n),\tau h(n))=\tau h(n)$, which then simplifies some of the expressions in Algorithm \ref{multistagealgo} and Theorem \ref{MultiStageThm}.  Specifically, the left side of the inequality in \eqref{multistageconv} becomes
\begin{subequations} \label{simpifycond}
\begin{equation}
\min(\tildeh(n_i),\tau h(n_i))\fC_{n_i} G_{n_i}(f) = \frac{\tau  G_{n_i}(f) } {4(n_i-1)(2n_i-2 -\tau)}.
\end{equation}
The function $N_A$ defined in \eqref{Nmindef} is
\begin{equation}
N_{A}(a)= \min\left\{ n \in \ci : \tau h(n) \le a \right\} = 1+ \left \lceil \sqrt{a/(8\tau)}\right \rceil.
\end{equation}
The denominator in the definition of the set of integrands $\cn$ in \eqref{nicefdefmulti} is
\begin{multline}
\fC_{N_{\max}/r} \fc_{N_{\max}/r} \min(\tildeh(N_{\max}/r),\tau h(N_{\max}/r)) \\
=
\frac{\tau}{2 (N_{\max}-2)(N_{\max}-2 -\tau)}.
\end{multline}
The function $\tN_A$ defined in \eqref{tNmindef} is
\begin{align}
\nonumber
\tN_A(a) &= \min\left\{ n \in \ci : \min(\tildeh(n),\tau h(n))\fC_n\fc_n \le a \right\} \\
&= 1+ \left \lceil \sqrt{\frac{\tau}{8a} + \frac{\tau^2}{16}} +\frac{\tau}{4} \right \rceil \le 1+ \left \lceil\sqrt{\frac{\tau}{8a}} + \frac{\tau}{2} \right \rceil.
\end{align}
\end{subequations}
With these preliminaries, Algorithm \ref{multistagealgo} and Theorem \ref{MultiStageThm} may be applied directly to  yield the following automatic, adaptive integration algorithm and its guarantee.

\begin{algo} \label{multistageintegalgo}
Let the error tolerance $\varepsilon$, the maximum cost budget $N_{\text{max}}$ and the cone constant $\tau$ be given inputs. Let the sequence of algorithms $\{A_n\}_{n\in \mathcal{I}}$, $\{G_n\}_{n\in \mathcal{I}}$ be described above. Set $i=1$. Let $n_1=\lceil(\tau+1)/2\rceil+1$. For any input function $f\in \mathcal{W}^{2,1}$, do the following:
\begin{description}
\item[Stage 1.\ Estimate {$\|f'\|_{1}$}.] Compute $G_{n_i}(f)$ in \eqref{1direst}.

\item[Stage 2. Check for convergence.] Check whether $n_i$ is large enough to satisfy the error tolerance, i.e.
    \begin{align*}
     G_{n_i}(f) \le \frac{4\varepsilon(n_i-1)(2n_i-2 - \tau)}{\tau}.
    \end{align*}
    If this is true, then set $W$ to be false, return $(A_{n_i}(f),W)$ and terminate the algorithm. If this is not true, go to Stage 3.

\item[Stage 3. Compute $n_{i+1}$.] Choose
$$
n_{i+1}=1+ (n_i-1)\max\left\{2,\left\lceil\frac{1}{(n_i-1)}\sqrt{\frac{\tau G_{n_i}(f)}{8\varepsilon}}\right\rceil\right\}.
$$
If $n_{i+1} \le N_{\max}$, increment $i$ by $1$, and return to Stage 1.

Otherwise, if $n_{i+1} > N_{\max}$, choose $n_{i+1}$ to be the largest number not exceeding $N_{\max}$ such that $A_{n_{i}}$ is embedded in $A_{n_{i+1}}$, and set $W$ to be true. Return $(A_{n_{i+1}}(f),W)$ and terminate the algorithm.
\end{description}
\end{algo}

\begin{theorem} \label{multistageintegthm} Let  $N_{\max}$, $\tau$, $n_1$, and $\varepsilon$ be given as described in Algorithm \ref{multistageintegalgo}.  Assume that $n_1 \le N_{\max}$.  Let $\cc_\tau$ be the cone of functions defined in \eqref{coneinteg}.  Let
$$
\cn
= \left \{ f \in \cc_\tau : \norm[1]{f'} \le \frac{2\varepsilon (N_{\max}-2)(N_{\max}-2 -\tau)}{\tau} \right\}
$$
be the nice subset of the cone $\cc_\tau$.  Then it follows that Algorithm \ref{multistageintegalgo} is successful for all functions in $\cn$,  i.e.,  $\success(A,W,\cn,\varepsilon,N_{\max}) = 1$.  Moreover, the cost of this algorithm is bounded above as follows:
\begin{align*}
\cost(A,\cn,\varepsilon,N_{\max},\norm[1]{f'})
&\le 2+2 \left \lceil \sqrt{\frac{\tau \norm[1]{f'}}{8\varepsilon} + \frac{\tau^2 }{16}} + \frac{\tau}{4} \right\rceil \\
&\le 2+2 \left \lceil\sqrt{\frac{\tau \norm[1]{f'}}{8\varepsilon}} + \frac{\tau}{2} \right \rceil.
\end{align*}
\end{theorem}

\subsection{Lower Bound on the Computational Cost}
Next, we derive a lower bound on the cost of approximating functions in the cone $\cc_{\tau}$ by constructing fooling functions. Following the arguments of Section \ref{LowBoundSec}, we choose  $f_0(x)=x.$ Then
\[
\norm[1]{f'_0}=1, \qquad \norm[1]{f''_0}=0, \qquad \tau_0=0.
\]
For any $n \in \naturals$, suppose that the one has the data $L_i(f)=f(\xi_i)$, $i=1, \ldots, n$ for arbitrary $\xi_i$, where $0=\xi_0 \le \xi_1 < \cdots < \xi_n \le \xi_{n+1} = 1$.  There must be some $j=0, \ldots, n$ such that $\xi_{j+1} - \xi_j \ge 1/(n+1)$.  The function $f_{1}$, defined as
$$
f_{1}(x):=\begin{cases} \displaystyle
\frac{30(x-\xi_{j})^{2}(\xi_{j+1}-x)^{2}}{(\xi_{j+1}-\xi_{j})^5} & \xi_{j} \le x \leq \xi_{j+1},\\
0 & \text{otherwise},
\end{cases}
$$
has $\int_0^1 f(x) \, \dif x=1$ and $f_1(\xi_i)=0$ for $i=0, \ldots, n+1$.  Moreover, the norms of the first and second derivatives of $f_1$ are
\begin{align*}
\norm[1]{f'_1}&=\frac{15}{4(\xi_{j+1}-\xi_{j})}\leq \tg(n),\\
\norm[1]{f''_1}&=\frac{40}{\sqrt{3}(\xi_{j+1}-\xi_{j})^2}
=\frac{32\norm[1]{f'_1}}{3\sqrt{3}(\xi_{j+1}-\xi_{j})}
 \leq g(n) \norm[1]{f'_1},
\end{align*}
where the inequality $\xi_{j+1} - \xi_j \ge 1/(n+1)$ is used to define $\tg$ and $g$ as
\[
\tg(n) = \frac{15(n+1)}{4}, \qquad g(n) = \frac{32(n+1)}{3\sqrt{3}}, \qquad (g\tg)(n) = \frac{40(n+1)^2}{\sqrt{3}}.
\]
Using these choices of $f_0$ and $f_1$, along with the corresponding $\tg$ and $g$ above, one may invoke Proposition \ref{optimalprop}, Theorem \ref{complowbd}, and Corollary \ref{optimcor} to obtain the following theorem.

\begin{theorem} \label{complowbdinteg} The adaptive trapezoidal algorithm is optimal for integration of functions in $\mathcal{W}^{1,1}$ and $\mathcal{W}^{2,1}$. Assuming an infinite cost budget, the complexity of the function recovery problem problem, over the cone of functions $\cc_{\tau}$ defined in \eqref{coneinteg} is
\begin{align*}
\comp(\varepsilon,\ca(\cc_{\tau},\ch,S,\Lambda),\infty,\norm[1]{f'})
&\ge \min\left(\frac{\norm[1]{f'}}{15 \varepsilon}, \sqrt{\frac{\sqrt{3}\tau\norm[1]{f'}}{160\varepsilon}} \right)-1\\
& \sim \sqrt{\frac{\sqrt{3}\tau\norm[1]{f'}}{160\varepsilon}}  \quad\text{as } \frac{\norm[1]{f'}}{\varepsilon} \to \infty.
\end{align*}
Moreover, Algorithm \ref{multistageapproalgo} is optimal for approximating functions in $\cc_{\tau}$.
\end{theorem}

\subsection{Numerical Example}
Consider the family of test functions defined by 
\begin{equation} \label{testfun}
f(x)=b\me^{-a^2(x-z)^2},
\end{equation}
where $z \sim \cu[1/\sqrt{2}a,1-1/\sqrt{2}a]$, $\log_{10}(a) \sim \cu[1,4]$, and $b$ is chosen to make $\int_0^1 f(x) \, \dif x = 1$.
If $a$ is large, then $f$ is spiky, and it is difficult to approximate the integral. From the numerical results, we can find the success rate for our algorithm. Straightforward calculations yield the norms of the first two derivatives of this test function: 
\begin{gather*}
\norm[1]{f'}=2-\me^{-a^2z^2}-\me^{-a^2(1-z)^2}, \\ 
\norm[1]{f''}=2a\left\{2\sqrt{\frac{2}{\me}}-2a \left[z \me^{-a^2z^2}+(1-z)\me^{-a^2(1-z)^2}\right] \right\}.
\end{gather*}
Monte Carlo simulation is used to determine the probability that $f \in \cc_{\tau}$ for $\tau = 10, 100$, and $1000$ (see Table \ref{integresultstable}). 
\begin{table}[h]
\centering
\begin{tabular}{cccc|cccc}
\multicolumn{4}{c|}{Algorithm \ref{multistageintegalgo}} \\
$\tau$ &  10 & 100 & 1000 & {\tt quad} & {\tt quadgk} & {\tt chebfun} \\
\toprule
Probability of $f \in \cc_{\tau}$ &  $0\%$ &  $25\%$  & $58\%$ \\
Observed Success Rate & $37\%$ &  $69\%$  & $97\%$ & $30\%$ & $77\%$ & $68\%$
\end{tabular}
\caption{Performance of multistage, adaptive Algorithm \ref{multistageintegalgo} for various values of $\tau$, and also of other automatic quadrature algorithms. The test function \eqref{testfun} with random parameters, $a$ and $z$, was used. \label{integresultstable}}
\end{table}

Ten thousand random instances of this test function are integrated by  Algorithm \ref{multistageintegalgo} and three existing automatic algorithms with an absolute error tolerance of $\varepsilon=10^{-8}$.  The observed success rates for these algorithms are shown.  As expected, the success rate of {Algorithm \ref{multistageintegalgo} increases as $\tau$ is increased.  All of the integrands lying inside $\cc_{\tau}$ are integrated to within the error tolerance, and a significant number of integrands lying outside the cone are integrated accurately as well.