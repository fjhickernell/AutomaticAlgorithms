\documentclass[]{elsarticle}
%\setlength{\marginparwidth}{0.5in}
\usepackage{amsmath,amssymb,amsthm,mathtools,bbm,booktabs,array,tikz,pifont,comment,multirow,url,graphicx}
\input FJHDef.tex

%Requires ApproxUnivariate_k.tex, univariate_integration_k.tex, ConesPaperSpikyquad.eps, ConesPaperFlukyquad.eps

\DeclareMathOperator{\Var}{Var}
\DeclareMathOperator{\INT}{INT}
\DeclareMathOperator{\APP}{APP}
\DeclareMathOperator{\lin}{lin}
\DeclareMathOperator{\up}{up}
\DeclareMathOperator{\lo}{lo}
\DeclareMathOperator{\fix}{fix}
\DeclareMathOperator{\err}{err}
\DeclareMathOperator{\maxcost}{maxcost}
\DeclareMathOperator{\mincost}{mincost}
\newcommand{\herr}{\widehat{\err}}

\newtheorem{theorem}{Theorem}
\newtheorem{prop}[theorem]{Proposition}
\newtheorem{lem}{Lemma}
\newtheorem{cor}{Corollary}
\theoremstyle{definition}
\newtheorem{algo}{Algorithm}
\newtheorem{condit}{Condition}
%\newtheorem{assump}{Assumption}
\theoremstyle{remark}
\newtheorem{rem}{Remark}
\newcommand{\Fnorm}[1]{\abs{#1}_{\cf}}
\newcommand{\Ftnorm}[1]{\abs{#1}_{\tcf}}
\newcommand{\Gnorm}[1]{\norm[\cg]{#1}}
\newcommand{\flin}{f_{\text{\rm{lin}}}}

\journal{Journal of Complexity}

\begin{document}

\begin{frontmatter}

\title{The Cost of Deterministic, Adaptive, Automatic Algorithms:  Cones, Not Balls}
\author{Nicholas Clancy} \ead{nclancy@hawk.iit.edu}
\author{Yuhan Ding} \ead{yding2@hawk.iit.edu}
\author{Caleb Hamilton} \ead{chamilt2@hawk.iit.edu}
\author{Fred J. Hickernell} \ead{hickernell@iit.edu}
\author{Yizhi Zhang} \ead{yzhang97@hawk.iit.edu}
\address{Room E1-208, Department of Applied Mathematics, Illinois Institute of Technology,\\ 10 W.\ 32$^{\text{nd}}$ St., Chicago, IL 60616}

\end{frontmatter}

There are two minor typographical errors that we would appreciate your correcting.

\begin{enumerate}

\item On p.\ 8 in Theorem 3, equation (15) the constant ${\color{red}\tau}$ is misplaced and should be moved(color added for emphasis).  \emph{At present} it reads
\begin{multline} \tag{15}
\max \left(n_1, h^{-1}\left(\frac{\varepsilon}{{\color{red}\tau} \Fnorm{f}} \right) \right) \le \max \left(n_1, h^{-1}\left(\frac{\varepsilon}{\Ftnorm{f}} \right) \right) \\ \le
\cost(A,f,\varepsilon) \le \quad \text{etc.}
\end{multline}
but it \emph{should read}
\begin{multline} \tag{15}
\max \left(n_1, h^{-1}\left(\frac{\varepsilon}{\Fnorm{f}} \right) \right) \le \max \left(n_1, h^{-1}\left(\frac{\varepsilon}{{\color{red}\tau} \Ftnorm{f}} \right) \right) \\ \le
\cost(A,f,\varepsilon) \le \quad \text{etc.}
\end{multline}

\item On p.\ 17 in Section 5.3, equation (28) a minus sign should be a plus sign.  \emph{At present} it reads
\begin{multline} \tag{28}
f(x)= \\
\begin{cases}
\displaystyle  b[4a^2 + (x-z)^2 {\color{red}-} (x-z-a)|x-z-a|\\
\qquad \qquad -(x-z+a)|x-z+a|], & z-2a\leq x\leq z+2a,\\[2ex]
\displaystyle  0, & \text{otherwise}.
\end{cases}
\end{multline}
but it \emph{should read}
\begin{multline} \tag{28}
f(x)= \\
\begin{cases}
\displaystyle  b[4a^2 + (x-z)^2 {\color{red}+} (x-z-a)|x-z-a|\\
\qquad \qquad -(x-z+a)|x-z+a|], & z-2a\leq x\leq z+2a,\\[2ex]
\displaystyle  0, & \text{otherwise}.
\end{cases}
\end{multline}

\end{enumerate}

\end{document}

