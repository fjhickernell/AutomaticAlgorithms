\documentclass[final]{elsarticle}
\setlength{\marginparwidth}{0.5in}
\usepackage{amsmath,amssymb,amsthm,natbib,mathtools,bbm,graphicx}
\input FJHDef.tex

\newcommand{\flin}{f_{\text{\rm{lin}}}}
\newcommand{\chin}{\ch_{\text{\rm{in}}}}
\newcommand{\chout}{\ch_{\text{\rm{out}}}}
\newcommand{\pin}{p_{\text{\rm{in}}}}
\newcommand{\pout}{p_{\text{\rm{out}}}}
%\newcommand{\cc}{\mathcal{C}}
\newcommand{\cq}{\mathcal{Q}}
\newcommand{\bbW}{\mathbb{W}}
%\newcommand{\tP}{\widetilde{P}}
\newcommand{\bg}{{\bf g}}
\newcommand{\bu}{{\bf u}}
\newcommand{\bbu}{\bar{\bf u}}
\newcommand{\bv}{{\bf v}}
\newcommand{\bbv}{\bar{\bf v}}
\newcommand{\bw}{{\bf w}}
\newcommand{\bbw}{\bar{\bf w}}
%\newcommand{\hv}{\hat{v}}
\newcommand{\Fnorm}[1]{\abs{#1}_{\cf}}
\newcommand{\FYnorm}[1]{\abs{#1}_{\cf_{\cy}}}
\newcommand{\Gnorm}[2]{\abs{#2}_{\cg,#1}}
\newcommand{\GYnorm}[1]{\abs{#1}_{\cg_{\cy}}}
\newcommand{\Hnorm}[1]{\norm[\ch]{#1}}
\newtheorem{theorem}{Theorem}
\newtheorem{prop}[theorem]{Proposition}
\newtheorem{lem}[theorem]{Lemma}
\theoremstyle{definition}
\newtheorem{algo}{Algorithm}
\newtheorem{condit}{Condition}
%\newtheorem{assump}{Assumption}
\theoremstyle{remark}
\newtheorem{rem}{Remark}
\DeclareMathOperator{\err}{err}
\DeclareMathOperator{\fix}{fix}
\DeclareMathOperator{\up}{up}
\DeclareMathOperator{\lo}{lo}

\journal{Journal of Complexity}

\begin{document}

\begin{frontmatter}

\title{A Better $\cg$-Semi-Norm for Algorithms Based on Linear Splines}
\author{Fred J. Hickernell}
\address{Room E1-208, Department of Applied Mathematics, Illinois Institute of Technology,\\ 10 W.\ 32$^{\text{nd}}$ St., Chicago, IL 60616}
\begin{abstract}
\end{abstract}

\begin{keyword}
%% keywords here, in the form: keyword \sep keyword

%% MSC codes here, in the form: \MSC code \sep code
%% or \MSC[2008] code \sep code (2000 is the default)

\end{keyword}
\end{frontmatter}

%%%%%%%%%%%%%%%%%%%%%%%%%%%%%%%%%%%%%%%%%%%%%%%%%%%
\section{Why Do We Need a Better $\cg$-Semi-Norm}
In \cite{HicEtal14b} we presented guaranteed automatic algorithms for integration and function recovery.  One weakness of these algorithms was that if $\norm[1]{f'}$ ($\norm[\infty]{f'}$) is large, but $\norm[1]{f''}$ ($\norm[\infty]{f''}$) is small, then the error bounds in the automatic algorithms for integration (function recovery) can be very conservative.  The input function $f:x\mapsto 100 x$ is a prime example.

Here we present a substitute for $\norm[p]{f'}, \ p=1,\infty$ that overcomes this weakness.  If the function is nearly linear, then the algorithms in \cite{HicEtal14b} converges quite quickly.

The following semi-norm compares measures the deviation of the first derivative of the function from its average value:
\begin{equation} \label{Gnormdef}
\Gnorm{p}{f}=\norm[p]{f' - f(1)+f(0)}.
\end{equation}
Let us verify that this indeed a semi-norm.  Clearly, it is non-negative and vanishes when $f$ is the zero function.  So, the only other thing to verify is the triangle inequality, which follows since $f \mapsto f' - f(1)-f(0)$ is a linear operator.

%%%%%%%%%%%%%%%%%%%%%%%%%%%%%%%%%%%%%%%%%%%%%%%%%%%
\section{Estimating the New $\cg$-Semi-Norm}

Given $n =2, 3, \ldots$, let $x_i=(i-1)/(n-1)$ for $i=1, \ldots, n$.  Let $\tf_n$ denote the linear spline based on the function values $f(x_{i,n-1})$, $i=0, \ldots, n-1$. We  estimate $\Gnorm{p}{f}$ by the algorithm $G_{p,n}:f \mapsto \bigabs{\tf_n}_{\cg,p}$.  

This algorithm never overestimates the norm.  To see why, first note that $\Gnorm{p}{f}=\Gnorm{p}{f+\flin}$ and $G_{p,n}(f)=G_{p,n}(f+\flin)$ for any linear function $\flin$.  For any given $f$, choose $\flin$ to interpolate $f$ at the two endpoints of the interval, i.e., $\flin:x \mapsto f(1)x+f(0)(1-x)$.  Then it follows that
\[
\Gnorm{p}{f} - G_{p,n}(f) = \bigabs{f-\flin}_{\cg,p}- G_{p,n}(f-\flin) = \norm[p]{f'-\flin'} - \bignorm[p]{\tf'-\flin'},
\]
since $f-\flin$ vanishes at $0$ and $1$. For the case $1 \le p < \infty$ it follows that
\begin{align*}
\MoveEqLeft{\norm[p]{f'-\flin'}^p - \bignorm[p]{\tf'-\flin'}^p}\\
 & = \int_0^1 \left[\bigabs{(f'-\flin')(x)}^p - \bigabs{(\tf'-\flin')(x)}^p \right] \, \dif x \\
& = \sum_{i=1}^{n-1} \int_{x_{i}}^{x_{i+1}} \left[\bigabs{(f'-\flin')(x)}^p - \bigabs{(\tf'-\flin')(x)}^p \right] \, \dif x \\
& = \sum_{i=1}^{n-1} \left[ \int_{x_{i}}^{x_{i+1}} \bigabs{(f'-\flin')(x)}^p - \biggabs{\int_{x_{i}}^{x_{i+1}} (\tf'-\flin')(x) \, \dif x}^p \right] \\
& \ge 0,
\end{align*}
since $\tf'-\flin'$ is constant over each interval $[x_i,x_{i+1}]$.  The proof is similar for $p=\infty$.

An upper bound on the the error of our approximation to the $\cg$-semi-norm is 
\begin{equation*}
\Gnorm{p}{f} - G_{p,n}(f) = \Gnorm{p}{f} - \bigabs{\tf_n}_{\cg,p} \le \bigabs{f-\tf_n}_{\cg,p} = \bignorm[p]{f' -\tf'_n},
\end{equation*}
since $(f-\tf_n)(x)$ vanishes for $x=0,1$.  We are most interested in the cases $p=1,\infty$.

For any $i=1, \ldots, n-1$, the idea is to estimate how large $(f-\tf_n)(x)$ can become for $x \in [x_i,x_{i+1}]$.  Note that
\begin{align}
\nonumber
f'(x)-\tf'_n(x) & = f'(x) - (n-1)[f(x_{i+1})-f(x_{i})] \\
& = -(n-1)\int_{x_i}^{x_{i+1}}v(t,x)f''(t) \, \dif t
\label{fprimeexpress}
\end{align}
where 
\begin{equation*}
v(t,x)=\begin{cases}  x_i-t, & x_i\leq t\leq x,\\
t-x_{i+1}, & x< t \leq x_{i+1}.
\end{cases}
\end{equation*}
This implies the following upper bound on a piece of $\|f'\|_{1}$:
\begin{align*}
\MoveEqLeft{\int_{x_i}^{x_{i+1}}|f'(x)-\tf'_n(x)| \, \dif x}\\
& \le (n-1) \int_{x_i}^{x_{i+1}} \int_{x_i}^{x_{i+1}}|v(t,x)||f''(t)| \dif t \, \dif x\\
& \le  (n-1)\int_{x_i}^{x_{i+1}}2(t-x_i)(x_{i+1}-t)|f''(t)| \, \dif t \\
& \le  (n-1) \max_{x_i \le t \le x_{i+1}} \abs{2(t-x_i)(x_{i+1}-t)} \int_{x_i}^{x_{i+1}} |f''(t)| \, \dif t \\
&  \le \frac{1}{2(n-1)}\int_{x_i}^{x_{i+1}} |f''(t)| \, \dif t .
\end{align*}
Applying this inequality for $i=1, \ldots, n-1$ leads to 
\begin{align*}
\bigabs{\Gnorm{1}{f} - G_{1,n}(f)} &  \le \bignorm[1]{f' -\tf_n} \\
& = \sum_{i=1}^{n-1} \left \{  \int_{x_i}^{x_{i+1}}|f'(x)|\, \dif x - |f(x_{i+1})-f(x_i)| \right \} \\
& \le \frac{1}{2(n-1)} \sum_{i=1}^{n-1} \int_{x_i}^{x_{i+1}} |f''(t)| \, \dif t = 
\frac{\|f''\|_{1}}{2(n-1)}.
 \end{align*}

The bound for the case $p=\infty$ follows by a similar argument.  Starting from \eqref{fprimeexpress}, it follows that for $x \in [x_i,x_{i+1}]$,
\begin{align*}
|f'(x)-\tf_n(x)|
&\le(n-1)\abs{ \int_{x_i}^{x_{i+1}} v(t,x) f''(t) \, \dif t} \\
& \le (n-1) \norm[\infty]{f''}\int_{x_i}^{x_{i+1}} \abs{v(t,x)} \, \dif t \\
&=(n-1)\norm[\infty]{f''} \left \{\frac{1}{2(n-1)^2} - (x-x_i)(x_{i+1}-x) \right\} \\
&\le \frac{\norm[\infty]{f''}}{2(n-1)}.
\end{align*}
Applying the above argument for $i=1, \ldots, n-1$ leads to 
\begin{equation*}
\bigabs{\Gnorm{\infty}{f} - G_{\infty,n}(f)} \le \bignorm[\infty]{f' -\tf_n} \le  
\frac{\|f''\|_{\infty}}{2(n-1)}.
 \end{equation*}


%%%%%%%%%%%%%%%%%%%%%%%%%%%%%%%%%%%%%%%%%%%%%%%%%%%
\section{Bounding the Error in Terms of the New $\cg$-Semi-Norm}
\subsection{Integration}

For the integration problem, we use the composite trapezoidal rule, 
\begin{equation*}
A_n(f) = \frac{1}{2(n-1)} \left[f(x_{1}) + 2f(x_{2}) + 2f(x_{3}) + \cdots
 + 2f(x_{n-1}) + f(x_{n})\right].
\end{equation*}
We consider how to bound the error of this approximation for integrands with given norm $\Gnorm{1}{f}$.  Using the trick of subtracting a the function $\flin$ defined above, note that
\begin{align*}
\abs{\int_0^1 f(x) \, \dif x - A_n(f)}
 & = \abs{\int_0^1 (f-\flin)(x) \, \dif x - A_n(f-\flin)} \\
& \le \frac{\norm[1]{f'-\flin'}}{8(n-1)^2}= \frac{\Gnorm{1}{f}}{8(n-1)^2}\\
\end{align*}

\subsection{Approximation}

For the approximation problem, we use linear splines.  Using again the trick of subtracting a the function $\flin$ defined above, note that
\begin{align*}
\abs{f(x) - A_n(f)(x)}
 & = \abs{(f-\flin)(x) \, \dif x - A_n(f-\flin)(x)} \\
& \le \frac{\norm[\infty]{f'-\flin'}}{8(n-1)^2}= \frac{\Gnorm{\infty}{f}}{8(n-1)^2}.
\end{align*}


%%%%%%%%%%%%%%%%%%%%%%%%%%%%%%%%%%%%%%%%%%%%%%%%%%%
\section{Lower Bound Using the New $\cg$-Semi-Norm}
With the new $\cg$-semi-norm, the function $f_0:x \mapsto x$ is no longer works because for this choice one would have $\Gnorm{p}{f_0}=0$.  Instead we choose:
\[
f_0(x) = x(1-x).
\]
For $p=1$ one has 
\[
\Gnorm{1}{f_0}=\norm[1]{f_0'}=\frac 12, \qquad \norm[1]{f_0''}=2, \qquad \tau_0=4.
\]
For $p=\infty$ one has 
\[
\Gnorm{1}{f_0}=\norm[1]{f_0'}=1, \qquad \norm[1]{f_0''}=2, \qquad \tau_0=2.
\]

Using these, the complexity lower bound for integration is
\[
\min\left(\frac{(\tau-4)\Gnorm{1}{f'}}{15(\tau-2) \varepsilon}, \sqrt{\frac{\sqrt{3}\tau(\tau-4)\Gnorm{1}{f'}}{160(\tau-2)\varepsilon}} \right)-1, 
\]
provided $\tau>4$.  The complexity lower bound for approximation is
\[
\min\left(\frac{3\sqrt{3}(\tau-2)\Gnorm{1}{f'}}{64(\tau-1) \varepsilon}, \sqrt{\frac{\tau(\tau-2)\Gnorm{1}{f'}}{128(\tau-1)\varepsilon}} \right)-1, 
\]
provided $\tau>2$.


\bibliographystyle{elsarticle-num.bst}
\bibliography{FJH22,FJHown22}
\end{document}

