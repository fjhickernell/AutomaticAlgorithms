\documentclass[final]{elsarticle}
\setlength{\marginparwidth}{0.5in}
\usepackage{amsmath,amssymb,amsthm,natbib,mathtools,bbm,graphicx}
\input FJHDef.tex

\newcommand{\chin}{\ch_{\text{\rm{in}}}}
\newcommand{\chout}{\ch_{\text{\rm{out}}}}
\newcommand{\pin}{p_{\text{\rm{in}}}}
\newcommand{\pout}{p_{\text{\rm{out}}}}
%\newcommand{\cc}{\mathcal{C}}
\newcommand{\cq}{\mathcal{Q}}
\newcommand{\bbW}{\mathbb{W}}
%\newcommand{\tP}{\widetilde{P}}
\newcommand{\bg}{{\bf g}}
\newcommand{\bu}{{\bf u}}
\newcommand{\bbu}{\bar{\bf u}}
\newcommand{\bv}{{\bf v}}
\newcommand{\bbv}{\bar{\bf v}}
\newcommand{\bw}{{\bf w}}
\newcommand{\bbw}{\bar{\bf w}}
%\newcommand{\hv}{\hat{v}}
\newcommand{\Fnorm}[1]{\abs{#1}_{\cf}}
\newcommand{\FYnorm}[1]{\abs{#1}_{\cf_{\cy}}}
\newcommand{\Gnorm}[3]{\abs{#3}_{\cg,#2,#1}}
\newcommand{\GYnorm}[1]{\abs{#1}_{\cg_{\cy}}}
\newcommand{\Hnorm}[1]{\norm[\ch]{#1}}
\newtheorem{theorem}{Theorem}
\newtheorem{prop}[theorem]{Proposition}
\newtheorem{lem}[theorem]{Lemma}
\theoremstyle{definition}
\newtheorem{algo}{Algorithm}
\newtheorem{condit}{Condition}
%\newtheorem{assump}{Assumption}
\theoremstyle{remark}
\newtheorem{rem}{Remark}
\DeclareMathOperator{\err}{err}
\DeclareMathOperator{\fix}{fix}
\DeclareMathOperator{\up}{up}
\DeclareMathOperator{\lo}{lo}

\journal{Journal of Complexity}

\begin{document}

\begin{frontmatter}

\title{A Better $\cg$-Semi-Norm for Algorithms Based on Linear Splines}
\author{Fred J. Hickernell}
\address{Room E1-208, Department of Applied Mathematics, Illinois Institute of Technology,\\ 10 W.\ 32$^{\text{nd}}$ St., Chicago, IL 60616}
\begin{abstract}
\end{abstract}

\begin{keyword}
%% keywords here, in the form: keyword \sep keyword

%% MSC codes here, in the form: \MSC code \sep code
%% or \MSC[2008] code \sep code (2000 is the default)

\end{keyword}
\end{frontmatter}

%%%%%%%%%%%%%%%%%%%%%%%%%%%%%%%%%%%%%%%%%%%%%%%%%%%
\section{Why Do We Need a Better $\cg$-Semi-Norm}
In \cite{HicEtal14b} we presented guaranteed automatic algorithms for integration and function recovery.  One weakness of these algorithms was that if $\norm[1]{f'}$ ($\norm[\infty]{f'}$) is large, but $\norm[1]{f''}$ ($\norm[\infty]{f''}$) is small, then the error bounds in the automatic algorithms for integration (function recovery) can be very conservative.  The input function $f:x\mapsto 100 x$ is a prime example.

Here we present a family of substitutes for $\norm[p]{f'}, \ p=1,\infty$ that overcome this weakness.  If the function is piecewise linear with a small number of discontinuities in $f'$, then the algorithms in \cite{HicEtal14b} converge quite quickly.

%%%%%%%%%%%%%%%%%%%%%%%%%%%%%%%%%%%%%%%%%%%%%%%%%%%
\section{A Proposed New $\cg$-Semi-Norm}
Define the integral functional on the interval $[\alpha,\beta]$ as 
\[
I(f;\alpha,\beta) = \int_\alpha^\beta f(x) \, \dif x,
\]
and define the $\cl_p$-norms on this as follows:
\[
\norm[p,{[\alpha,\beta]}]{f} = \begin{cases} \Bigabs{\int_\alpha^\beta \abs{f(x)}^p \, \dif x}^{1/p}, & 1 \le p < \infty,\\
\displaystyle \sup_{a \le x \le b}\abs{f(x)}, & p=\infty.
\end{cases}
\]
The norm over the larger interval can be written in terms of the norms of subintervals.  Namely, for any $\alpha_j$ satisfying $\alpha=\alpha_0 < \alpha_1 < \cdots < \alpha_r=\beta$, it follows that
\begin{equation}\label{subintnorm}
\norm[p,{[\alpha,\beta]}]{f} = \Bignorm[\ell_p]{\bigvecpar{\norm[p,{[\alpha_{k-1},\alpha_k]}]{f}}{k=1}{r}}.
\end{equation}

For any $m \in \naturals$, define the following equally spaced points:
\[
x_{i,m}=\frac{i}{m}, \qquad i=0, \ldots m.
\]
The following semi-norm compares measures the deviation of the first derivative of the function from its average value:
\begin{equation} \label{Gnormdef}
\Gnorm{p}{m}{f}=\Bignorm[\ell_p]{\Bigvecpar{\norm[p,{[x_{i-1,m},x_{i,m}]}]{\norm[p,{[x_{i-1,m},x_{i,m}]}]{f' - m I(f';x_{i-1,m},x_{i,m})}}} {i=1}{m}}.
\end{equation}
Let us verify that this indeed a semi-norm.  Clearly, it is non-negative and vanishes when $f$ is the zero function.  So, the only other thing to verify is the triangle inequality, which follows since $f \mapsto f' - m I(f';x_{i-1,m},x_{i,m})$ is a linear operator.

It can be shown that $\Gnorm{2}{rm}{f} \le \Gnorm{2}{m}{f}$ for positive integer $r$.  First, note that the one may re-write the norm over $[x_{i-1,m},x_{i,m}]$ in terms of norms over $[x_{r(i-1)+k-1,rm},x_{r(i-1)+k,rm}]$ by \eqref{subintnorm}:
\begin{multline} \label{twonormsub}
\norm[2,{[x_{i-1,m},x_{i,m}]}]{f' - m I(f';x_{i-1,m},x_{i,m})} \\
= \Bignorm[\ell_2]{\bigvecpar{\norm[2,{[x_{r(i-1)+k-1,rm},x_{r(i-1)+k,rm}]}]{f' - m I(f';x_{i-1,m},x_{i,m})}}{k=1}{r}}.
\end{multline}
Moreover, the norms inside the expression above are no smaller than the norms comprising $\Gnorm{2}{rm}{f}$ :
\begin{align*}
\MoveEqLeft{\norm[2,{[x_{r(i-1)+k-1,rm},x_{r(i-1)+k,rm}]}]{f' - m I(f';x_{i-1,m},x_{i,m})}^2}\\
&=\int_{x_{r(i-1)+k-1}}^{x_{r(i-1)+k,rm}} [f'(x) - m I(f';x_{i-1,m},x_{i,m})]^2 \, \dif x \\
&=\int_{x_{r(i-1)+k-1}}^{x_{r(i-1)+k,rm}} \bigl\{ [f'(x) - rm I(f';x_{r(i-1)+k-1},x_{r(i-1)+k,rm})]^2  \\
& \qquad \qquad + 2m[f'(x) - rm I(f';x_{r(i-1)+k-1},x_{r(i-1)+k,rm})] \\
& \qquad \qquad \qquad \times [r I(f';x_{r(i-1)+k-1},x_{r(i-1)+k,rm})- I(f';x_{i-1,m},x_{i,m})]\\
&\qquad \qquad + m^2 [r I(f';x_{r(i-1)+k-1},x_{r(i-1)+k,rm})- I(f';x_{i-1,m},x_{i,m})]^2 \bigr\} \, \dif x \\
&=\norm[2,{[x_{r(i-1)+k-1},x_{r(i-1)+k,rm}]}]{f' - rm I(f';x_{r(i-1)+k-1},x_{r(i-1)+k,rm})}^2  \\
&\qquad \qquad + \frac{1}{r^2} [r I(f';x_{r(i-1)+k-1},x_{r(i-1)+k,rm})- I(f';x_{i-1,m},x_{i,m})]^2\\
& \ge \norm[2,{[x_{r(i-1)+k-1},x_{r(i-1)+k,rm}]}]{f' - rm I(f';x_{r(i-1)+k-1},x_{r(i-1)+k,rm})}^2 
\end{align*}

Substituting this inequality into \eqref{twonormsub} yields 
\begin{multline*}
\norm[2,{[x_{i-1,m},x_{i,m}]}]{f' - m I(f';x_{i-1,m},x_{i,m})} \\
\ge \Bignorm[\ell_2]{\bigvecpar{\norm[2,{[x_{r(i-1)+k-1},x_{r(i-1)+k,rm}]}]{f' - rm I(f';x_{r(i-1)+k-1},x_{r(i-1)+k,rm})}}{k=1}{r}}.
\end{multline*}
According to the definition of $\Gnorm{2}{m}{f}$ in \eqref{Gnormdef}, it then follows that $\Gnorm{2}{rm}{f} \le \Gnorm{2}{m}{f}$.

%%%%%%%%%%%%%%%%%%%%%%%%%%%%%%%%%%%%%%%%%%%%%%%%%%%
\section{Estimating the New $\cg$-Semi-Norm}

Let $\tf_n$ denote the linear spline based on the function values $f(x_{i,n-1})$, $i=0, \ldots, n-1$. We  estimate $\Gnorm{p}{(n-1)/2}{f}$ by the algorithm $G_{(n-1)/2,p,n}:f \mapsto \bigabs{\tf_n}_{\cg,(n-1)/2,p}$.  Then it follows that the error of our approximation to the $\cg$-semi-norm is 
\begin{align*}
\Gnorm{p}{(n-1)/2}{f} - G_{(n-1)/2,p,n}(f) &= \Gnorm{p}{(n-1)/2}{f} - \bigabs{\tf_n}_{\cg,(n-1)/2,p} \\
&\le \bigabs{f-\tf_n}_{\cg,(n-1)/2,p},
\end{align*}


For any $i=1, \ldots, n-2$, the idea is to approximate  The idea is to approximate
\[
\int_{x_{i-1}}^{x_{i+1}} \abs{f'(x)-\frac{n-1}{2}I(f';x_{i-1},x_{i+1})}^p \, \dif x 
\]
by piecewise linear approximation to $f$, which implies a piecewise constant approximation to $f'$.  
\[
a
\] 

%%%%%%%%%%%%%%%%%%%%%%%%%%%%%%%%%%%%%%%%%%%%%%%%%%%
\section{Bounding the Error in Terms of the New $\cg$-Semi-Norm}
\subsection{Integration}

For the integration problem, we use the composite trapezoidal rule, 
\begin{multline*}
A_n(f) = \frac{1}{2(n-1)} \left[f(x_{0,n-1}) + 2f(x_{1,n-1}) + 2f(x_{2,n-1}) + \cdots \right.\\
\left . + 2f(x_{n-2,n-1}) + f(x_{n-1,n-1})\right].
\end{multline*}
We consider how to bound the error of this approximation for integrands with given norm $\Gnorm{p}{(n-1)/2}{f}$.

For simplicity of notation let $x_i=x_{i,n-1}$.  For a fixed $i=1, \ldots, n-2$, consider the error in integrating over the interval $[x_{i-1},x_{i+1}]$ using two trapezoids.  The derivation proceeds by applying applying basic calculus ideas.  First we express the trapezoidal rule approximation as the integral of a piecewise linear function, and then use integration by parts to write the integral in terms of the first derivative of the integrand:
\begin{align*}
\MoveEqLeft{\biggabs{\int_{x_{i-1}}^{x_{i+1}} f(x) \, \dif x - \frac{1}{2(n-1)} [f(x_{i-1}) + 2f(x_{i}) + f(x_{i+1}) ]}}\\
&=\biggl \lvert \int_{x_{i-1}}^{x_{i}} \{f(x) - (n-1)[f(x_{i-1})(x_{i}-x) + f(x_{i})(x-x_{i-1})] \}\, \dif x \\
&\qquad \qquad + \int_{x_{i}}^{x_{i+1}} \{ f(x) - (n-1)[f(x_{i})(x_{i+1}-x) + f(x_{i+1})(x-x_{i})] \} \, \dif x \biggr \rvert \\
&=(n-1) \biggl \lvert \int_{x_{i-1}}^{x_{i}} \biggl \{(x_{i}-x)\int_{x_{i-1}}^x f'(t) \, \dif t -    (x-x_{i-1}) \int_{x}^{x_i} f'(t) \, \dif t  \biggr \}\, \dif x \\
&\qquad \qquad + \int_{x_{i}}^{x_{i+1}} \biggl \{(x_{i+1}-x)\int_{x_{i}}^x f'(t) \, \dif t -    (x-x_{i}) \int_{x}^{x_{i+1}} f'(t) \, \dif t  \biggr \}\, \dif x \biggr \rvert.
\end{align*}
Next the order of integration is switched and the error is expressed as a univariate integral of $f'$ multiplied by a weight:
\begin{align*}
\MoveEqLeft{\biggabs{\int_{x_{i-1}}^{x_{i+1}} f(x) \, \dif x - \frac{1}{2(n-1)} [f(x_{i-1}) + 2f(x_{i}) + f(x_{i+1}) ]}}\\
&=(n-1) \biggl \lvert \int_{x_{i-1}}^{x_{i}} f'(t) \biggl\{ \int_{t}^{x_{i}} (x_{i}-x) \, \dif x  - \int_{x_{i-1}}^{t}  (x-x_{i-1}) \, \dif x \biggr \} \, \dif t \\
&\qquad \qquad + \int_{x_{i}}^{x_{i+1}} f'(t) \biggl \{\int_{t}^{x_{i+1}} (x_{i+1}-x) \, \dif x - \int_{x_i}^{t} (x-x_{i}) \, \dif x  \biggr \}\, \dif t \biggr \rvert \\
&=(n-1) \biggl \lvert \int_{x_{i-1}}^{x_{i}} f'(t) \biggl\{ \frac{(t-x_i)^2-(t-x_{i-1})^2}{2} \biggr \} \, \dif t \\
&\qquad \qquad + \int_{x_{i}}^{x_{i+1}} f'(t) \biggl\{ \frac{(t-x_{i+1})^2-(t-x_{i})^2}{2} \biggr \}\, \dif t \biggr \rvert  \\
&=\biggl \lvert \int_{x_{i-1}}^{x_{i}} f'(t) \biggl ( \frac{x_{i-1}+x_{i}-2t}{2} \biggr ) \, \dif t + \int_{x_{i}}^{x_{i+1}} f'(t) \biggl ( \frac{x_{i}+x_{i+1}-2t}{2} \biggr ) \, \dif t\biggr \rvert .
\end{align*}
Since the weight is odd with respect to the center of the integration interval, one may subtract a constant multiple of this weight without altering the integral, and then apply H\"older's inequality with $p^{-1}+q^{-1}=1$.
\begin{align*}
\MoveEqLeft{\biggabs{\int_{x_{i-1}}^{x_{i+1}} f(x) \, \dif x - \frac{1}{2(n-1)} [f(x_{i-1}) + 2f(x_{i}) + f(x_{i+1}) ]}}\\
&=\biggl \lvert \int_{x_{i-1}}^{x_{i}} \Bigl[f'(t)-\frac{n-1}{2}I(f';x_{i-1},x_{i+1}) \Bigr] \biggl ( \frac{x_{i-1}+x_{i}-2t}{2} \biggr ) \, \dif t \\
& \qquad \qquad + \int_{x_{i}}^{x_{i+1}} \Bigl[f'(t)-\frac{n-1}{2}I(f';x_{i-1},x_{i+1}) \Bigr] \biggl ( \frac{x_{i}+x_{i+1}-2t}{2} \biggr ) \, \dif t\biggr \rvert \\
&=\Bignorm[p,{[x_{i-1},x_{i+1}]}]{f'-\frac{n-1}{2}I(f',x_{i-1},x_{i+1})} \\
&\qquad \qquad \times \biggl\{ \int_{x_{i-1}}^{x_{i}}\abs{\frac{x_{i-1}+x_{i}-2t}{2}}^q \dif t + \int_{x_{i}}^{x_{i+1}}\abs{\frac{x_{i}+x_{i+1}-2t}{2}}^q \dif t \biggr\}^{1/q}
\\ 
&=\left[\frac{4}{(q+1)(2n-2)^{q+1}} \right]^{1/q} \Bignorm[p,{[x_{i-1},x_{i+1}]}]{f'-\frac{n-1}{2}I(f';x_{i-1},x_{i+1})}.
\end{align*}

Finally, these error bounds over the subintervals $[x_{i-1},x_{i+1}]$ are combined to obtain an error bound for the trapezoidal rule, again using H\"older's inequality:
\begin{align*}
\MoveEqLeft{\biggabs{\int_{0}^{1} f(x) \, \dif x - A_n(f)}} \\
& \le \sum_{i=1}^{(n-1)/2}\biggabs{\int_{x_{2i}}^{x_{2i+2}} f(x) \, \dif x - \frac{1}{2(n-1)} [f(x_{2i}) + 2f(x_{2i+1}) + f(x_{2i+2}) ]} \\
& \le \sum_{i=1}^{(n-1)/2}\left[\frac{4}{(q+1)(2n-2)^{q+1}} \right]^{1/q} \Bignorm[p,{[x_{2i},x_{2i+2}]}]{f'-\frac{n-1}{2}I(f';x_{2i},x_{2i+2})} \\
& \le  \left(\frac{1}{q+1}\right)^{1/q} \frac{\Gnorm{p}{(n-1)/2}{f}}{2n-2} .
\end{align*}

\bibliographystyle{elsarticle-num.bst}
\bibliography{FJH22,FJHown22}
\end{document}

