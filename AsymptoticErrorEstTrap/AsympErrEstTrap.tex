\documentclass[]{elsarticle}
\setlength{\marginparwidth}{0.5in}
\usepackage{amsmath,amssymb,amsthm,natbib,mathtools,graphicx}
\input FJHDef.tex

\DeclareMathOperator{\lin}{lin}
\DeclareMathOperator{\up}{up}
\DeclareMathOperator{\lo}{lo}
\DeclareMathOperator{\fix}{non}
\DeclareMathOperator{\err}{err}
\DeclareMathOperator{\tol}{tol}
\newcommand{\fudge}{\mathfrak{C}}

\newtheorem{theorem}{Theorem}
\newtheorem{prop}[theorem]{Proposition}
\newtheorem{lem}{Lemma}
\theoremstyle{definition}
\newtheorem{algo}{Algorithm}
\newtheorem{condit}{Condition}
%\newtheorem{assump}{Assumption}
\theoremstyle{remark}
\newtheorem{rem}{Remark}


\journal{Journal of Complexity}

\begin{document}

\begin{frontmatter}

\title{Asymptotic Error Estimation for Trapezoidal Rule}
\author{Fred J. Hickernell}
\address{Room E1-208, Department of Applied Mathematics, Illinois Institute of Technology,\\ 10 W.\ 32$^{\text{nd}}$ St., Chicago, IL 60616}
\begin{abstract}
\end{abstract}

\begin{keyword}
adaptive \sep cones \sep function recovery \sep integration \sep quadrature
%% keywords here, in the form: keyword \sep keyword

\MSC[2010] 65D05 \sep 65D30 \sep 65G20
%% MSC codes here, in the form: \MSC code \sep code
%% or \MSC[2008] code \sep code (2000 is the default)

\end{keyword}
\end{frontmatter}

\section{Problem}

Consider the integral 
\begin{equation*}
I(f)= \int_0^1 f(x) \, \dif x
\end{equation*}
that is approximated by the trapezoidal rule 
\begin{equation*}
\hI_n(f)= \frac{1}{n} \left[ \frac{1}{2}f(0) + f(1/n) + f(2/n) + \cdots +  f (1-1/n) + \frac{1}{2} f(1)\right].
\end{equation*}
The error is often estimated as 
\begin{equation*}
E_n(f) := I(n) - \hI_n(f) \approx \frac{\hI_n(f)-\hI_{n/2}(f)}{3} =: \hE_n(f).
\end{equation*}
We want to find a bound on $E_n(f)-\hE_n(f)$.


\section{Error for One or Two Trapezoids}

Looking at a piece of the trapezoidal rule, given $a$ and $h$ we have
\begin{align*}
\MoveEqLeft{\int_a^{a+h} f(x) \, \dif x - \frac{h}{2}[f(a)+f(a+h)]}\\
& = f(x) (x-a-h/2) \Bigr \rvert_a^{a+h} - \int_a^{a+h} f'(x) (x-a-h/2) \, \dif x \\
&\qquad \qquad - \frac{h}{2}[f(a)+f(a+h)] \\
& = - \int_a^{a+h} f'(x) (x-a-h/2) \, \dif x \\
& = f'(x) \frac{(x-a)(a+h-x)}{2} \Bigr \rvert_a^{a+h} - \int_a^{a+h} f''(x) \frac{(x-a)(a+h-x)}{2} \, \dif x \\
& = - \int_a^{a+h} f''(x) \frac{(x-a)(a+h-x)}{2} \, \dif x
\end{align*}
Thus, for the integral over $[a,a+2h]$ using one trapezoid, the error is 
\begin{multline} \label{onetraperror}
\int_a^{a+2h} f(x) \, \dif x - h[f(a)+f(a+2h)] \\
= - \int_a^{a+2h} f''(x) \frac{(x-a)(a+2h-x)}{2} \, \dif x,
\end{multline}
while the error of the same integral using two trapezoids is
\begin{subequations} \label{twotraperror}
\begin{align} 
\nonumber
E_2(f) &= \int_a^{a+2h} f(x) \, \dif x - \frac{h}2[f(a)+2f(a+h)+f(a+2h)] \\
\nonumber
& = - \int_a^{a+2h} f''(x) \left\{\left[\frac{(x-a)(a+h-x)}{2}\right] 1_{[a,a+h]}(x) \right . \\
\nonumber
& \qquad \qquad \left . + \left[\frac{(x-a-h)(a+2h-x)}{2}\right] 1_{[a+h,a+2h]}(x) \right\} \, \dif x \\
& = - \int_a^{a+2h} f''(x) \left[\frac{\abs{x-a-h}(h-\abs{x-a-h})}{2}\right] \dif x \\
& = \int_a^{a+2h} f''(x) u(x) \, \dif x,
\intertext{where}
u(x)& = -\left[\frac{\abs{x-a-h}(h-\abs{x-a-h})}{2}\right] = \Theta(h^2).
\end{align}
\end{subequations}
Since $u(x) = \Theta(h^2)$, it follows that $E_2(f)=\Theta(h^3)$, as expected.

An expression for 
\begin{align*}
\hE_2(f) &= \frac{1}{3} \left\{ \frac{h}2[f(a)+2f(a+h)+f(a+2h)] - h[f(a)+f(a+2h)]  \right\} \\
& = \frac{h}{6} [-f(a)+2f(a+h)-f(a+2h)]
\end{align*}
may be obtained from \eqref{onetraperror} and \eqref{twotraperror} as follows:
\begin{align*}
\hE_2(f) &= \int_a^{a+2h} f''(x) v(x) \, \dif x, \\
v(x) &= \frac{1}{3} \left\{ \left[\frac{\abs{x-a-h}(h-\abs{x-a-h})}{2}\right]- \frac{(x-a)(a+2h-x)}{2} \right\} \\
& = \frac{1}{6} \left\{ \abs{x-a-h}(h-\abs{x-a-h}) - (x-a)(a+2h-x) \right\}\\
& = \frac{1}{6} \left\{ \abs{x-a-h}(h-\abs{x-a-h}) - [h^2- \abs{x-a-h}^2] \right\}\\
& = \frac{h}{6} \left\{ \abs{x-a-h} - h \right\}= \Theta(h^2).
\end{align*}
Since $v(x) = \Theta(h^2)$, it follows that $\hE_2(f)=\Theta(h^3)$, as expected.

It is hoped that $E_2(f) - \hE_2(f) = o(h^3)$.  The difference between the true and approximate errors is 
\begin{align*}
E_2(f) - \hE_2(f) &= \int_a^{a+2h} f''(x) w(x) \, \dif x, \\
w(x) &= u(x) - v(x) \\
&= -\left[\frac{\abs{x-a-h}(h-\abs{x-a-h})}{2}\right] -  \frac{h}{6} \left\{ \abs{x-a-h} - h \right\}\\
&=\frac{1}{6} \left[3 (x-a-h)^2 - 4 h \abs{x-a-h} + h^2\right] \\
&=\frac{1}{6} (3 \abs{x-a-h} - h)( \abs{x-a-h} - h) = \Theta(h^2).
\end{align*}
Now let 
\begin{align*}
W(x) &= \int_{a}^x w(t) \, \dif t \\
&=\frac{1}{6} \int_{a}^x \left[3 (t-a-h)^2 - 4 h \abs{t-a-h} + h^2\right] \, \dif t\\
&=\frac{1}{6} \left . \left[(t-a-h)^3 - 2 h \abs{t-a-h}(t-a-h) + h^2 (t-a-h)\right] \right \rvert_a^x \\
&=\frac{1}{6}  \left[(x-a-h)^3 - 2 h \abs{x-a-h}(x-a-h) + h^2 (x-a-h)\right] \\
&=\frac{1}{6}  (x-a-h)\left[(x-a-h)^2 - 2 h \abs{x-a-h} + h^2\right] \\
&=\frac{1}{6}  (x-a-h)(\abs{x-a-h} - h)^2 = \Theta(h^3).
\end{align*}
Note that $W(a)=W(a+2h)=0$.  Thus,
\begin{align*}
E_2(f) - \hE_2(f) &= \int_a^{a+2h} f''(x) w(x) \, \dif x, \\
&= f''(x) W(x) \Bigr \rvert_a^{a+h} - \int_a^{a+2h} f'''(x) W(x) \, \dif x \\
&= - \int_a^{a+2h} f'''(x) W(x) \, \dif x, \\
\abs{E_2(f) - \hE_2(f)} &=\abs{\int_a^{a+2h} f'''(x) W(x) \, \dif x} \\
&\le \sup_{a \le x \le a+2h} \abs{W(x)} \times \int_a^{a+2h} \abs{f'''(x)} \, \dif x.
\end{align*}
Noting that $\abs{W(x)}$ attains its maximum where $w(x)$ vanishes, i.e., at $x=a+h \pm h/3$, and that
\[
\abs{W(a+h \pm h/3)}= \frac{1}{6} \times \frac{h}{3} \times \left(\frac{2h}{3}\right)^2 = \frac{2h^3}{81}
\]
it follows that 
\begin{equation*}
\abs{E_2(f) - \hE_2(f)} \le \frac{2h^3}{81} \int_a^{a+2h} \abs{f'''(x)} \, \dif x = \Theta(h^4).
\end{equation*}

If $f^{(4)}$ is integrable, then one may bound the difference of the true and approximate error better as follows.  Let 
\begin{align*}
\tW(x) &= \int_{a}^x W(t) \, \dif t \\
&=\frac{1}{6} \int_{a}^x  (x-a-h)(\abs{x-a-h} - h)^2 \, \dif t\\
&=\frac{1}{6} \left . \left[\frac{1}{4}(t-a-h)^4 - \frac{2h}3  \abs{t-a-h}^3 + \frac{h^2}2 (t-a-h)^2\right] \right \rvert_a^x \\
&=\frac{1}{72} \left[3(t-a-h)^4 - 8h\abs{t-a-h}^3 + 6h^2 (t-a-h)^2 - h^4\right],
\end{align*}
and note that $\tW(a)=\tW(a+2h)=0$.  Then it follows that
\begin{align*}
E_2(f) - \hE_2(f) &= - \int_a^{a+2h} f'''(x) W(x) \, \dif x, \\
&= -f'''(x) \tW(x) \Bigr \rvert_a^{a+h} + \int_a^{a+2h} f^{(4)}(x) \tW(x) \, \dif x \\
&= \int_a^{a+2h} f^{(4)}(x) \tW(x) \, \dif x, \\
\abs{E_2(f) - \hE_2(f)} &=\abs{\int_a^{a+2h} f^{(4)}(x) \tW(x) \, \dif x} \\
&\le \sup_{a \le x \le a+2h} \abs{\tW(x)} \times \int_a^{a+2h} \abs{f^{(4)}(x)} \, \dif x.
\end{align*}
Noting that $\abs{\tW(x)}$ attains its maximum where $W(x)$ vanishes, i.e., at $x=a+h$, and that
\[
\abs{\tW(a+h)}= \frac{h^4}{72},
\]
it follows that 
\begin{equation*}
\abs{E_2(f) - \hE_2(f)} \le \frac{h^4}{72} \int_a^{a+2h} \abs{f^{(4)}(x)} \, \dif x = \Theta(h^5).
\end{equation*}

\section{Error for the Whole Integral}
Applying the work from the previous section with $h=1/n$, and $a=x_i$ for $i=0, 2, 4 \ldots, n-2$ it follows that
\begin{align*}
\abs{E_n(f) - \hE_n/2(f)} & \le \frac{2h^3}{81} \sum_{j=0}^{n/2-1} \int_{x_{2j}}^{x_{2j+2}} \abs{f'''(x)} \, \dif x \\
& = \frac{2}{81n^3} \int_{0}^{1} \abs{f'''(x)} \, \dif x  = \frac{2}{81n^3} \norm[1]{f'''},  \\
\abs{E_n(f) - \hE_n/2(f)} & \le \frac{h^4}{72} \sum_{j=0}^{n/2-1} \int_{x_{2j}}^{x_{2j+2}} \abs{f^{(4)}(x)} \, \dif x\\
& = \frac{1}{72 n^4} \int_{0}^{1} \abs{f^{(4)}(x)} \, \dif x  = \frac{1}{72 n^4}\norm[1]{f^{(4)}}.
\end{align*}





\bibliographystyle{spbasic}
\bibliography{FJH22,FJHown22}
\end{document}
