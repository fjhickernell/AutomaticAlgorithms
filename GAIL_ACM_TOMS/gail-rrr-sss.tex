\documentclass[acmtoms]{doc_acmtrans2m}

\acmVolume{xx}
\acmNumber{x}
\acmYear{xx}
\acmMonth{xx}

\usepackage{amsmath,amssymb}
\usepackage{graphicx}
\usepackage{xspace}
\usepackage{listings}
\usepackage{longtable}
\usepackage{enumerate}
\usepackage{wrapfig}
\usepackage{verbatim}   
\usepackage{color}

\newcommand{\note}[1]{ {\textcolor{red}    { #1 }}}
 
\input{rrr-sss-macros.tex}
 

\newcommand{\BibTeX}{{\rm B\kern-.05em{\sc i\kern-.025em b}
\kern-.08emT\kern-.1667em\lower.7ex\hbox{E}\kern-.125emX}}

\usepackage[vlined,ruled,nofillcomment,linesnumbered]{algorithm2e}
\SetAlgoSkip{}\SetAlgoInsideSkip{smallskip}
\newcommand{\commentboxA}[1]{\makebox[48ex][l]{#1}}
\newcommand{\commentboxB}[1]{\makebox[44.3ex][l]{#1}}
\newenvironment{algo}[1]
{\begin{algorithm}[#1]%
    \small%
    \DontPrintSemicolon%
    \SetArgSty{texttsf}%
    \SetTitleSty{textsf}{}%
    \SetNlSty{textrm}{}{}%
    \SetKwInput{Inputs}{input}%
    \SetKwInput{Outputs}{output}%
    \SetKwData{Converged}{converged}%
    \SetKwComment{tcc}{[}{]}
}
{\end{algorithm}}


\newenvironment{ttlongtable}[1]%
{\ttfamily \begin{longtable}{#1}}%
{\end{longtable}}

 
\title{ALGORITHM xxx:
      \\Guaranteed Automatic Integration Library (GAIL)}
\author{SOU-CHENG T.~CHOI
     \\NORC at the University of Chicago/Illinois Institute of Technology
     \\YUHAN DING
     \\FRED J.~HICKENRELL
     \\LAN JIANG
     \\LLU\'{I}S ANTONI JIM\'{E}NEZ RUGAMA
     \\YIZHI ZHANG 
     \and XUAN ZHOU
     \\Illinois Institute of  Technology }

\markboth{S.-C. Choi, Y. Ding,  F. Hickernell, L. Jiang, L. Rugama, Y.~Zhang, and X.~Zhou}
         {ALGORITHM xxx: GAIL}
 

%%%%%%%%%%%%%%%%%%%%%%%%%%%%%%%%%%%%%%%%%%%%%%%%%
\begin{abstract}  s
%%%%%%%%%%%%%%%%%%%%%%%%%%%%%%%%%%%%%%%%%%%%%%%%%

Automatic and adaptive numerical approximation and integration of functions in a cone
with guarantee of accuracy is a relatively new paradigm. Our purpose
is to create a reliable, open-source MATLAB package,
Guaranteed Automatic Integration Library (GAIL), following the
philosophy of reproducible research championed by Jon Claerbout and
David Donoho, and the principles of robust scientific software
development.  In this paper, we  review some of the key ideas and
features in the ongoing theoretical and computational developments of our
guaranteed algorithms in GAIL, which includes one-dimensional function
approximation using linear splines, one-dimensional numerical
integration using trapezoidal and Simpson's rules, and last but not least, estimation
of mean of a random variable by (Quasi) Monte Carlo methods.
 
\end{abstract}


% References:
% (1) http://toms.acm.org/Authors.html
% (2) http://www.computer.org/portal/web/publications/acmmath#G1
\category{G.1.3}{Numerical Analysis}{Numerical Linear Algebra}
                [linear systems (direct and iterative methods)]

\category{G.3}{Mathematics of Computing}{Probability and Statistics}
              [statistical computing; statistical software]

\category{G.m}{Mathematics of Computing}{Miscellaneous}
              [FORTRAN program units]

\terms{Algorithms}

\keywords{automatic, adaptive, fixed-cost, guaranteed, 
Monte Carlo, Quasi Monte Carlo, linear splines, reliable, reproducible
%
\newline\newline  Version 1 of \today}


\begin{document}

\lstdefinestyle{numbers}
{numbers=left, stepnumber=1, numberblanklines=false,
numberstyle=\tiny, numbersep=8pt}
\lstdefinestyle{nonumbers}
{numbers=none}

\lstset{
language=[90]Fortran, % for Fortran code listing
basicstyle=\small,    % font size
style=numbers,        % line number. To toggle, use: style=numbers
                      % or style=nonumbers
emptylines=*1,        % not printing any block with more than 1
                      % empty line
breaklines=true,      % sets automatic line breaking
escapeinside=<>,      % for doing latex \vdots
%framextopmargin=5mm, % space between frame's top and first code line
framesep=2.5mm,       % framextopmargin and framexbottommargin.
                      % I am using this for its side effect of
                      % increased space between caption and frame's
                      % top line
%rulesep=5mm          % if frame uses double lines, then this control
                      % space between the double lines
}


\begin{bottomstuff}
This work was supported in part by grants from
the National Science Foundation under grant NSF-DMS-1115392, and
the Office of Advanced Scientific Computing Research, Office of Science, 
U.S. Department of Energy, under contract DE-AC02-06CH11357. 

Authors' addresses:
%
   S.-C.~T.~Choi, NORC at the University of Chicago,
   Chicago, IL 60603; email: sctchoi@uchicago.edu;
%  
   Y. Ding, Illinois Institute of Technology (IIT),
   Chicago, IL 60616; email: xxx@iit.edu;  
%
   F.~J.~Hickernell, IIT; email: hickernell@iit.edu; 
%
   L.~Jiang, IIT; email: xxx@iit.edu; 
%
   L.~A.~J.~Rugama, IIT; email: xxx@iit.edu; 
%
   Y.~Zhang, IIT; email: xxx@iit.edu; 
%
   X.~Zhou, IIT; email: xxx@iit.edu; 
\end{bottomstuff}


\maketitle

\clearpage


%%%%%%%%%%%%%%%%%%%%%%%%%%%%%%%%%%%%%%%%%%%%%%%%%
\section{Introduction} \label{sec:intro}  
%%%%%%%%%%%%%%%%%%%%%%%%%%%%%%%%%%%%%%%%%%%%%%%%%

 
 
\subsection{Reliable Reproducible Research}  



\subsection{Mathematical Problems and Solutions} \label{sec:prob}
 

Function recovery, Global Minimization (?), Numerical integration (one dimension)

Numerical integration (higher dimensions)

\subsection{Review} \label{sec:lit}
 

Existing work, strengths, and limitations

\subsection{Notation} \label{sec:not}

\subsection{Outline} \label{sec:lit}

%%%%%%%%%%%%%%%%%%%%%%%%%%%%%%%%%%%%%%%%%%%%%%%%%
\section{Iterative Numerical Methods} 
\label{sec:meth}
%%%%%%%%%%%%%%%%%%%%%%%%%%%%%%%%%%%%%%%%%%%%%%%%%
 
\subsection{Cones} \label{sec:cones}


\subsection{Discretization} \label{sec:dis}
Splines for function recovery, Global Minimization (?), Numerical integration (one dimension)
 
Sobol points for numerical integration (higher dimensions)

\subsection{Initial Values}

\subsection{Stopping Conditions}

%************************************************
\section{Implementations} 
\label{sec:impl}
%************************************************


\subsection{Main functions and User Interfaces} \label{sec:ui}  
 
 \texttt{funappx\_g.m, etc}
 
 Function handle of function
 
 Structure of inputs and outputs
 
 Main API patterns
 
 (No graphical ui yet)
 
\subsection{Input parsing} 
  

\subsection{Documentation} \label{sec:doc}
  
  Inline comments
  
  Text
  
  Searchable HTML

\subsection{Tests} \label{sec:tests}  
 

Unit tests

Stress tests
 

\subsection{Repository, Release, and Installation} 
 

%************************************************
\section{Key Properties}\label{sec:prop}
%************************************************  
\subsection{Automatic}

\subsection{Adaptive}

\subsection{Fixed Costs}

\subsection{Complexity}

\subsection{Convergence}

\subsection{Discrepency} 





%************************************************
\section{Examples/Applications}
\label{sec:ex}
%************************************************
  
  
%************************************************
\section{Conclusion}
\label{sec:con}
%************************************************

 
\subsection{Limitation}

 
\subsection{Future Work}

%%%%%%%%%%%%%%%%%%%%%%%%%%%%%%%%%%%%%%%%%%%%%%%%%%
% REFERENCES
%%%%%%%%%%%%%%%%%%%%%%%%%%%%%%%%%%%%%%%%%%%%%%%%%%%

\frenchspacing


\bibliographystyle{plain} %abbrv
\bibliography{rrr-sss-refs}

\vfill
%************************************************
\subsection*{Acknowledgments}
%************************************************
 

\bigskip \bigskip \bigskip

{\small }

\end{document}
